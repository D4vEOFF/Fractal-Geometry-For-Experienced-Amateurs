\section{Stručně k zápisu programů}\label{sec:zapis-programu}

V době psaní tohoto textu existuje mnoho programovacích jazyků\index{jazyk!programovací}\index{programovací jazyk} a nejspíše lze bezpečně předpokládat, že budou přibývat další. Nikoho tak nejspíše nepřekvapí, že vzhledem k současnému (dosti rychlému) vývoji v oblasti informatiky mnoho jazyků, které dříve byly považovány za nové a inovativní, postupně zastaraly a jiné pro mnoho jedinců dokonce upadly v zapomnění. Avšak jiné naopak si svoji pozici drží dodnes. Pro účely tohoto textu byl v rámci praktických ukázek, které uvidíte, zvolen jazyk \textbf{Python}\index{Python}, neboť jeho syntaxe není složitá\footnote{Složitost programovacího jazyka je, z pochopitelných důvodů, dosti subjektivní pojem, neboť závisí i na zkušenostech programátora.} a zároveň tak není příliš obtížné si v mnoha případech domyslet význam jednotlivých příkazů\footnote{Samozřejme nelze v tomto ohledu mluvit za každého (potenciálního) čtenáře. Pokud by tak kdykoliv vznikla nějaká nejasnost ohledně významu použitých příkazů, lze se podívat na stránky oficiální dokumentace jazyka Python: \url{https://docs.python.org}}. Zároveň však poznamenejme, že stejně jako v případě matematické části tohoto textu, i zde budeme pracovat s předpokladem, že čtenář je seznámen se základními koncepty programování a algoritmizace všeobecně. Nebudeme se zde tedy řešit, co je to proměnná\index{proměnná}, pole\index{pole} (resp. v Pythonu seznamem\index{seznam}), funkce, podmínky nebo základy objektově orientovaného programování.

Zároveň bychom neměli zapomínat na zájemce používající jiné programovací jazyky. Proto kromě praktických ukázek si prezentované algoritmy uvedeme i pomocí tzv. \emph{pseudokódu}\index{pseudokód}. Pseudokód nepředstavuje sám o sobě žádnou formu programovacího jazyka. Jedná se čistě o abstraktní popis psaný především pro člověka, který lze však s minimálním úsilím přepsat do libovolného programovacího jazyka. Jednoduchým příkladem pseudokódu je např. \ref{alg:ukazka-pseudokodu}.
\begin{algorithm}
    \KwIn{Seznam čísel $x_1,x_2,\ldots,x_n$.}
    $\text{max}\gets x_1$\\
    \For{$i=1,2,\ldots,n$}{
        \If{$x_i>\textup{max}$}{
            $\textup{max}\gets x_i$
        }
    }
    \Return{\textup{max}}
    \caption{Ukázkový pseudokód (hledání minima)}
    \label{alg:ukazka-pseudokodu}
\end{algorithm}

Nejdříve si vysvětleme některé značení:
\begin{itemize}
    \item symbol $\gets$ používáme pro operaci přiřazení hodnoty (místo standardního $=$),
    \item naopak symbol $=$ budeme používat ve smyslu porovnávání hodnot,
    \item indexy (u řetězců a polí) v rámci pseudokódu budeme počítat od jedné.
\end{itemize}
Implementace zmíněného algoritmu např. právě v jazyce Python si čtenář může prohlédnout u programu \ref{prog:ukazka-implementace-pseudokodu}.
\begin{program}[h]
\begin{lstlisting}[style=python]
def findMax(numbers: list) -> int:
    maximum = numbers[0]
    for i in range(len(numbers)):
        number = numbers[i]
        if number > maximum:
            maximum = number
    return maximum
\end{lstlisting}
    \caption{Možná implementace algoritmu \ref{alg:ukazka-pseudokodu}}
    \label{prog:ukazka-implementace-pseudokodu}
\end{program}
Pochopitelně se v konkrétní implementaci mohou vyskytovat různé odchylky. Např. v programu \ref{prog:ukazka-implementace-pseudokodu} využíme proměnnou \texttt{number}, kterou bychom jistě mohli vypustit a pracovat přímo se seznamem \texttt{numbers}, resp. jeho $i$-tým prvkem \texttt{numbers[i]}. Nebo bychom například nemuseli program vůbec zapisovat jako funkci, či bychom mohli např. jinak pojmenovat proměnné (viz program \ref{prog:ukazka-jine-implementace-pseudokodu}).
\begin{program}[h]
\begin{lstlisting}[style=python]
count = int(input("Enter number count: "))
if (count <= 0): exit(0)

numbers = [int(input("Enter a number: "))
for _ in range(count)]

max_value = numbers[0]
for i in range(len(numbers)):
    if numbers[i] > max_value: max_value = numbers[i]

print(f"Max value: {max_value}")
\end{lstlisting}
    \caption{Jiná možná implementace algoritmu \ref{alg:ukazka-pseudokodu}}
    \label{prog:ukazka-jine-implementace-pseudokodu}
    \end{program}
To jsou však v celkovém kontextu pouhé drobnosti. V rámci textu se ovšem budeme snažit držet jednotné konvence, tedy že programy budeme vždy psát jako funkce/procedury/metody a proměnné budeme pojmenovávat vždy v angličtině, neboť je to při programování zkrátka zvyklost.
