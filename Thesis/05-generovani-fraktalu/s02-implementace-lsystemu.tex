\section{Implementace L-systémů a~želví grafiky}\label{sec:implementace-lsystemu-a-zelvi-grafiky}

Na základní princip L-systémů\index{L-systém} jsme se podívali v~části~\ref{sec:L-systemy}. Jejich implementace je do jisté míry přímočará. Řešení je potřeba rozdělit na dvě části: implementace \emph{samotného L-systému} a~\emph{želví grafiky}\index{želví grafika}.

\subsection{L-systémy}\label{subsec:implementace-lsystemu}

L-systém nepředstavuje nikterak složitou matematickou strukturu. Z~definice (viz~\ref{def:lsystem}) je potřeba znát pouze používané \emph{neterminály}\index{neterminál}, \emph{axiom}\index{axiom} a~seznam přepisovacích pravidel. O~to jednodušší je situace, započítáme-li fakt, že neterminální symboly (resp. jejich význam) v~případě námi používaných L-systémů jsou pevně dané, tedy není třeba je explicitně uvádět v~definici. L-systém lze tak implementovat jako jednoduchou třídu s~atributy \texttt{word} obsahující aktuální slovo po $k$-té iteraci a~slovník pravidel \texttt{rules} (viz program~\ref{prog:konstruktor-lsystem}).
\begin{program}[h]
    \begin{lstlisting}[style=python]
class LSystem:
    def __init__(self, axiom: str, rules: dict) -> None:
        self._word = axiom
        self._rules = rules
\end{lstlisting}
    \caption{Konstruktor třídy pro L-systém}
    \label{prog:konstruktor-lsystem}
\end{program}
Slovní pravidel \texttt{rules} má jednoduchou strukturu. Klíče tvoří levé strany pravidel a~k nim přiřazené hodnoty naopak tvoří pravé strany pravidel. Jeho vzhled může vypadat např. takto:
\begin{verbatim}
rules = {
    "X": "F-[[X]+X]+F[+FX]-X",
    "F": "FF"
}
\end{verbatim}
Poměrně zásadní pro nás však bude především metoda pro aplikaci jednotlivých pravidel. Pro další výklad si však zavedeme pohodlnější zápis řetězců, který je v~programování zcela běžný.
\begin{definition}\label{def:index-retezce}
    Nechť $\alpha=x_1x_2\ldots x_n$ je slovo nad~libovolnou abecedou $\Sigma\neq\emptyset$. Pak pro každé $1\leqslant i\leqslant n$ definujeme $\alpha[i]=x_i$.
\end{definition}
Myšlenka je velice intuitivní. Obecně máme-li řetězec $w$ po $m$-té iteraci a~množinu přepisovacích pravidel $P\subseteq\set{a\to\alpha\mid a\in V,\;\alpha\in V^*}$, kde $V$ je abeceda, pak stačí pro každý znak $w[i]$, kde $1\leqslant i\leqslant n$, pouze zkontrolovat, zda není na levé straně nějakého pravidla v~$P$. Pokud ano, dojde k~aplikaci příslušného pravidla\footnote{Technicky vzato jsme z~formálních důvodů v~definici L-systému~\ref{def:lsystem} přidali i~pravidla tvaru $a\to a$, aby nedošlo k~situaci, že pro $a$ neexistuje pravidlo. Avšak z~hlediska praktické implementace toto není překážkou, neboť v~případě absence takového pravidla jednoduše symbol přeskočíme.}. Viz pseudokód~\ref{alg:iterace-slova-lsystem}.
\begin{algorithm}[h]
    \KwIn{Množina pravidel $P$, slovo $w$, číslo $k\in\N$}
    $w_{\textup{prev}}\gets\lambda,w_{\textup{new}}\gets w$\;
    \For{$m=1,2,\ldots,k$}{
        $w_{\textup{prev}}\gets w_{\textup{new}}$\;
        $w_{\textup{new}}\gets\lambda$\;
        \For{$i=1,2,\ldots,|w_{\textup{prev}}|$}{
            \If{\textup{existuje pravidlo tvaru $(w_{\textup{prev}}[i]\to\alpha)\in P$}}{
                $w_{\textup{new}}\gets w_{\textup{new}}[1]\dots w_{\textup{new}}[i-1]\alpha$\;
            }
            \Else{
                $w_{\textup{new}}\gets w_{\textup{new}} w[i]$\;
            }
        }
    }
    \Return{$w_{\textup{new}}$}\;
    \KwOut{Slovo $w_{\textup{new}}$ odvozené po $k$ iteracích ze slova $w$}
    \caption{Algoritmus pro $k$-tou iteraci slova $w$}
    \label{alg:iterace-slova-lsystem}
\end{algorithm}
Implementace je, vzhledem k~dostupným funkcím v~Pythonu, až překvapivě jednoduchá. O~tom se čtenář může přesvědčit sám v~případě kódu~\ref{prog:iterace-slova-lsystem}.
\begin{program}[h]
\begin{lstlisting}[style=python]
def iterate(self, iteration_count: int) -> None:
    for _ in range(iteration_count):
        self._word = self._word.translate(str.maketrans(self._rules))
\end{lstlisting}
    \caption{Implementace algoritmu~\ref{alg:iterace-slova-lsystem}}
    \label{prog:iterace-slova-lsystem}
\end{program}
Pojďme si stručně rozebrat použité funkce, resp. metody, v~programu~\ref{prog:iterace-slova-lsystem}.
\begin{itemize}
    \item \texttt{str.maketrans} vytvoří ze zadaného slovníku překladovou tabulku pro metodu \texttt{translate}. Její struktura odpovídá slovníku obsahující dvojice \emph{(Unicode\index{Unicode} hodnota, znak)}.
    \item \texttt{translate} nahradí každý ze znaků řetězcem uvedeným v~překladové tabulce.
\end{itemize}
Tímto způsobem lze implementovat třídu, kde vygenerujeme příslušný řetězec na základě poskytnutých přepisovacích pravidel, a~který následně budeme interpretovat pomocí želví grafiky\index{želví grafika}.

\subsection{Želví grafika}\label{subsec:implementace-zelvi-grafiky}

Druhou částí je naprogramování želví grafiky. Nyní pracujeme se scénářem, že máme vygenerovaný příslušný řetězec znaků $w$, jehož znaky chceme interpretovat. Za účelem zjednodšení se pokusíme striktně oddělit samotnou \emph{geometrickou interpretaci řetězce} od jeho \emph{grafické interpretace}.

Pro připomenutí významů jednotlivých symbolů doporučujeme se znovu podívat do tabulek~\ref{table:vyznam-symbolu-zelva} a~\ref{table:vyznam-symbolu-zelva-zasobnik}. Nejdříve si však ujasněme, jaké informace si potřebujeme o~želvě uchovávat.
\begin{itemize}
    \item Vzdálenost $d$, o~kterou se želva při každém kroku posune,
    \item aktuální pozice želvy $(x,y)$,
    \item úhel $\alpha\in\langle 0,2\pi)$ udávající směr želvy
    \item přírůstek úhlu $\delta$,
    \item seznam nakreslených úseček prezentovaných uspořádanými čtveřicemi
    \[(x_0,y_0,x_1,y_1),\]
    kde $(x_0,y_0)$ a~$(x_1,y_1)$ jsou souřadnice počátečního, resp. koncového bodu.
\end{itemize}
Podobně jako v~případě L-systému, i~zde můžeme želvu reprezentovat jako třídu (viz program~\ref{prog:konstruktor-zelva}).
\begin{program}[h]
\begin{lstlisting}[style=python]
class Turtle:
    def __init__(self, step: float, position: Vector = Vector(0, 0), angle: float = 0) -> None:
        self._position = position
        self._step = step
        self._angle = (angle % 360) * math.pi / 180
        self._pen_down = False
        self._lines = []

        self._x_min, self._y_min = position.x, position.y
        self._x_max, self._y_max = position.x, position.y
\end{lstlisting}
    \caption{Konstruktor třídy pro želvu}
    \label{prog:konstruktor-zelva}
\end{program}

V tomto případě zvolíme při implementaci želvy následující strategii. Představíme si ji tak, že na sobě připevněné pero a~budeme si pouze pamatovat, zda je či není v~danou chvíli položeno na plátně. Pokud ano a~želva provede krok vpřed, nakreslí za sebou úsečku.

Všechny atributy jsou vysvětleny níže.
\begin{itemize}
    \item \texttt{self.\_position} uchovává pozici želvy $(x,y)$,
    \item \texttt{self.\_step} reprezentuje velikost kroku $d$,
    \item \texttt{self.\_angle} je počáteční úhel otočení želvy přepočítaný v~radiánech,
    \item \texttt{self.\_pen\_down} udává, zda je pero položeno na plátně.
    \item \texttt{self.\_lines} ukládá seznam dosud nakreslených úseček.
\end{itemize}
V konstruktoru třídy \texttt{Turtle} se navíc nachází soukromé atributy
\begin{center}
    \texttt{self.\_x\_min}, \texttt{self.\_x\_max}, \texttt{self.\_y\_min} a~\texttt{self.\_y\_max}.
\end{center}
Ty nám budou sloužit pro pozdější vykreslování výsledného útvaru. Průběžně si v~nich budeme uchovávat minimální, resp. maximální souřadnici $x$ a~$y$ ze všech dosud vygenerovaných úseček. Tyto informace se nám budou hodit pro záverečnou manipulaci s~výsledným obrazcem.

Jednotlivé akce želvy lze v~rámci třídy naprogramovat poměrně snadno. V~případě otočení se jedná o~jednoduchou práci s~úhly. Totéž lze říci i~o posunutí želvy ve směru její orientace, neboť její novou pozici stanovíme pomocí vzorce
\[(x^\prime,y^\prime)=d\cdot(\cos\varphi,\sin\varphi),\]
kde $\varphi$ je úhel otočení želvy. Navíc v~závislosti na tom, zda je pero položeno na plátně, buď s~pohybem želvy současně vytvoříme novou úsečku, nebo pouze posuneme želvu. Pseudokód si lze prohlédnout v~ukázce~\ref{alg:posunuti-zelvy-a-usecka} a~výslednou implementaci společně s~přepočítáváním minimálních a~maximálních souřadnic v~programu~\ref{prog:implementace-kroku-zelvy}.
\begin{algorithm}[h]
    \KwIn{Pozice želvy $(x,y)$, úhel $\varphi$, délka kroku $d$, seznam úseček $L$}
    $x^\prime\gets d\cos(\varphi),y^\prime\gets d\sin(\varphi)$\;
    \If{\textup{je pero položeno na plátně}}{
        Do seznamu $L$ přidej úsečku $\ell=(x,x^\prime,y,y^\prime)$\;
    }
    \Return{$(x^\prime,y^\prime,\ell)$}\;
    \KwOut{Nová pozice želvy s~nakreslenou úsečkou $(x^\prime,y^\prime,\ell)$}
    \caption{Posunutí želvy ve směru a~nakreslení úsečky}
    \label{alg:posunuti-zelvy-a-usecka}
\end{algorithm}
\begin{program}[h]
\begin{lstlisting}[style=python]
def forward(self) -> None:
    prev = self._position
    self._position += self._step * Vector(math.cos(self._angle), math.sin(self._angle))

    if self._x_min > self._position.x: self._x_min = self._position.x
    if self._y_min > self._position.y: self._y_min = self._position.y
    if self._x_max < self._position.x: self._x_max = self._position.x
    if self._y_max < self._position.y: self._y_max = self._position.y

    if self._pen_down:
        self._lines.append([prev, self._position])
\end{lstlisting}
    \caption{Implementace kroku želvy}
    \label{prog:implementace-kroku-zelvy}
\end{program}

\subsection{Zásobník}\label{subsec:zasobnik}

S želví grafikou jsme kromě základní čtveřice symbolů zavedli i~dvojici symbolů \texttt{[} a~\texttt{]}, které složily pro uložení aktuálního stavu želvy na vrchol zásobníku.

Implementace zásobníku je typicky jednou ze základních programovacích úloh a~nejspíše není nikterak složité si jeho implementaci rozmyslet. Pro její vzorovou ukázku viz~\ref{prog:implementace-zasobniku}.
\begin{program}[h]
\begin{lstlisting}[style=python]
class Stack:
    def __init__(self, items: list = []) -> None:
        self._items = items

    def push(self, item: object) -> None:
        self._items.insert(0, item)
    
    def pop(self) -> object:
        item = self._items[0]
        del self._items[0]
        return item  
\end{lstlisting}
    \caption{Implementace zásobníku}
    \label{prog:implementace-zasobniku}
\end{program}

\subsection{Vykreslení obrazce pomocí knihovny \texttt{tkinter}}\label{subsec:vykresleni-obrazce}

Záverečná část implementace želví grafiky spočívá již pouze v~samotné implementaci řetězce, který jsme vygenerovali pomocí našeho L-systému (viz podsekce~\ref{subsec:implementace-lsystemu}). Začneme však záležitotí, která se spíše týká estetické stránky, a~to manipulací s~výsledným obrazcem.

Existují různé knihovny pro práci s~grafickým rozhraním\index{grafické rozhraní}\index{rozhraní!grafické} v~jazyce Python. My si pro jednoduchost zvolíme knihovnu \emph{tkinter}\index{tkinter}. S~prominutím zde nebudeme rozebírat její kompletní obsah\footnote{Pokud by čtenáře zajímala dokumentace, lze ji nalézt např. zde: \url{https://docs.python.org/3/library/tkinter.html\#module-tkinter}.}, využijeme pouze některé funkce. Uveďme si však alespoň nutný základ, který budeme potřebovat.

Základem je vytvoření grafického okna, do něhož budeme vykreslovat výsledné obrazce. K~tomu je potřeba zavolat konstruktor \texttt{Tk()}. Dále pomocí metod \texttt{geometry} a~\texttt{title} lze nastavit jeho šířku a~titulek.

Do samotného okna však ještě nelze nic vykreslovat. K~tomu je potřeba vytvořit v~okně plátno (anglicky \emph{canvas}) pomocí třídy \texttt{Canvas}. Do okna jej následně vložíme zavoláním metody \texttt{pack()} (viz ukázka~\ref{prog:tkinter-zaklad}).

Okno nakonec spustíme pomocí metody \texttt{mainloop()}, jejíž volání skočí až po jeho uzavření.
\begin{program}[h]
\begin{lstlisting}[style=python]
import tkinter

window = tkinter.Tk()
window.geometry(f"1280x720")
window.title(f"Ukazkove okno")

canvas=tkinter.Canvas(window, width=600, height=800)
canvas.pack()

window.mainloop()
\end{lstlisting}
    \caption{Základní práce s~knihovnou \texttt{tkinter}}
    \label{prog:tkinter-zaklad}
\end{program}

Knihovna \texttt{tkinter} nabízí mnoho dalších funkcí a~toto nebyl ani zdaleka základ v~pravém slova smyslu, nicméně bude to pro naše potřeby dostačující.

Problém, s~nímž se však nyní potřebujeme vypořádat, je zarovnání obrázku na střed grafického okna, neboť počátek soustavy souřadnic je umístěn v~levém horním rohu plátna, tedy celé plátno tvoří její \textbf{první kvadrant}. To však znamená, že např. bod $(x,y)$, kde např. $x<0$, nebude viditelný. Pro tento účel jsme si však uchovávali minimální a~maximální hodnoty obou souřadnic. Díky nim můžeme následně provést posunutí celého obrazce do požadované polohy.

Střed výsledného (zatím nevykresleného) obrazce dokážeme jednoduše spočítat. Známe-li minimální, resp. maximální hodnoty obou souřadnic, které označíme $x_{\text{min}},x_{\text{max}},y_{\text{min}}$ a~$y_{\text{max}}$, pak střed $(x_c,y_c)$ určíme jako
\[(x_c,y_c)=\left(\dfrac{x_{\text{max}}+x_{\text{min}}}{2},\dfrac{y_{\text{max}}+y_{\text{min}}}{2}\right).\]
Pokud budeme tedy chtít posunout střed obrazce obecně do bodu $(x,y)$, pak vektor posunutí bude jednoduše $\vec{v}=(x,y)-(x_c,y_c)$. Každou z~úseček, kterou jsme vygenerovali pomocí želví grafiky, stačí posunout o~spočítaný vektor $\vec{v}$ (viz algoritmus~\ref{alg:posun-obrazce}). Implementaci si lze opět prohlédnout v~ukázce programu~\ref{prog:posun-obrazce}.
\begin{algorithm}[h]
    \KwIn{Seznam úseček $L$, nový střed $(x,y)$}
    $(x_c,y_c)\gets((x_{\text{max}}+x_{\text{min}})/2,(y_{\text{max}}+y_{\text{min}})/2)$\;
    $\vec{v}\gets(x,y)-(x_c,y_c)$\;
    $L^\prime\gets\emptyset$\;
    \ForEach{\textup{úsečka $\ell=(x_0,y_0,x_1,y_1)$ v~seznamu $L$}}{
        $(x_0^\prime,y_0^\prime)\gets(x_0,y_0)+\vec{v}$\;
        $(x_1^\prime,y_1^\prime)\gets(x_1,y_1)+\vec{v}$\;
        Do $L^\prime$ přidej úsečku $\ell^\prime=(x_0^\prime,y_0^\prime,x_1^\prime,y_1^\prime)$\;
    }
    \Return{$L^\prime$}\;
    \KwOut{Posunuté úsečky $L^\prime$}
    \caption{Algoritmus pro posun obrazce}
    \label{alg:posun-obrazce}
\end{algorithm}
\begin{program}[h]
\begin{lstlisting}[style=python]
def center_to(self, xc: float, yc: float) -> None:
    lines_center = Vector((self._x_min + self._x_max) // 2, (self._y_min + self._y_max) // 2)
    translation_vector = Vector(xc, yc) - lines_center

    for line in self._lines:
        line[0] += translation_vector
        line[1] += translation_vector
\end{lstlisting}
    \caption{Posunutí středu obrazce do zvoleného bodu}
    \label{prog:posun-obrazce}
\end{program}
Nyní jsme již zcela připraveni provést závěrečné vykreslení obrazce. Stačí procházet vygenerovaný řetězec znak po znaku a~simulovat pohyb želvy, která nám bude do příslušného seznamu ukládat postupně vygenerované úsečky. Následně náš obrazec zarovnáme na střed okna a~každou z~úseček vykreslíme na plátno (viz algoritmus~\ref{alg:simulace-pohybu-zelvy} a~ukázka implementace~\ref{prog:simulace-pohybu-zelvy-a-vykresleni} společně s~vykreslením obrazce).
\begin{algorithm}[H]
    \KwIn{Vygenerované slovo $w$, pozice želvy $(x,y)$, úhel otočení $\varphi$, inkrementace úhlu $\delta$, délka kroku $d$}
    Založ prázdný zásobník $Z$\;
    $L\gets\emptyset$\;
    \For{$i=1,2,\ldots,|w|$}{
        \lIf{$w[i]=\mathtt{+}$}{$\varphi\gets\varphi+\delta$}
        \lIf{$w[i]=\mathtt{-}$}{$\varphi\gets\varphi-\delta$}
        \If{$w[i]=\mathtt{f}$}{
            $(x,y)\gets(x+d\cos\delta,y+d\sin\delta)$\;
        }
        \If{$w[i]=\mathtt{[}$}{
            Na vrchol zásobníku $Z$ ulož stav želvy $(x,y,\varphi)$\;
        }
        \If{$w[i]=\mathtt{]}$}{
            Z~vrcholu zásobníku $Z$ odstraň stav $(x_t,y_t,\varphi_t)$\;
            $(x,y)\gets(x_t,y_t)$\;
            $\varphi\gets\varphi_t$\;
        }
        \Else{
            $(x^\prime,y^\prime)\gets(x+d\cos\varphi,y+d\sin\varphi)$\;
            Do seznamu $L$ přidej úsečku $(x,y,x^\prime,y^\prime)$\;
            $(x,y)\gets(x^\prime,y^\prime)$\;
        }
    }
    \Return{$(x,y,\varphi,L)$}\;
    \KwOut{Stav želvy a~seznam vzniklých úseček $(x,y,\varphi,L)$}
    \caption{Simulace pohybu želvy}
    \label{alg:simulace-pohybu-zelvy}
\end{algorithm}
\begin{program}[h]
\begin{lstlisting}[style=python]
lsystem = LSystem(axiom, rules)
lsystem.iterate(iteration_count)

# Simulate turtle movement
stack = Stack()
for char in lsystem.word:
    if char == '+':
        turtle.rotate(angle)
    elif char == '-':
        turtle.rotate(-angle)
    elif char == 'f':
        turtle.pen_down = False
        turtle.forward()
    elif char == '[':
        stack.push((turtle.position, turtle.angle))
    elif char == ']':
        state = stack.pop()
        turtle.position = state[0]
        turtle.angle = state[1]
    else:
        turtle.pen_down = True
        turtle.forward()

# Center the figure        
turtle.center_to(window_width // 2, window_height // 2)

# Draw figure
for line in turtle.lines:
    canvas.create_line(line[0].x, line[0].y, line[1].x, line[1].y)
\end{lstlisting}
    \caption{Implementace algoritmu~\ref{alg:simulace-pohybu-zelvy} s~vykreslením}
    \label{prog:simulace-pohybu-zelvy-a-vykresleni}
\end{program}