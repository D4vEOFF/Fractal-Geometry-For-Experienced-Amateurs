\chapter{Generování fraktálů}\label{chapter:generovani-fraktalu}

V této poslední kapitole navážeme na znalosti z kapitoly předchozí. Proto zde opět čtenáři doporučuji se podívat na obsah kapitoly \ref{chapter:klasifikace-fraktalu} (vyjma sekcí týkajících se matematického základu). Podíváme se stručný teoretický rozbor algoritmů pro generování fraktálních útvarů, i na jejich praktickou implementaci.

V době psaní tohoto textu existuje mnoho programovacích jazyků\index{jazyk!programovací}\index{programovací jazyk} a nejspíše lze bezpečně předpokládat, že budou přibývat další. Nikoho tak nejspíše nepřekvapí, že vzhledem k současnému (dosti rychlému) vývoji v oblasti informatiky mnoho jazyků, které dříve byly považovány za nové a inovativní, postupně zastaraly a jiné pro mnoho jedinců dokonce upadly v zapomnění. Avšak jiné naopak si svoji pozici drží dodnes. Pro účely tohoto textu byl v rámci praktických ukázek, které uvidíte, zvolen jazyk \textbf{Python}\index{Python}, neboť jeho syntaxe není složitá\footnote{Složitost programovacího jazyka je, z pochopitelných důvodů, dosti subjektivní pojem, neboť závisí i na zkušenostech programátora.} a zároveň tak není příliš obtížné si v mnoha případech domyslet význam jednotlivých příkazů\footnote{Samozřejme nelze v tomto ohledu mluvit za každého (potenciálního) čtenáře. Pokud by tak kdykoliv vznikla nějaká nejasnost ohledně významu použitých příkazů, lze se podívat na stránky oficiální dokumentace jazyka Python: \url{https://docs.python.org}}. Zároveň však poznamenejme, že stejně jako v případě matematické části tohoto textu, i zde budeme pracovat s předpokladem, že čtenář je seznámen se základními koncepty programování a algoritmizace všeobecně. Nebudeme se zde tedy řešit, co je to proměnná\index{proměnná}, pole\index{pole} (resp. v Pythonu seznamem\index{seznam}), funkce, podmínky aj.

Zároveň bychom neměli zapomínat na zájemce používající jiné programovací jazyky. Proto kromě praktických ukázek si prezentované algoritmy uvedeme i pomocí tzv. \emph{pseudokódu}\index{pseudokód}. Pseudokód nepředstavuje sám o sobě žádnou formu programovacího jazyka. Jedná se čistě o abstraktní popis psaný především pro člověka, který lze však s minimálním úsilím přepsat do libovolného programovacího jazyka. Jednoduchým příkladem pseudokódu je např. \ref{alg:ukazka-pseudokodu}.
\begin{algorithm}[h]
    \KwIn{Seznam čísel $x_1,x_2,\ldots,x_n$.}
    $\text{max}\gets x_1$\\
    \For{$i=1,2,\ldots,n$}{
        \If{$x_i>\textup{max}$}{
            $\textup{max}\gets x_i$
        }
    }
    \Return{\textup{max}}
    \caption{Ukázkový pseudokód (hledání minima)}
    \label{alg:ukazka-pseudokodu}
\end{algorithm}
Ten lze snadno přepsat do jazyka Python např. následovně:
\begin{lstlisting}[style=python]
def findMax(numbers: list) -> int:
    maximum = numbers[0]
    for i in range(len(numbers)):
        number = numbers[i]
        if number > maximum:
            maximum = number
    return maximum
\end{lstlisting}
