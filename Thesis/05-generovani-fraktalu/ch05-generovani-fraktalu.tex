\chapter{Generování fraktálů}\label{chapter:generovani-fraktalu}

V této poslední kapitole navážeme na znalosti z kapitoly předchozí, tj. č. \ref{chapter:klasifikace-fraktalu}. Proto zde opět čtenáři doporučuji se podívat na její obsah (vyjma sekcí týkajících se matematického základu). Podíváme se stručný teoretický rozbor algoritmů pro generování fraktálních útvarů, i na jejich praktickou implementaci. Než tak však učiníme, uděláme si krátké povídání o způsobu, jakým budeme rozbory vůbec provádět.

Vzorový způsob implementace všech zde zmíněných algoritmů lze nalézt v přiloženém programu.

\section{Stručně k zápisu programů}\label{sec:zapis-programu}

V době psaní tohoto textu existuje mnoho programovacích jazyků\index{jazyk!programovací}\index{programovací jazyk} a nejspíše lze bezpečně předpokládat, že budou přibývat další. Nikoho tak nejspíše nepřekvapí, že vzhledem k současnému (dosti rychlému) vývoji v oblasti informatiky mnoho jazyků, které dříve byly považovány za nové a inovativní, postupně zastaraly a jiné pro mnoho jedinců dokonce upadly v zapomnění. Avšak jiné naopak si svoji pozici drží dodnes. Pro účely tohoto textu byl v rámci praktických ukázek, které uvidíte, zvolen jazyk \textbf{Python}\index{Python}, neboť jeho syntaxe není složitá\footnote{Složitost programovacího jazyka je, z pochopitelných důvodů, dosti subjektivní pojem, neboť závisí i na zkušenostech programátora.} a zároveň tak není příliš obtížné si v mnoha případech domyslet význam jednotlivých příkazů\footnote{Samozřejme nelze v tomto ohledu mluvit za každého (potenciálního) čtenáře. Pokud by tak kdykoliv vznikla nějaká nejasnost ohledně významu použitých příkazů, lze se podívat na stránky oficiální dokumentace jazyka Python: \url{https://docs.python.org}}. Zároveň však poznamenejme, že stejně jako v případě matematické části tohoto textu, i zde budeme pracovat s předpokladem, že čtenář je seznámen se základními koncepty programování a algoritmizace všeobecně. Nebudeme se zde tedy řešit, co je to proměnná\index{proměnná}, pole\index{pole} (resp. v Pythonu seznamem\index{seznam}), funkce, podmínky aj.

Zároveň bychom neměli zapomínat na zájemce používající jiné programovací jazyky. Proto kromě praktických ukázek si prezentované algoritmy uvedeme i pomocí tzv. \emph{pseudokódu}\index{pseudokód}. Pseudokód nepředstavuje sám o sobě žádnou formu programovacího jazyka. Jedná se čistě o abstraktní popis psaný především pro člověka, který lze však s minimálním úsilím přepsat do libovolného programovacího jazyka. Jednoduchým příkladem pseudokódu je např. \ref{alg:ukazka-pseudokodu}.
\begin{algorithm}
    \KwIn{Seznam čísel $x_1,x_2,\ldots,x_n$.}
    $\text{max}\gets x_1$\\
    \For{$i=1,2,\ldots,n$}{
        \If{$x_i>\textup{max}$}{
            $\textup{max}\gets x_i$
        }
    }
    \Return{\textup{max}}
    \caption{Ukázkový pseudokód (hledání minima)}
    \label{alg:ukazka-pseudokodu}
\end{algorithm}
Implementace takového algoritmu např. právě v jazyce Python si čtenář může prohlédnout u programu \ref{prog:ukazka-implementace-pseudokodu}.
\begin{program}[h]
    \begin{lstlisting}[style=python]
def findMax(numbers: list) -> int:
    maximum = numbers[0]
    for i in range(len(numbers)):
        number = numbers[i]
        if number > maximum:
            maximum = number
    return maximum
    \end{lstlisting}
    \caption{Možná implementace algoritmu \ref{alg:ukazka-pseudokodu}}
    \label{prog:ukazka-implementace-pseudokodu}
\end{program}
Pochopitelně se v konkrétní implementaci mohou vyskytovat různé odchylky. Např. v programu \ref{prog:ukazka-implementace-pseudokodu} využíme proměnnou \texttt{number}, kterou bychom jistě mohli vypustit a pracovat přímo se seznamem \texttt{numbers}, tzn. \texttt{numbers[i]}. Nebo bychom například nemuseli program vůbec zapisovat jako funkci, či bychom mohli např. jinak pojmenovat proměnné. To jsou však v celkovém kontextu pouhé drobnosti. V rámci textu se ovšem budeme snažit držet jednotné konvence, tedy že programy budeme vždy psát jako funkce/procedury a proměnné budeme pojmenovávat vždy v angličtině, neboť je to při programování zkrátka zvyklost.
\section{Implementace L-systémů a želví grafiky}\label{sec:implementace-lsystemu-a-zelvi-grafiky}

Na základní princip L-systémů\index{L-systém} jsme se podívali v části \ref{sec:L-systemy}. Jejich implementace je do jisté míry přímočará. Řešení je potřeba rozdělit na dvě části: implementace \emph{samotného L-systému} a \emph{želví grafiky}\index{želví grafika}.

\subsection{Implementace L-systémů}\label{subsec:implementace-lsystemu}

L-systém nepředstavuje nikterak složitou matematickou strukturu. Z definice (viz \ref{def:lsystem}) je potřeba znát pouze používané \emph{neterminály}\index{neterminál}, \emph{axiom}\index{axiom} a seznam přepisovacích pravidel. O to jednodušší je situace, započítáme-li fakt, že neterminální symboly (resp. jejich význam) v případě námi používaných L-systémů jsou pevně dané, tedy není třeba je explicitně uvádět v definici. L-systém lze tak implementovat jako jednoduchou třídu s atributy \texttt{word} obsahující aktuální slovo po $k$-té iteraci a slovník pravidel \texttt{rules} (viz program \ref{prog:konstruktor-lsystem}).
\begin{program}[h]
    \begin{lstlisting}[style=python]
class LSystem:
    def __init__(self, axiom: str, rules: dict) -> None:
        self._word = axiom
        self._rules = rules
\end{lstlisting}
    \caption{Konstruktor třídy pro L-systém}
    \label{prog:konstruktor-lsystem}
\end{program}
Slovní pravidel \texttt{rules} má jednoduchou strukturu. Klíče tvaří levé strany pravidel a k nim přiřazené hodnoty naopak tvoří pravé strany pravidel. Jeho vzhled může vapadat např. takto:
\begin{verbatim}
rules = {
    "X": "F-[[X]+X]+F[+FX]-X",
    "F": "FF"
}
\end{verbatim}
Poměrně zásadní pro nás však bude především metoda pro aplikaci jednotlivých pravidel. Pro další výklad si však zavedeme pohodlnější zápis řetězců, který je v programování zcela běžný.
\begin{definition}\label{def:index-retezce}
    Nechť $\alpha=x_1x_2\ldots x_n$ je slovo nad libovolnou abecedou $\Sigma\neq\emptyset$. Pak pro každé $1\leqslant i\leqslant n$ definujeme $\alpha[i]=x_i$.
\end{definition}
Myšlenka je velice intuitivní. Obecně máme-li řetězec $w$ po $m$-té iteraci a množinu přepisovacích pravidel $P\subseteq\set{a\to\alpha\mid a\in V,\;\alpha\in V^*}$, kde $V$ je abeceda, pak stačí pro každý znak $w[i]$, kde $1\leqslant i\leqslant n$, pouze zkontrolovat, zda není na levé straně nějakého pravidla v $P$. Pokud ano, dojde k aplikaci příslušného pravidla\footnote{Technicky vzato jsme z formálních důvodů v definici L-systému \ref{def:lsystem} přidali i pravidla tvaru $a\to a$, aby nedošlo k situaci, že pro $a$ neexistuje pravidlo. Avšak z hlediska prakticné implementace toto není překážkou, neboť v případě absence takového pravidla jednoduše symbol přeskočíme.}. Viz pseudokód \ref{alg:iterace-slova-lsystem}.
\begin{algorithm}[h]
    \KwIn{Množina pravidel $P$, slovo $w$, číslo $k\in\N$}
    $w^\prime\gets\lambda$\\
    \For{$m=1,2,\ldots,k$}{
        \For{$i=1,2,\ldots,|w|$}{
            \If{\textup{existuje pravidlo tvaru $(w[i]\to\alpha)\in P$}}{
                $w^\prime\gets w^\prime[1]\dots w^\prime[i-1]\alpha$
            }
            \Else{
                $w^\prime=w^\prime w[i]$
            }
        }
    }
    \Return{$w^\prime$}\\
    \KwOut{Slovo $w^\prime$ odvozené po $k$ iteracích ze slova $w$}
    \caption{Algoritmus pro $k$-tou iteraci slova $w$}
    \label{alg:iterace-slova-lsystem}
\end{algorithm}
Implementace je, vzhledem k dostupným funkcím v Pythonu, až překvapivě jednoduchá. O tom se čtenář může přesvědčit sám v případě kódu \ref{prog:iterace-slova-lsystem}.
\begin{program}[h]
\begin{lstlisting}[style=python]
def iterate(self, iteration_count: int) -> None:
    for _ in range(iteration_count):
        self._word = self._word.translate(str.maketrans(self._rules))
\end{lstlisting}
    \caption{Implementace algoritmu \ref{alg:iterace-slova-lsystem}.}
    \label{prog:iterace-slova-lsystem}
\end{program}
Pojďme si stručně rozebrat použíté funkce, resp. metody, v programu \ref{prog:iterace-slova-lsystem}.
\begin{itemize}
    \item \texttt{str.maketrans} vytvoří ze zadaného slovníku překladovou tabulku pro metodu \texttt{translate}. Její struktura odpovídá slovníku obsahující dvojice \emph{(Unicode\index{Unicode} hodnota, znak)}
    \item \texttt{translate} nahradí každý ze znaků řetězcem uvedeným v překladové tabulce.
\end{itemize}
Tímto způsobem lze implementovat třídu, kde vygenerujeme příslušný řetězec, který následně budeme interpretovat pomocí želví grafiky\index{želví grafika}.

\subsection{Implementace želví grafiky}\label{subsec:implementace-zelvi-grafiky}

Druhou částí je naprogramování želví grafiky. Nyní pracujeme se scénářem, že máme vygenerovaný příslušný řetězec znaků $w$, jehož znaky chceme interpretovat. Za účelem jednoduchosti se pokusíme striktně oddělit samotnou \emph{geometrickou interpretaci řetězce} od jeho \emph{grafické interpretace}.

Pro připomenutí významů jednotlivých symbolů doporučuji se znovu podívat do tabulek \ref{table:vyznam-symbolu-zelva} a \ref{table:vyznam-symbolu-zelva-zasobnik}. Nejdříve si však ujasněme, jaké informace si potřebujeme o želvě uchovávat.
\begin{itemize}
    \item Vzdálenost $d$, o kterou se želva při každém kroku posune,
    \item aktuální pozice želvy $(x,y)$,
    \item úhel $\alpha\in\langle 0,2\pi)$ udávající směr želvy
    \item přírůstek úhlu $\delta$,
    \item seznam nakrelených úseček reprezentované jako uspořádané čtveřice
    \[(x_0,y_0,x_1,y_1),\]
    kde $(x_0,y_0)$ a $(x_1,y_1)$ jsou souřadnice počátečního, resp. koncového bodu.
\end{itemize}
Podobně jako v případě L-systému, i zde můžeme želvu reprezentovat jako třídu (viz program \ref{prog:konstruktor-zelva}).
\begin{program}[h]
\begin{lstlisting}[style=python]
class Turtle:
    def __init__(self, step: float, position: Vector = Vector(0, 0), angle: float = 0) -> None:
        self._position = position
        self._step = step
        self._angle = (angle % 360) * math.pi / 180
        self._pen_down = False
        self._lines = []

        self._x_min, self._y_min = position.x, position.y
        self._x_max, self._y_max = position.x, position.y
\end{lstlisting}
    \caption{Konstruktor třídy pro želvu}
    \label{prog:konstruktor-zelva}
\end{program}

V tomto případě zvolíme při implementaci želvy následující strategii. Představíme si ji tak, že na sobě připevněné pero a budeme si pouze pamatovat, zda je či není v danou chvíli položeno na plátně. Pokud ano a želva provede krok vpřed, nakreslí za sebou úsečku.

Všechny atributy jsou vysvětleny níže.
\begin{itemize}
    \item \texttt{self.\_position} uchovává pozici želvy $(x,y)$,
    \item \texttt{self.\_step} reprezentuje velikost kroku $d$,
    \item \texttt{self.\_angle} je počáteční úhel otočení želvy přepočítaný v radiánech,
    \item \texttt{self.\_pen\_down} udává, zda je pero položeno na plátně.
    \item \texttt{self.\_lines} ukládá seznam dosud nakrelených úseček.
\end{itemize}
V konstruktoru třídy \texttt{Turtle} se navíc nachází soukromé atributy \texttt{self.\_x\_min}, \texttt{self.\_x\_max}, \texttt{self.\_y\_min} a \texttt{self.\_y\_max}. Ty nám budou sloužit pro pozdější vykreslování výsledného útvaru. Průběžně si v nich budeme uchovávat minimální, resp. maximální, souřadnici $x$ a $y$ ze všech dosud vygenerovaných úseček.
\section{Implementace IFS}\label{sec:implementace-ifs}

Systémy iterovaných funkcí a~k nim související teorii jsme si vyložili již v~části \ref{sec:ifs}. Podobně jako v~případě L-systémů,~i zde budeme postupovat přímo z~definice. Konkrétně jsme si definovali IFS jako množinu kontrakcí
\[\set{\mapping{\psi_i}{X}{X}\mid 1\leqslant i~\leqslant n},\]
přičemž jsme následně zkoumali a~pracovali se zobrazením $\Psi$ daným předpisem
\[\Psi(B)=\bigcup_{i=1}^n\psi_i(B)\;,\;B\in\hyperspace(X).\]
O zobrazení $\Psi$ jsme následně dokázali,~že se též jedná o~kontrakci (viz věta \ref{thm:sjednoceni-kontrakci}).

Konkrakce,~s nimiž jsme pracovali,~byla tzv. \emph{afinní zobrazení}\index{zobrazení!afinní}\index{afinní zobrazení} v~$\R^2$,~tedy jejich předpis byl ve tvaru
\begin{equation}\label{eq:afinni-zobrazeni}
    f(x,y)=\left(\begin{matrix}
        a~& b\\
        c & d
    \end{matrix}\right)\left(\begin{matrix}
        x\\
        y
    \end{matrix}\right)+\left(\begin{matrix}
        e\\
        f
    \end{matrix}\right)=\left(\begin{matrix}
        ax+by+e\\
        cx+dy+f
    \end{matrix}\right).
\end{equation}
V konečném důsledku si tedy stačilo uchovat pouze koeficienty $a,b,\ldots,f$. A~toho přesně využijeme i~zde. Daná afinní zobrazení budeme reprezentovat seznamem uspořádaných šestic
\[(a,b,c,d,e,f).\]
Dále potřebujeme již pouze znát počáteční obrazec. Ačkoliv bychom mohli jistě různými sofistikovanými způsoby reprezentovat celou řadu množin,~my se omezíme pouze na mnohoúhelníky\footnote{V konečném důsledku,~jak již víme z~Banachovy věty \ref{thm:banach},~počáteční obrazec nehraje žádnou roli pro atraktor zobrazení $\Psi$.},~neboť jejich reprezentace je velmi jednoduchá. Stačí si pamatovat pozice jeho vrcholů
\[(x_1,y_1),(x_2,y_2),\ldots,(x_n,y_n).\]
I zde provedeme implementaci IFS pomocí třídy (viz ukázka \ref{prog:konstruktor-ifs}).
\begin{program}[h]
\begin{lstlisting}[style=python]
from copy import deepcopy

class IFS:

    def __init__(self,~starting_figure: list,~tr_coefs: list = []) -> None:
        self._figures = [starting_figure]
        self._total_iterations = 0

        # Min/max coordinates (used for centering)
        self._x_min,~self._y_min,~self._x_max,~self._y_max = 0,~0,~0,~0
        self.__update_min_max_coords()

        self._transformations = set()
        for tpl in tr_coefs:
            def transformation(point,~tpl=deepcopy(tpl)):
                return Vector(
                    tpl[0]*point.x + tpl[1]*point.y + tpl[4],
                    tpl[2]*point.x + tpl[3]*point.y + tpl[5]
                )
            self._transformations.add(transformation)
    
    
    def __update_min_max_coords(self) -> None:
        self._x_min = min(point.x for figure in self._figures for point in figure)
        self._y_min = min(point.y for figure in self._figures for point in figure)
        self._x_max = max(point.x for figure in self._figures for point in figure)
        self._y_max = max(point.y for figure in self._figures for point in figure)
\end{lstlisting}
    \caption{Konstruktor pro třídu \texttt{IFS}}
    \label{prog:konstruktor-ifs}
\end{program}
Po vzoru třídy \texttt{Turtle},~kterou jsme si ukázali v~minulé sekci \ref{sec:implementace-lsystemu-a-zelvi-grafiky} i~zde si budeme průběžně aktualizovat minimální a~maximální hodnoty souřadnic pro pozdější manipulaci s~obrazcem. Dále zde máme dvojici důležitých atributů:
\begin{itemize}
    \item \texttt{self.\_figures} uchovává všechny vygenerované obrazce po obecně $k$-té iteraci jako seznam uspořádaných $n$-tic vrcholů.
    \item \texttt{self.\_transformations} ukládá zadané kontrakce $\psi_1,\psi_2,\ldots,\psi_m$ jako first-class funkce\index{first-class funkce}\index{funkce!first-class}. Výpočet probíhá podle \eqref{eq:afinni-zobrazeni}.
\end{itemize}
Co je to \emph{first-class funkce}\footnote{Též se lze někdy setkat s~českým označením funkce nebo obecněji objekty \emph{první kategorie}\index{objekty první kategorie}\index{funkce první kategorie}\index{funkce!první kategorie}.}\index{first-class funkce}\index{funkce!first-class}? Jedná se o~koncept práce s~funkcemi jakožto standardními objekty,~na které se lze odkazovat. V~praxi to znamená možnost předávat funkce jako parametry,~používat je jako návratové hodnoty,~nebo ukládat je do proměnných. To se nám v~tomto případě velmi hodí,~neboť tyto funkce potřebujeme vytvářet až za běhu programu v~závislosti na zadaných koeficientech afinních zobrazení.

V této části se zaměříme pouze na generování výsledného obrazce v~jednotlivých iteracích. V~tomto ohledu bude potřeba si uchovávat nově vygenerované útvary do nějaké struktury. To vše je shrnuto v~algoritmu \ref{alg:iterace-ifs}.
\begin{algorithm}[h]
    \KwIn{IFS $\set{\psi_1,\ldots,\psi_n}$,~množina útvarů $\mathcal{F}$,~číslo $k\in\N$}
    \For{$i=1,2,\ldots,k$}{
        $\mathcal{F}^\prime\gets\emptyset$\;
        \ForEach{$F\in\mathcal{F}$}{
            \ForEach{$\psi\in\set{\psi_1,\ldots,\psi_n}$}{
                $\mathcal{F}^\prime\gets\mathcal{F}^\prime\cup\set{\psi(F)}$\;
            }
        }
        $\mathcal{F}\gets \mathcal{F}^\prime$\;
    }
    \Return{$\mathcal{F}$}\;
    \KwOut{Nová množina útvarů $F$}
    \caption{$k$-tá iterace IFS}
    \label{alg:iterace-ifs}
\end{algorithm}
Speciálně,~pokud bychom chtěli vygenerovat $k$-tou z~počátečního obrazce $F_0$,~stačí algoritmus zavolat na množinu $\set{F_0}$. Praktickou implementaci si lze prohlédnout v~ukázce \ref{prog:iterace-ifs}.
\begin{program}[h]
\begin{lstlisting}[style=python]
def iterate(self,~iterations: int) -> None:
    for _ in range(iterations):
        figures_new = []
        
        for figure in self._figures:
            for tr in self._transformations:
                figure_new = []
                for point in figure: figure_new.append(tr(point))

                figures_new.append(figure_new)
    
        self._figures = figures_new
\end{lstlisting}
    \caption{Implementace algoritmu \ref{alg:iterace-ifs} ve třídě \texttt{IFS}}
    \label{prog:iterace-ifs}
\end{program}
Posunutí obrazce na střed si již může čtenář samostatně rozmyslet. Provedení by však bylo obdobné jako v~případě L-systémů.
\section{Implementace Time Escape algoritmů}\label{sec:implementace-tea}

Poslední kategorii fraktálních útvarů tvořily tzv. \emph{Juliovy množiny}, u nichž jsme si jednoduše vysvětlili, že jejich generování probíhá pomocí tzv. \emph{Time Escape algoritmů}\index{Time Escape algoritmus}\index{algoritmus!Time Escape}. Jejich princip lze nastínit následovně: na vstupu zadáme nějakou komplexní polynomiální funkci $f$ a dále čísla $m\in\N_0$ a $r\in\R$. Čislo $m$ bude sloužit jako horní hranice počtu iterací, který pro každý bod v určitě omezené oblasti komplexní roviny provedeme (to lze pochopitelně dospecifikovat, avšak teď to není úplně podstatné). Pokud v kterékoliv iteraci nastane, že $|f^{\circ k}(z)|>r$, pak bod vyloučíme ze zkoumané množiny. Naopak v případě, že pro každé $0\leqslant k\leqslant m$ je $|f^{\circ k}(z)|\leqslant r$, pak prohlásíme, že bod náleží Juliově množině.

Podívejme se na tento algoritmus trochu blíže v pseudokódu \ref{alg:generovani-vyplnene-jf}.
\begin{algorithm}
    \KwIn{Komplexní polynomiální funkce $f$, maximální počet iterací $m\in\N_0$, číslo $r\in\R$, konečné množiny $X\subset\langle x_{\text{min}},x_{\text{max}}\rangle$ a $Y\subset\langle y_{\text{min}},y_{\text{max}}\rangle$}
    $K\gets\emptyset$\;
    \ForEach{\textup{$(a,b)\in X\times Y$}}{
        $z\gets a+b\imag$\;
        $t\gets\id$\;
        \For{$k=0,1,\ldots,m$}{
            $t\gets t\circ f$\;
            \If{$|t(z)|>r$}{
                pokračuj další iterací vnějšího cyklu\;
            }
        }
        $K\gets K\cup\set{z}$\;
    }
    \Return{$K$}\;
    \KwOut{Aproximace vyplněné Juliovy množiny $K(f)$}
    \caption{Generování vyplněné Juliovy množiny při pevném počtu iterací}
    \label{alg:generovani-vyplnene-jf}
\end{algorithm}
Nejspíše nikoho nepřekvapí, že při vyšších hodnotách čísla $m$ obdržíme lepší odhad Juliovy množiny příslušné polynomiální funkci $f$. Avšak vždy je potřeba zvážit náročnost výpočtu.

Znázornění (vyplněné) Juliovy množiny lze provést různými způsoby. Prezentovaný algorimus \ref{alg:generovani-vyplnene-jf} pouze určuje pro každý zvolený bod $z$, zda naleží, či nenáleží množině $J$. Avšak čtenář zajímající se o tuto partii matematiky již nejspíše viděl poměrně známý způosb vyobrazení těchto množin s barevným rozlišováním bodů. Tuto záležitost jsme již zmínili ke konci části \ref{subsec:juliovy-fatouovy-mnoziny}, avšak jeho podstatné aspekty jsou především algoritmické povahy a tedy teprve v této kapitole je více rozvedeme. K tomu se však dostaneme později.

Praktická implementace Time Escape algoritmů bude podstatně složitější, neboť si musíme vypořádat s konverzí samotného polynomu a rovněž vyřešit způsob vzorkování vybrané části komplexní roviny. Dále se budeme držet realizace pomocí třídy (viz ukázka \ref{prog:konstruktor-tea}).
\begin{program}[h]
\begin{lstlisting}[style=python]
class TEA:
    def __init__(self, width: int, height: int, sequence: str, step: int = 1, escape_radius: int = 2, bounds: tuple, var: str, explore_var: str):
        self._x_count, self._y_count = width // step, height // step
        
        self._iter_counts = [[0 for _ in range(self._x_count)] for _ in range(self._y_count)]
        self._width, self._height = width, height
        self._sequence = sequence
        self._var = var
        self._explore_var = explore_var
        self._total_iterations = 0
        self._escape_radius = escape_radius

        x_min, x_max, y_min, y_max = bounds

        x_vals = [x_min + step * (x_max - x_min) * j / width for j in range(self._x_count + 1)]
        y_vals = [y_min + step * (y_max - y_min) * i / height for i in range(self._y_count + 1)]
        self._complex_grid = [[x + 1j * y for x in x_vals] for y in y_vals]

        self._point_last_values = [[0 for _ in range(self._x_count)] for _ in range(self._y_count)]
\end{lstlisting}
    \caption{Konstruktor třídy \texttt{TEA}}
    \label{prog:konstruktor-tea}
\end{program}
Pojďme si konstruktor \ref{prog:konstruktor-tea} opět rozebrat.
\begin{itemize}
    \item \texttt{self.\_x\_count} a \texttt{self.\_y\_count} udávají počet bodů, které budeme prozkoumávat, ve směru reálné a imaginární osy. Jejich hodnoty jsou závislé na velikosti kroku \texttt{step}, kterou konstruktor přijímá jako parametr. Tedy např. s krokem $1$ v rámci intervalu $\langle-1,2\rangle$ budeme zkoumat celkem 4 body.
    \item V seznamu \texttt{self.\_iter\_counts} si budeme pro každý bod uchovávat, kolik iterací zadané funkce bylo potřeba, než posloupnost začala divergovat (tedy $|f^{\circ k}(z)|>r$). Ten využijeme především později při určování barev jednotlivých bodů.
    \item \texttt{self.\_sequence} uchovává předpis polynomální funkce, kterou budeme iterovat, jako řetězec. Předpisy budeme zadávat standardní syntaxí v Pythonu, tzn. např. pro Mandelbrotovu množinu, kde $f_c(z)=z^2+c$, bychom předpis napsali
    \begin{center}
        \texttt{z**2 + c}.
    \end{center}
    Levou stranu "\texttt{f(z) =}" pochopitelně psát netřeba.
    \item \texttt{self.\_var} udává, která proměnná v \texttt{self.\_sequence} je argumentem zadané funkce (typicky \texttt{z}).
    \item \texttt{self.\_explore\_var} uchovává proměnnou, za níž budeme dosazovat hodnoty zkoumaných bodů. Pro Juliovy množiny se typicky jedná přímo o argument zadané funkce $f$, ale např. pro Mandelbrotovu množinu je to \texttt{c}.
    \item \texttt{self.\_escape\_radius} reprezentuje číslo $r$, tedy hranici absolutní hodnoty čísla $z$, po jejímž překročení prohlásíme posloupnost iterací za divergentní.
    \item Parametr \texttt{bounds} specifikuje část komplexní roviny, z niž budeme zkoumat vybrané body. Jedná se o datový typ \texttt{tuple}, v našem případě uspořádanou čtveřici $(x_{\text{min}},x_{\text{max}},y_{\text{min}},y_{\text{max}})$.
    \item \texttt{self.\_complex\_grid} uchovává všechny body ze zadané části komplexní roviny jako komplexní čísla, tedy v Pythonu datový typ \texttt{complex}.
    \item Do seznamu \texttt{self.\_point\_last\_values} budeme ukládat hodnoty $f^{\circ k}(z)$ pro zadané $z$, kde $k$ je číslo poslední prozkoumané iterace (tzn. buď jsme u bodu prozkoumali maximální počet zadaných iterací, nebo výpočet skončil dříve kvůli překročení povolené absolutní hodnoty).
\end{itemize}
Je vidět, že atributů zde máme poměrně hodně. Zkusme si tedy nejdříve vyjasnit, jak bychom mohli pomocí těchto informací implementovat algoritmus pro iterování zadané polynomiální funkce. Již jsme si uvedli asi nejjednodušší možnost v rámci algoritmu \ref{alg:generovani-vyplnene-jf}. Jak by se ale situace změnila, když si budeme uchovávat počty iterací, které proběhly než jsme překročili zadanou absolutní hodnotu $r$ nebo dosáhli maximálního počtu $m$? Označme si takové pole např. $T$. Podívejme se na algoritmus \ref{alg:generovani-vyplnene-jf-pole}.
\begin{algorithm}[h]
    \KwIn{Komplexní polynomiální funkce $f$, maximální počet iterací $m\in\N_0$, číslo $r\in\R$, konečné množiny $X\subset\langle x_{\text{min}},x_{\text{max}}\rangle$ a $Y\subset\langle y_{\text{min}},y_{\text{max}}\rangle$}
    $K\gets\emptyset$\;
    \ForEach{$(x,y)\in X\times Y$}{
        $z\gets x+y\imag$\;
        $t\gets\id$\;
        \For{$k=0,1,\ldots,m$}{
            $t\gets t\circ f$\;
            $T[x,y]\gets k$\;
            \lIf{$|t(z)|>r$}{opusť cyklus}
        }
        \lIf{$T[x,y]=m$}{$K\gets K\cup\set{z}$}
    }
    \Return{$K$}\;
    \KwOut{Aproximace vyplněné Juliovy množiny $K$}
    \caption{Generování vyplněné Juliovy množiny pomocí pole iterací $T$}
    \label{alg:generovani-vyplnene-jf-pole}
\end{algorithm}
Pro implementaci algoritmu \ref{alg:generovani-vyplnene-jf-pole} si však budeme muset rozmyslet, jak budeme pomocí řetězce s předpisem pro funkci $f$ (tj. atributu \texttt{self.\_sequence}) počítat její funkční hodnoty. Jistě se nabízí možnost vytvořit funkci pro vyhodnocení obecného matematického výrazu. Ač se jedná o poměrně hezké programovací cvičení a čtenář si jej může vyzkoušet, my si poradíme trochu jinak -- funkcí \texttt{eval}. Funkce \texttt{eval} jednoduše vyhodnotí zadaný výraz a za proměnné dosadí hodnoty podle slovníku poskytnutého v příslušném parametru/parametrech.

Tím se nám situace podstatně zlehčuje. Implementaci si může čtenář prohlédnout v ukázce \ref{prog:generovani-vyplnene-jf-pole}.
\begin{program}[h]
\begin{lstlisting}[style=python]
def iterate(self, iterations: int) -> None:
    for i in range(self._y_count):
        for j in range(self._x_count):
            # Initialize variables
            vars_dict = {self._var: 0, self._explore_var: self._complex_grid[i][j]}

            # Iterate
            for k in range(1, iterations + 1):
                try:
                    # Evaluate the next value in the sequence
                    vars_dict[self._var] = eval(self._sequence, {"math": math}, vars_dict)

                    # Check for escape condition
                    self._iter_counts[i][j] = k
                    if abs(vars_dict[self._var]) > self._escape_radius:
                        break
                except OverflowError:
                    self._iter_counts[i][j] = k
                    break
            
            self.point_last_values[i][j] = vars_dict[self._var]
\end{lstlisting}
    \caption{Implementace algoritmu \ref{alg:generovani-vyplnene-jf-pole}}
    \label{prog:generovani-vyplnene-jf-pole}
\end{program}