\chapter{Teorie míry a dimenze}\label{chapter:teorie-miry-a-dimenze}

V této kapitole se budeme nyní věnovat fraktálům a jim příbuzným záležitostem trochu formálněji. Do této chvíle jsme si již stihli představit některé základní fraktály,~jako je např. \emph{Sierpińského trojúhelník},~\emph{Kochova vločka} nebo \emph{Cantorovo diskontinuum},~na nichž jsme si ilustrovali především myšlenku soběpodobnosti a na to navazující pojetí dimenze (viz kapitola \ref{chapter:uvod_do_fraktalu},~sekce \ref{sec:sobepodobnost} a \ref{sec:fraktalni_dimenze}).

Ačkoliv leckterý čtenář by se s~poskytnutým vysvětlením jistě spokojil,~jiný by mohl namítat,~že jsme řadu věcí vynechali. A měl by jistě pravdu. Proto se v~této kapitole budeme věnovat některým záležitostem z~tzv. \emph{teorie míry}\index{teorie míry},~která je v~tomto ohledu klíčová a poskytne nám nástroje pro měření fraktálních útvarů,~jejichž geometrie často přesahuje možnosti klasické "eukleidovské analýzy". \emph{Míra} pro nás představuje zobecnění pojmů jako je \emph{délka,~obsah} a \emph{objem},~které známe ze školní geometrie. Na jejím základě pak budeme schopni detailněji prozkoumat fraktální dimenzi,~kterou jsme již v~základu pokryli v~předešlé kapitole. Jmenovitě se budeme zabývat
\begin{itemize}
    \item \emph{měřitelnými prostory}\index{měřitelný prostor} a \emph{prostory s~mírou}\index{prostor s~mírou} obecně,
    \item \emph{lebesgueovou mírou}\index{Lebesgueova míra},
    \item \emph{box-counting dimenzí}\index{box-counting dimenze}\footnote{Též ji lze nalézt pod~názvem \emph{Minkowského dimenze}\index{Minkowského dimenze} nebo \emph{Minkowského-Bouligandova dimenze}\index{Minkowského-Bouligandova dimenze}. Je pojmenována po polském matematikovi \name{Hermannovi Minkowském} (1864--1909) a francouzském matematikovi \name{Georgesovi Bouligandovi} (1889--1979).}
    \item \emph{Hausdorffovou mírou} a z~ní vycházející \emph{Hausdorffovou dimenzí}.
\end{itemize}

Ačkoliv je toto téma jinak velice obsáhlé jsou mu věnované samostatné texty i~knihy,~spokojíme se pouze s~naprostým základem. Pro další znalosti si dovolím čtenáře odkázat na knihy \cite{Falconer2014},~\cite{Lukes2013},~\cite{NetukaIntegral2016} a~\cite{Edgar2008}.

Než se však pustíme do samotné problematiky,~je důležité zmínit,~že pro rigorózní budování budeme potřebovat některé základní znalosti. V~dalším textu předpokladáme,~že je s~nimi čtenář obeznámen. I~přesto si zde dovolíme začít soupisem pojmů a značení,~které budeme dále potřebovat. Některé záležitosti využijeme ještě v~kapitole \ref{chapter:hausdorffuv-mp} o~Hausdorffově metrickém prostoru.

\section{Základní pojmy a značení}\label{sec:zakladni-pojmy-a-znaceni}

V tomto oddílu se v krátkosti zaměříme na připomenutí některých pojmů a značení, které budeme dále využívat.

\todo{Doplnit pojmy a značení podle dalšího textu}
\section{Prostory s~mírou}\label{sec:prostory-s-mirou}

Jak již bylo zmíněno v~úvodu, klíčovým pojmem v~této kapitole (a pro studium fraktálů obecně) je takzvaná \emph{míra}. Ta pro nás představuje obecný způsob, jak můžeme množinám přiřadit v~jistém smyslu "velikost". Konkrétněji, byť vágně, lze říci, že sestává-li množina z konečného nebo spočetného množství "rozumných" částí, pak součet velikostí všech těchto dílčích množin je roven velikosti celé množiny, kterou nazveme její \emph{mírou}. Pro začátek celkem jednoduchá myšlenka.

Pro formální zavedení tohoto pojmu však budeme muset nejprve zavést ještě jiný pojem, a to tzv. \emph{$\sigma$-algebru}.

\subsection{$\sigma$-algebra}\label{subsec:sigma-algebra}

\begin{definition}[$\sigma$-algebra]\label{def:sigma-algebra}
    Nechť $X$ je libovolná množina a systém podmnožin $\mathcal{A}\subseteq\powset{X}$. Pak $\mathcal{A}$ je \emph{$\sigma$-algebra} na množině $X$, pokud:
    \begin{enumerate}[label=(\roman*)]
        \item\label{def:sigma-algebra-podm1} $X\in\mathcal{A}$,
        \item\label{def:sigma-algebra-podm2} $\forall A\in X: A\in\mathcal{A}\implies X\setminus A\in\mathcal{A}$,
        \item\label{def:sigma-algebra-podm3} pro libovolné množiny $A_1,A_2,\ldots\in\mathcal{A}$ platí $\bigcup_{i=1}^\infty A_i\in\mathcal{A}$.
    \end{enumerate}
\end{definition}

\begin{example}
    Jednoduché příklady $\sigma$-algeber:
    \begin{itemize}
        \item Triviálními příklady $\sigma$-algeber jsou množiny $\emptyset$, $\powset{X}$ a~$\set{\emptyset,X}$ pro libovolnou množinu $X$.
        \item Pro konečnou množinu $X=\set{a,b,c,d}$ je jednou možnou $\sigma$-algebrou systém množin
        \[\Sigma=\set{\emptyset,\set{a,b},\set{c,d},\set{a,b,c,d}}.\]
    \end{itemize}
    Sami se zkuste přesvědčit, že všechny zmíněné příklady vyhovují definici \ref{def:sigma-algebra}.
\end{example}

Než vyslovíme něco dalšího o~$\sigma$-algebrách a jejich významu, podíváme se seznam některých vesměs jednoduchých pozorováních zformulovaných níže v~tvrzení \ref{thm:sigma-algebra-vlastnosti}.
\begin{theorem}[Vlastnosti $\sigma$-algebry]\label{thm:sigma-algebra-vlastnosti}
    Nechť $\mathcal{A}$ je $\sigma$-algebra na množině $X$. Pak platí:
    \begin{enumerate}[label=(\roman*)]
        \item $\emptyset\in\mathcal{A}$,
        \item pro libovolné množiny $A_1,A_2,\ldots\in\mathcal{A}$ platí $\bigcap_{i=1}^\infty A_i\in\mathcal{A}$,
        \item pro všechny množiny $A_1,A_2,\ldots,A_n\in\mathcal{A}$ platí
        \[\bigcup_{i=1}^n A_i\in\mathcal{A}\land\bigcap_{i=1}^n A_i\in\mathcal{A}\],
        \item $\forall A,B\in\mathcal{A}\implies A\setminus B\in\mathcal{A}$.
    \end{enumerate}
\end{theorem}

Z tohoto tvrzení je již lépe vidět, proč jsou pro nás $\sigma$-algebry tak příjemným objektem. Jsou totiž \emph{uzavřené} na všechny základní množinové operace. To se nám bude později hodit při zavedení míry, ke které směřujeme. Důkaz těchto dílčích tvrzení přitom není nikterak složitý.
\begin{proof}
    Mějme $\sigma$-algebru $\mathcal{A}$ na množině $X$.
    \begin{enumerate}[label=\textit{(\roman*)}]
        \item Z podmínky \ref{def:sigma-algebra-podm1} definice \ref{def:sigma-algebra} víme, že $X\in\mathcal{A}$ a z podmínky \ref{def:sigma-algebra-podm2} tedy plyne $X\setminus X=\emptyset\in\mathcal{A}$.
        \item Mějme množiny $A_1,A_2,\ldots\in\mathcal{A}$. Společně s~využitím De Morganových zákonů plyne následující:
        \[\bigcap\limits_{i=1}^\infty A_i=\overbrace{X\setminus\underbrace{\bigcup\limits_{i=1}^\infty \overbrace{(X\setminus A_i)}^{\text{$\in\mathcal{A}$ podle \ref{def:sigma-algebra-podm2}}}}_{\text{$\in\mathcal{A}$ podle \ref{def:sigma-algebra-podm3}}}}^\text{$\in\mathcal{A}$ podle \ref{def:sigma-algebra-podm2}}\in\mathcal{A}.\]
        \item Nechť jsou dány množiny $A_1,A_2,\ldots,A_n\in\mathcal{A}$. Když pro každé $j>n$ položíme $A_j=\emptyset$, pak platí
        \[\bigcup\limits_{i=1}^n A_i=\bigcup\limits_{i=1}^\infty A_i\in\mathcal{A}\]
        a podobně pro $\bigcap_{i=1}^n A_i\in\mathcal{A}$ podle předešlého bodu.
        \item Pro libovolné množiny $A,B\in\mathcal{A}$ platí
        \[A\setminus B=\overbrace{A\cup\underbrace{(X\setminus B)}_{\text{$\in\mathcal{A}$ podle \ref{def:sigma-algebra-podm2}}}}^{\text{$\in\mathcal{A}$ podle \ref{def:sigma-algebra-podm3}}}\in\mathcal{A}.\]
    \end{enumerate}
\end{proof}

\subsection{Míra}\label{subsec:mira}

V tuto chvíli máme již vše potřebné k~zavedení pojmu míra, resp. prostor s~mírou.
\begin{definition}[Prostor s~mírou]\label{def:prostor-s-mirou}
    Nechť $\mathcal{A}$ je $\sigma$-algebra na množině $X$. Zobrazení $\mapping{\mu}{\mathcal{A}}{\langle0,\infty\rangle}$ se nazývá \emph{míra} na $\mathcal{A}$, pokud platí:
    \begin{enumerate}[label=(\roman*)]
        \item $\mu(\emptyset)=0$,
        \item pro množiny $A_1,A_2,\ldots\in\mathcal{A}$ po dvou disjunktní je
        \[\mu\left(\bigcup\limits_{i=1}^\infty A_i\right)=\sum_{i=1}^\infty\mu(A_i).\mathrightnote{$\sigma$-aditivita}\]
    \end{enumerate}
    Uspořádanou trojici $(X,\mathcal{A},\mu)$ nazýváme \emph{prostor s~mírou}.
\end{definition}

Vzhledem k tomu, co míra reprezentuje (tj. zobecnění délky, obsahu, objemu), jsou tyto požadavky intuitivně dosti smysluplné.

\begin{example}
    Příklady prostorů s mírou:
    \begin{itemize}
        \item Asi pro nás nejtypičtější způsob, jak měřit "velikost" množiny, je podle \emph{počtu prvků}. Pro libovolnou množinu $X$ a potenční množinu $\powset{X}$ lze definovat prostor s mírou $(X,\powset{X},\mu)$, kde pro libovolnou množinu $A\in\powset{X}$ položíme $\mu(A)=|A|$. Takto definované míře $\mu$ říkáme \emph{aritmetická míra}.
        \item Zobrazení přiřazující náhodnému jevu pravděpodobnost je též případem míry. Označíme-li si $\Omega=\set{\omega_1,\omega_2,\ldots,\omega_n}$ množinu všech elementárních jevů a $\mathcal{F}\subseteq\powset{\Omega}$, pak $\mapping{\mathsf{P}}{\mathcal{F}}{\langle0,1\rangle}$ definovaná pro $A\in\mathcal{F}$ jako
        \[\mathsf{P}(A)=\dfrac{|A|}{|\Omega|}\]
        je mírou na $\mathcal{F}$. Speciálně $\mathsf{P}(\Omega)=1$.
    \end{itemize}
\end{example}
\section{Lebesgueova míra}\label{sec:lebesgueova-mira}

\todo{Doplnit zmínku o~Jordanově-Peanově obsahu}

V předešlé sekci \ref{sec:prostory-s-mirou} jsme si povídali o~pojmu \emph{míra} obecně a~podívali jsme se na několik příkladů. Obecnou ideu měření "velikosti" lze založit např. na aproximaci obecné množiny pomocí \emph{spočetných sjednocení útvarů},~jejichž "velikost" umíme jednoduše určit. V~dalším textu se omezíme pouze na množinu $\R^n$.

Na zmíněné myšlence je postavena definice tzv. \emph{$n$-rozměrné Lebesgueovy míry},~kdy obecnou množinu budeme pokrývat pomocí \emph{kvádrů}. Připomeňme,~že obecně\linebreak\mbox{$n$-rozměrným} kvádrem\index{$n$-rozměrný kvádr} $I$ rozumíme kartézský součin \emph{intervalů}
\[\langle a_1,b_1\rangle,\ldots,\langle a_n,b_n\rangle\subseteq\R,\]
tj.
\[I=\prod_{i=1}^{n}\langle a_i,b_i\rangle=\langle a_1,b_1\rangle\times\langle a_2,b_2\rangle\times\dots\times\langle a_n,b_n\rangle\]
a jeho objem\index{kvádr!objem kvádru} definujeme jako
\[\vol_n(I)=\prod_{i=1}^{n}(b_i-a_i).\]
Lze nejspíše ihned vidět,~že objem kvádru $\vol_n(I)$ je \emph{aditivní} i~\textit{subaditivní}.

\todo{Přidat definici kvádru a~jeho objemu do sekce do značením}

Nyní si definujeme tzv. \emph{vnější Lebesgueovu míru}.
\begin{definition}[Vnější Lebesgueova míra]\label{def:vnejsi-lebegueova-mira}
    Nechť $A\subseteq\R^n$. Pak vnější $n$-rozměr\-nou Lebesgueovou mírou\index{vnější $n$-rozměrná Lebesgueova míra} $A$ je
    \[\lambda_n^*(A)=\inf\set{\sum_{j=1}^{\infty}\vol_n(I_j)\;\middle|\;\text{$I_j$ je kvádr}\;,\;A\subseteq\bigcup_{i=1}^\infty I_j}.\]
\end{definition}
Vnější Lebesgueova míra množiny intuitivně zachycuje informaci o~"velikosti" dané množiny.  Lze ihned vidět,~že pro libovolnou množinu $A\subseteq\R^n$ je $\lambda_n^*(A)\in\R_0^+$,~protože $\vol_n(I_j)\geqslant0$ pro každé $j\in\N$.
\begin{example}\label{ex:lebegueova-mira-trivialni-priklady}
    Ukažme si některé triviální příklady výpočtů vnější Lebesgueovy míry z definice (viz \ref{def:vnejsi-lebegueova-mira}),~tedy budeme hledat příslušné pokrytí dané množiny.
    \begin{itemize}
        \item Pro prázdnou množinu $\emptyset$ je $\lambda_n^*(\emptyset)=0$,~neboť $\emptyset\subseteq\prod_{i=1}^{n}\langle0,0\rangle$ (prázdnou množinu lze pokrýt jakýmkoliv kvádrem nulového objemu) a
        \[\vol_n\left(\prod_{i=1}^{n}\langle0,0\rangle\right)=0.\]
        \item Mějme libovolnou konečnou množinu $A=\set{x_1,x_2,\dots,x_n}\subseteq\R^n$. Pro každé $x_j$ stačí položit $I_j=\set{x_j}$ pro každé $1\leqslant j\leqslant n$,~což je degenerovaný interval,~jehož objem $\vol_n(I_j)=0$.
        \item Pro libovolnou spočetnou množinu $A=\set{x_i\mid i\in\N}\subseteq\R^n$ je $\lambda_n^*(A)=0$. Pokrytí volíme stejně jako v~předešlém bodě. Tedy např. pro $\Q\subset\R$ je $\lambda_1^*(\Q)=0$,~neboť $\Q$ je spočetná.
        \item Pro množinu reálných čísel $\R$ je $\lambda_1^*(\R)=\infty$,~avšak pro 
        \[A=\set{(x,0)\mid x\in\R}\subset\R^2\]
        (reálná osa v~$\R^2$) je $\lambda_2^*(A)=0$.
    \end{itemize}
\end{example}
Jako poslední si ukážeme,~že v~případě kvádru $n$-rozměrného kvádru odpovídá jeho vnější Lebesgueova míra skutečně jeho objemu.
\begin{proposition}\label{prop:lebegueova-mira-objem-kvadru}
    Je-li $I\subset\R^n$ kvádr,~pak $\lambda_n^*(I)=\vol_n(I)$.
\end{proposition}
\begin{proof}
    Ukážeme zvlášť,~že $\lambda_n^*(I)\leqslant\vol_n(I)$ a~$\lambda_n^*(I)\geqslant\vol_n(I)$.
    \begin{itemize}
        \item \textbf{Důkaz $\lambda_n^*(I)\leqslant\vol_n(I)$.} Zvolme pokrytí $\mathcal{I}=\set{I_1,I_2,\ldots}$ kvádru $I$,~tzn. $I\subseteq\bigcup_{i=1}^\infty I_i$,~tak,~aby
        \[\sum_{i=1}^{\infty}\vol_n(I_i)\leqslant(1+\varepsilon)\lambda_n^*(I).\]
        pro nějaké $\varepsilon>0$. Nyní si zvolíme nové kvádry\footnote{Formálně $\mathcal{I}$ tvoří zjemnění pokrytí $\mathcal{J}$} $\mathcal{J}=\set{J_1,J_2,\ldots}$ tak,~aby pro každé $i\in\N$ platilo
        \[I\subset\interior{J}\land\vol_n(J_i)\leqslant(1+\varepsilon)\vol_n(I_i).\]
        To není nikterak složité,~stačí např. pro každé $i\in\N$ položit
        \[J_i=\prod_{j=1}^{n}\left\langle a_j\sqrt[n]{1+\varepsilon},b_j\sqrt[n]{1+\varepsilon}\right\rangle.\]
        Protože však $I$ je uzavřená a~omezená množina,~je podle věty \todo{doplnit odkaz} kompaktní,~tedy z otevřeného pokrytí $\interior{J_1},\interior{J_2},\ldots$ lze vybrat konečné podpokrytí. Existuje tedy $m\in\N$,~takové,~že
        \[I\subseteq\bigcup_{i=1}^m J_i.\]
        Celkově tedy dostáváme
        \begin{align*}
            \vol_n(I)&\leqslant\sum_{i=1}^{m}\vol_n(J_i)\leqslant(1+\varepsilon)\sum_{i=1}^{m}\vol_n(I_i)\leqslant(1+\varepsilon)\sum_{i=1}^{\infty}\vol_n(I_i)\\
            &\leqslant (1+\varepsilon)^2\lambda_n^*(I)
        \end{align*}
        pro každé $\varepsilon>0$,~což jsme chtěli.
        \item \textbf{Důkaz $\lambda_n^*(I)\geqslant\vol_n(I)$.} Důkaz opačné nerovnosti je oproti té předešlé naopak velmi jednoduchý. Kvádr $I$ totiž reprezentuje pokrytí sebe samotného,~tzn. lze volit $I_1=I$ a~zbylé kvádry $I_j$,~kde $j\geqslant 2$,~mohou být libovolné s~nulovým objemem.
    \end{itemize}
    Z kombinací obou nerovností lze vidět,~že vnější Lebesgueova míra $n$-rozměrného kvádru skutečně odpovídá jeho objemu,~tj.
    \[\lambda_n^*(I)=\vol_n(I).\]
\end{proof}
% \begin{example}[Výpočet vnější Lebesgueovy míry kvádru]\label{ex:lebegueova-mira-objem-kvadru}
%     Jako poslední si ukážeme,~že v~případě kvádru $n$-rozměrného kvádru odpovídá jeho vnější Lebesgueova míra skutečně jeho objemu,~tzn. pro kvádr $I=\prod_{i=1}^{n}\langle a_i,b_i\rangle$ je
%     \[\lambda_n^*(I)=\vol_n(I)=\prod_{i=1}^{n}(b_i-a_i).\]
%     \begin{itemize}
%         \item \textbf{Důkaz $\lambda_n^*(I)\leqslant \prod_{i=1}^{n}(b_i-a_i)$.} Zvolme pokrytí $\mathcal{I}=\set{I_1,I_2,\ldots}$ kvádru $I$,~tzn. $I\subseteq\bigcup_{i=1}^\infty I_i$,~tak,~aby
%         \[\sum_{i=1}^{\infty}\vol_n(I_i)\leqslant(1+\varepsilon)\lambda_n^*(I).\]
%         pro nějaké $\varepsilon>0$. Nyní si zvolíme nové kvádry\footnote{Formálně $\mathcal{I}$ tvoří zjemnění pokrytí $\mathcal{J}$} $\mathcal{J}=\set{J_1,J_2,\ldots}$ tak,~aby pro každé $i\in\N$ platilo
%         \[I\subset\interior{J}\land\vol_n(J_i)\leqslant(1+\varepsilon)\vol_n(I_i).\]
%         To není nikterak složité,~stačí např. pro každé $i\in\N$ položit
%         \[J_i=\prod_{j=1}^{n}\left\langle a_j\sqrt[n]{1+\varepsilon},b_j\sqrt[n]{1+\varepsilon}\right\rangle.\]
%         Protože však $I$ je uzavřená a~omezená množina,~je podle věty \todo{doplnit odkaz} kompaktní,~tedy z otevřeného pokrytí $\interior{J_1},\interior{J_2},\ldots$ lze vybrat konečné podpokrytí. Existuje tedy $m\in\N$,~takové,~že
%         \[I\subseteq\bigcup_{i=1}^m J_i.\]
%         Celkově tedy dostáváme
%         \begin{align*}
%             \vol_n(I)&\leqslant\sum_{i=1}^{m}\vol_n(J_i)\leqslant(1+\varepsilon)\sum_{i=1}^{m}\vol_n(I_i)\leqslant(1+\varepsilon)\sum_{i=1}^{\infty}\vol_n(I_i)\\
%             &\leqslant (1+\varepsilon)^2\lambda_n^*(I)
%         \end{align*}
%         pro každé $\varepsilon>0$,~což jsme chtěli.
%         \item \textbf{Důkaz $\lambda_n^*(I)\geqslant \prod_{i=1}^{n}(b_i-a_i)$.} Důkaz opačné nerovnosti je oproti té předešlé naopak velmi jednoduchý. Kvádr $I$ totiž reprezentuje pokrytí sebe samotného,~tzn. lze volit $I_1=I$ a~zbylé kvádry $I_j$,~kde $j\geqslant 2$,~mohou být libovolné s~nulovým objemem.
%     \end{itemize}
%     Z kombinací obou nerovností lze vidět,~že vnější Lebesgueova míra $n$-rozměrného kvádru skutečně odpovídá jeho objemu,~tj.
%     \[\lambda_n^*(I)=\prod_{i=1}^{n}(b_i-a_i).\]
% \end{example}
\begin{remark}
    Vraťme se na chvíli ještě k~větě \ref{thm:mira-vlastnosti} o~vlastnostech míry,~konkrétně bod \ref{thm:mira-nerost-posl}. Předpoklad $\mu(A_1)<\infty$ zde vynechat nelze. Stačí vzít množiny $A_j=\langle j,\infty)$,~tzn. $\lambda_n^*(A_j)=\lambda_n^*(\langle j,\infty))=\infty$ pro každé $j\in\N$. Snadno si rozmyslíme,~že
    \[\bigcap_{i=1}^\infty A_i=\emptyset,\]
    nicméně lze vidět,~že zatímco $\lim_{j\to\infty}\mu(A_j)=\infty$,~tak $\mu\left(\bigcap_{i=1}^\infty A_i\right)=0$.
\end{remark}

Z příkladů \ref{ex:lebegueova-mira-trivialni-priklady} a~\ref{prop:lebegueova-mira-objem-kvadru} můžeme vidět,~že pro rozumně zvolené množiny zachycuje vnější Lebesgueova míra jejich intuitivní "velikost". V~případě intervalu odpovídá jeho délce,~v~případě diskrétní množiny je nulová a~podobně např.~pro obdélník odpovídá jeho obsahu,~pro kvádr jeho objemu,~atd.

Nyní se však nabízí jedna otázka. Čtenář by mohl již od chvíle,~kdy jsme zavedli pojem vnější Lebesgueovy míry (opět viz definice \ref{def:vnejsi-lebegueova-mira}) namítat,~co nás opravňuje nazývat zobrazení $\lambda_n^*$ mírou,~tj. ve smyslu definice \ref{def:prostor-s-mirou}. Jak víme,~že splňuje podmínku $\sigma$-aditivity? Odpověď na tuto otázku není zcela přímočará a~vlastně není ani jednoduchá.

Bohužel pro libovolně zvolenou množinu $X$ a~$\sigma$-algebru $\mathcal{A}$ v~případě vnější Lebesgueovy míry obecně neplatí vlastnost aditivity,~tedy existují množiny $A,B\in\mathcal{A}$,~takové,~že
\[\lambda_n^*(A\cup B)\neq\lambda_n^*(A)+\lambda_n^*(B).\]
Příklad takové množiny využívá např. takzvaná \emph{Vitaliho konstrukce}\index{Vitaliho konstrukce},~se kterou přišel italský matematik \name{Giuseppe Vitali} (1875--1932) roku 1905,~využívající invariance vnější Lebesgueovy míry vůči posunutí,~tzn. $\lambda_n^*(x+A)=\lambda_n^*(A)$. \cite{OConnor2025} V~rámci tohoto textu se jí zde zabývat nebudeme,~avšak pro zájemce doporučuji zdroje \citep[str. 3]{Lukes2013} a~\cite{Verner2025},~kde je tato konstrukce podrobněji rozepsána.

Je tedy potřeba se omezit na takové množiny,~kde je $\lambda_n^*$ aditivní. Existuje více způsobů jejich charakterizace,~avšak my si zde uvedeme ten,~se kterým přišel řecký matematik \name{Constantin Carathéodory} (1873--1950).
\begin{definition}[Lebesgueovská měřitelnost]\label{def:lebesgueovska-meritelnost}
    Množinu $A\subseteq\R^n$ nazveme (lebesgueovsky) měřitelnou\index{lebesgueovsky měřitelná množina},~pokud pro každou množinu $G$ platí
    \[\lambda_n^*(G)=\lambda_n^*(G\cap A)+\lambda_n^*(G\setminus A).\]
    Systém všech měřitelných množin v~$\R^n$ značíme $\mathcal{L}^n$.  Pokud $A\in\mathcal{L}^n$,~pak číslo $\lambda_n(A)=\lambda_n^*(A)$ nazýváme $n$-rozměrnou Lebesgueovou mírou\index{$n$-rozměrná Lebesgueova míra} množiny $A$.
\end{definition}
Podmínka v~definici \ref{def:lebesgueovska-meritelnost} se někdy nazývá říká \emph{Carathéodoryho kritérium}. Zjednodušeně říká,~že množina $A$ je lebesgueovsky měřitelná,~když při "rozdělení" \emph{libovolně} zvolené množiny $G$ na dvě části pomocí $A$ lze míru $G$ stanovit součtem měr daných částí (viz obrázek \ref{fig:caratheodoryho-kriterium}). Zároveň je dobré podotknout, že kvádry, které figurují v definici vnější Lebesgueovy míry jsou měřitelné.
\begin{figure}[h]
    \centering
    \includegraphics{ch02-caratheodoryho-kriterium.pdf}
    \caption{Ilustrace měřitelnosti množiny $A$}
    \label{fig:caratheodoryho-kriterium}
\end{figure}
O systému $\mathcal{L}^n$ a Lebesgueově míře $\lambda_n$ lze dokázat následující tvrzení.
\begin{theorem}\label{thm:prostor-s-Lebesgueovou-mirou}
    Platí:
    \begin{enumerate}[label=(\roman*)]
        \item $(\R^n,\mathcal{L}^n)$ je měřitelný prostor.
        \item $(\R^n,\mathcal{L}^n,\lambda_n)$ je prostor s~mírou.
    \end{enumerate}
\end{theorem}
Čtenář snad promine,~že formální důkaz v~případě tohoto tvrzení v~zájmu zachování stručnosti textu zcela vynecháme,~nicméně "zvídavec" jej může nalézt např. v~knize \citep[str. 347]{Royden2010},~kde jsou příslušné záležitosti rozepsány.
\section{Box-counting dimenze}\label{sec:box-counting-dimenze}

Tomuto typu dimenze jsme se již v~základu věnovali v~kapitole \ref{chapter:uvod_do_fraktalu},~specificky sekci \ref{sec:fraktalni_dimenze},~kde jsme rozebrali jeho způsob jejího výpočtu a ukázali jsme si jej několika příkladech. V~této části si blíže rozebereme některé další vlastnosti týkající se právě \emph{box-counting dimenze}\index{box-counting dimenze}\footnote{V kapitole \ref{chapter:uvod_do_fraktalu} jsme pro jednoduchost používali obecnější termín \emph{fraktální dimenze}. Ten však zahrnuje daleko širší škálu možných definic,~než jen tu,~kterou jsme si představili. Avšak dále v~tomto textu budeme používat výhradně její skutečný název,~tj. box-counting dimenze.} a pokusíme se ji lépe zasadit do kontextu teorie míry,~které jsme se samostatně až do této chvíle věnovali.

\subsection{Definice a výpočet}\label{subsec:definice-a-vypocet-bc-dimenze}

Jako první se podíváme na myšlenku box-counting trochu blíže a maličko si ji zobecníme. Původně jsme nahlíželi na dimenzi jako na exponent,~s nímž roste "velikost" zkoumaného útvaru. Tato myšlenka nás se ukázala jako rozumná,~neboť pro "klasické" geometrické útvary vycházela tato dimenze vždy celočíselně,~nicméně už tomu tak nebylo v~případě fraktálních útvarů. Myšlenka byla taková,~že jsme útvar $F$ rozdělili na určitý počet stejně "velkých částí",~označme $F_1,F_2,\ldots,F_m$,~v nějakém měřítku $\varepsilon>0$. Zkusme nyní požadavek na striktně stejnou velikost (formálně vzato míru) trochu rozvolnit. Bude nám stačit,~když pro každé $i$ je $\diam{F_i}\leqslant\delta$,~kde $\delta>0$. Zároveň nebudeme požadovat,~aby množiny\linebreak{}$F_1,F_2,\ldots,F_m$ byly všechny striktně po dvou téměř disjunktními\footnote{Množiny $M,N$ jsou \emph{téměř disjunktní},~pokud $\interior{M}\cap\interior{N}=\emptyset$,~tedy může nastat,~že se na hranici mohou "dotýkat",~tzn. $\boundary{M}\cap\boundary{N}\neq\emptyset$.} podmnožinami $F$,~ale stačí,~když budou tvořit pokrytí $F$.

Mějme tedy nějakou neprázdnoou omezenou množinu $F\subset\R^n$,~kde pro každé $\delta>0$ budeme hledat \emph{nejmenší počet} množin,~takových,~že pokrývají $F$. Toto číslo si označíme $N_\delta(F)$. Dimenze množiny $F$ by tedy měla odrážet "rychlost" růstu $N_\delta(F)$ pro $\delta\to 0^+$. Je-li splněna aproximace
\begin{equation}\label{eq:odhad-n-delta}
    N_\delta(F)\approx c\delta^{-s}
\end{equation}
pro $c>0$,~pak řekneme,~že množina $F$ má box-counting dimenzi $s$. (Převzato z~\citep[str. 27]{Falconer2014}.)
\begin{remark}
    V~dalším textu budeme místo $\delta\to 0^+$ psát pro jednoduchost pouze $\delta\to 0$,~byť by se slušilo používat první variantu. Čtenáři je však nejspíše jasné,~že uvažovat záporný průměr množiny nemá smysl.
\end{remark}
Logaritmováním a úpravou výrazu \eqref{eq:odhad-n-delta} dostaneme:
\begin{align}\label{eq:odvozeni-box-counting-dimenze}
    \ln{N_\delta(F)}&\approx\ln{c}+\ln{\delta^{-s}}\\
    \ln{N_\delta(F)}&\approx\ln{c}-s\ln{\delta}\\
    s&\approx\dfrac{\ln{N_\delta(F)}}{-\ln{\delta}}+\dfrac{\ln{c}}{\ln{\delta}}.
\end{align}
Když porovnáme výsledek v~\eqref{eq:odvozeni-box-counting-dimenze} s~rovností \eqref{eq:fraktalni-dimenze} z~minulé kapitoly,~můžeme si všimnout,~že zde navíc figuruje člen $\ln{c}/\ln{\delta}$. Když však uvážíme limitu daného výrazu pro $\delta\to 0$,~dostaneme původní vzorec,~který jsme již viděli,~tj.
\[\lim_{\delta\to 0}\left(\dfrac{\ln{N_\delta(F)}}{-\ln{\delta}}+\dfrac{\ln{c}}{\ln{\delta}}\right)=\lim_{\delta\to 0}\dfrac{\ln{N_\delta(F)}}{-\ln{\delta}}+\lim_{\delta\to 0}\dfrac{\ln{c}}{\ln{\delta}}=\lim_{\delta\to 0}\dfrac{\ln{N_\delta(F)}}{-\ln{\delta}}.\]
% \begin{definition}[$\delta$-pokrytí]\label{def:delta-pokryti}
%     Nechť $(X,\rho)$ je metrický prostor,~$F\subseteq X$ a $\delta>0$. Jestliže $\mathcal{F}=\set{F_1,F_2,\ldots}\subseteq\powset{X}$ je pokrytí $F$ a zároveň $\diam{F_j}\leqslant\delta$ pro každé $j$,~pak $\mathcal{F}$ nazýváme $\delta$-pokrytí\index{$\delta$-pokrytí} množiny $F$.
% \end{definition}
Předchozí úvahu můžeme shrnout do následující definice.
\begin{definition}[Box-counting dimenze]\label{def:box-counting-dimenze}
    Nechť $F\subset \R^n$ je neprázdná omezená množina. Pak definujeme následující:
    \begin{enumerate}[label=(\alph*)]
        \item \emph{Nejmenší počet množin v~$\delta$-pokrytí množiny $F$} značíme $N_\delta(F)$,~tj.
        \[N_\delta(F)=\inf\set{m\in\N_0\;\middle|\;F\subseteq\bigcup_{i=1}^n F_i\;,\;\diam{F_j}\leqslant\delta\;\text{pro}\;1\leqslant j\leqslant m}.\]
        \item \emph{Horní box-counting dimenze}\index{box-counting dimenze!horní box-counting dimenze} množiny $F$ je
        \[\upperdimB{F}=\limsup_{\delta\to 0}\dfrac{\ln{N_\delta(F)}}{-\ln{\delta}}.\]
        \item \emph{Dolní box-counting dimenze}\index{box-counting dimenze!dolní box-counting dimenze} množiny $F$ je
        \[\lowerdimB{F}=\liminf_{\delta\to 0}\dfrac{\ln{N_\delta(F)}}{-\ln{\delta}}.\]
    \end{enumerate}
    V~případě,~že $\lowerdimB{F}=\upperdimB{F}$,~pak společnou hodnotu nazýváme \emph{box-counting dimenzí}\index{box-counting dimenze} množiny $F$,~značíme $\dimB{F}$,~přičemž platí
    \[\dimB{F}=\lim_{\delta\to 0}\dfrac{\ln{N_\delta(F)}}{-\ln{\delta}}.\]
\end{definition}
\begin{remark}
    Zde je důležité zmínit,~že pro v~dalším textu budeme uvažovat $\delta$ dostatečně malé,~takové,~že hodnota $-\ln{\delta}$ je vždy kladná. Dále též budeme pracovat (podle definice \ref{def:box-counting-dimenze}) pouze s~neprázdnými omezenými množinami,~abychom se vyhnuli problémům s~případy,~kdy $N_\delta(F)=\infty$ nebo $N_\delta(F)=0$.
\end{remark}
Abychom uvedli vše na pravou míru,~box-counting dimenzi lze taktéž definovat více způsoby. V~tuto uvažujeme obecně $\delta$-pokrytí dané množiny $F$,~tj. pokrytí \emph{obecnými} množinami o~průměru maximálně $\delta>0$. Lze se však zaměřit i~na konkrétní útvary,~jak ukazuje následující věta.
\begin{theorem}[Ekvivalentní definice box-counting dimenze]\label{thm:ekvivalentni-def-box-counting-dimenze}
    Nechť $F\subset \R^n$ je neprázdná omezená množina. Pak
    \begin{align*}
        \lowerdimB{F}&=\liminf_{\delta\to 0}\dfrac{\ln{M_\delta(F)}}{-\ln{\delta}},\\
        \upperdimB{F}&=\limsup_{\delta\to 0}\dfrac{\ln{M_\delta(F)}}{-\ln{\delta}},\\
        \dimB{F}&=\lim_{\delta\to 0}\dfrac{\ln{M_\delta(F)}}{-\ln{\delta}},
    \end{align*}
    kde pro $M_\delta(F)$ platí
    \begin{enumerate}[label=(\roman*)]
        \item\label{thm:pokryti-delta-uz-koulemi} $\displaystyle M_\delta(F)=\inf\set{m\;\middle|\;F\subseteq\bigcup\limits_{i=1}^m K_\delta(x_i)\;,\;x_j\in\R^n\;\text{pro}\;1\leqslant j\leqslant m}$,
        \item\label{thm:pokryti-delta-kvadry} $\displaystyle M_\delta(F)=\inf\set{m\;\middle|\;F\subseteq\bigcup\limits_{i=1}^m I_i\;,\;I_j=\prod_{k=1}^{n}\langle a_k,a_k+\delta\rangle\;\text{pro}\;1\leqslant j\leqslant m}$,
        \item\label{thm:pokryti-delta-sit} $\displaystyle M_\delta(F)=\left|\set{I\;\middle|\;I\cap F\neq\emptyset\;,\;I\in\mathcal{D}}\right|$,~kde $\mathcal{D}$ je $\delta$-síť.
        \item\label{thm:pokryti-delta-dis-ot-koulemi} $\displaystyle M_\delta(F)=\sup\set{m\;\middle|\;B_\delta(x_i)\cap B_\delta(x_j)=\emptyset\;;\;x_i,x_j\in\R^n\;\text{pro}\;1\leqslant i,j\leqslant m}$.
    \end{enumerate}
\end{theorem}

Pojďme si větu \ref{thm:ekvivalentni-def-box-counting-dimenze} nyní trochu rozebrat.
\begin{itemize}
    \item Body \ref{thm:pokryti-delta-uz-koulemi} a \ref{thm:pokryti-delta-dis-ot-koulemi} říkájí,~že $N_\delta(F)$ je rovno \emph{nejmenšímu počtu uzavřených koulí o~poloměru $\delta$,~které pokrývají $F$},~resp. \emph{nejvyšší počet disjunktních otevřených koulí,~které mají střed v~$F$}.
    \item Podobně body \ref{thm:pokryti-delta-kvadry} a \ref{thm:pokryti-delta-sit} tvrdí,~že $N_\delta(F)$ lze ekvivalentně definovat jako pokrytí kvádry o~stranách délky $\delta$,~resp. počet všech kvádrů z~$\delta$-sítě,~které mají s~$F$ neprázdný průnik.
\end{itemize}
Pro představu viz obrázek \ref{fig:ilustrace-definic-bc-dimenze}. Důkaz věty je delší a opět jej vynecháme,~nicméně lze jej nalézt v~knize \citep[str. 30]{Falconer2014}.
\begin{figure}[h]
    \centering
    \begin{subfigure}{0.4\textwidth}
        \centering
        \includegraphics{ch02-bc-dimenze.pdf}
        \caption{Množina $B=\bigcup_{i=1}^4 B_i$}
        \label{subfig:bc-dimenze-pokryvana-mnozina}
    \end{subfigure}
    \qquad
    \begin{subfigure}{0.4\textwidth}
        \centering
        \includegraphics{ch02-bc-dimenze-delta-pokryti.pdf}
        \caption{$\delta$-pokrytí množiny $B$ (viz definice \ref{def:box-counting-dimenze})}
        \label{subfig:bc-dimenze-delta-pokryti}
    \end{subfigure}
    \qquad
    \begin{subfigure}{0.4\textwidth}
        \centering
        \includegraphics{ch02-bc-dimenze-pokryti-uz-koule.pdf}
        \caption{Pokrytí uzavřenými koulemi (viz bod \ref{thm:pokryti-delta-uz-koulemi})}
        \label{subfig:bc-dimenze-uz-koule}
    \end{subfigure}
    \qquad
    \begin{subfigure}{0.4\textwidth}
        \centering
        \includegraphics{ch02-bc-dimenze-pokryti-kvadry.pdf}
        \caption{Pokrytí pomocí kvádrů (viz bod \ref{thm:pokryti-delta-kvadry})}
        \label{subfig:bc-dimenze-kvadry}
    \end{subfigure}
    \qquad
    \begin{subfigure}{0.4\textwidth}
        \centering
        \includegraphics{ch02-bc-dimenze-pokryti-delta-sit.pdf}
        \caption{$\delta$-síť (viz bod \ref{thm:pokryti-delta-sit})}
        \label{subfig:bc-dimenze-delta-sit}
    \end{subfigure}
    \qquad
    \begin{subfigure}{0.4\textwidth}
        \centering
        \includegraphics{ch02-bc-dimenze-pokryti-ot-koule.pdf}
        \caption{Pokrytí otevřenými po dvou disjunktními koulemi (viz bod \ref{thm:pokryti-delta-dis-ot-koulemi})}
        \label{subfig:bc-dimenze-ot-koule}
    \end{subfigure}
    \caption{Ilustrace věty \ref{thm:ekvivalentni-def-box-counting-dimenze} (Inspirováno \citep[str. 29]{Falconer2014})}
    \label{fig:ilustrace-definic-bc-dimenze}
\end{figure}

Zároveň body \ref{thm:pokryti-delta-kvadry} a \ref{thm:pokryti-delta-sit} nám dávají dobré opodstatnění názvu tohoto typu dimenze,~neboť v~podstatě zkoumáme pokrývání daného obrazce "kostkami". Při aproximacích box-counting dimenze obrazce $F\subset\R^2$ tak lze pracovat s~mřížkou čtverců o~libovolné straně $\delta>0$,~kdy $N_\delta(F)$ stanovíme jako počet čtverců,~které se překrývají se zkoumaným obrazcem $F$. Když se tedy zpět vrátíme k~otázce rozebírané v~úvodu tohoto textu týkající se délky pobřeží (viz kapitola \ref{chapter:uvod_do_fraktalu}),~lze jeho "fráktálnost" do jisté míry vyjádřit právě popsaným způsobem (viz obrázek \ref{fig:aproximace-delky-pobrezi-vb}).
\begin{figure}[h]
    \centering
    \includegraphics[width=\textwidth]{Great_Britain_Box.pdf}
    \caption[Aproximace box-counting dimenze pobřeží Velké Británie]{Aproximace box-counting dimenze pobřeží Velké Británie. Převzato z~Wikipedia Commons,~\url{https://en.wikipedia.org/wiki/Minkowski\%E2\%80\%93Bouligand\_dimension\#/media/File:Great\_Britain\_Box.svg}. Viz též \url{https://en.wikipedia.org/wiki/Minkowski\%E2\%80\%93Bouligand\_dimension}.}
    \label{fig:aproximace-delky-pobrezi-vb}
\end{figure}
Nyní se opět vrátíme k~fraktálům a výpočtům jejich dimenze,~čemuž jsme se věnovali již v~sekci \ref{sec:fraktalni_dimenze} kapitoly \ref{chapter:uvod_do_fraktalu},~konkrétně \ref{subsec:dimenze-fraktalu}. Tentokrát však budeme postupovat přímo podle definice box-counting dimenze \ref{def:box-counting-dimenze},~tedy budeme zvlášť zkoumat horní a dolní box-counting dimenzi.
\begin{example}[Cantorovo diskontinuum]\label{ex:cantorovo-diskontinuum}
    Formálně můžeme popsat Cantorovo diskontinuum $C$ jako průnik množin $C_k$ pro $k=0,1,2,\ldots$, přičemž
    \[C_k=\bigcup _{j=0}^{3^{k-1}-1}\left(\left\langle{\frac {3j+0}{3^{k}}},{\frac {3j+1}{3^{k}}}\right\rangle\cup \left\langle{\frac {3j+2}{3^{k}}},{\frac {3j+3}{3^{k}}}\right\rangle\right).\]
    $C_k$ tedy reprezentuje $k$-tou iteraci a~$C=\bigcap_{k=0}^\infty C_k$. Již jsme měli možnost se přesvědčit,~že tento fraktál má box-counting dimenzi $\ln{2}/\ln{3}$. Zkusme nyní výpočet zopakovat,~avšak vzlášť vypočítáme $\lowerdimB{C}$ a $\upperdimB{C}$ podíváme se,~zda se shodují.

    Jako první provedeme horní odhad. Je potřeba zvolit $\delta$ a na jeho základě dopočítat $N_\delta(C)$. V~$k$-té iteraci,~kde $k=0,1,2,\ldots$,~bude obecně $2^k$ intervalů,~každý o~délce $(1/3)^k$,~tedy pokud zvolíme $3^{-k}<\delta\leqslant 3^{-k+1}$,~pak intervaly o~délce nejvýše $\delta$ (viz věta \ref{thm:ekvivalentni-def-box-counting-dimenze},~bod \ref{thm:pokryti-delta-uz-koulemi}) tvoří $\delta$ pokrytí,~přičemž $N_\delta(C)\leqslant 2^k$. Tedy celkově pro $\delta$-pokrytí všech intervalů bude potřeba nejvýše $N_\delta(C)\leqslant 2^k$ intervalů $I_1,I_2,\ldots,I_{N_\delta(C)}$ o~průměru $3^{-k}<\diam{F_i}\leqslant 3^{-k+1}$ pro každé $i$. Z~toho dostáváme
    \[\upperdimB{C}=\limsup_{\delta\to 0}\dfrac{\ln{N_\delta(C)}}{-\ln{\delta}}\leqslant\limsup_{k\to\infty}\dfrac{\ln{2^k}}{-\ln{3^{-k+1}}}=\limsup_{k\to\infty}\dfrac{k\ln{2}}{(k-1)\ln{3}}=\dfrac{\ln{2}}{\ln{3}}.\]
    Naopak pokud uvážíme intervaly délky $3^{-k-1}\leqslant\delta<3^{-k}$,~pak každý z~nich má neprázdný průnik s~maximálně jedním intervalem $k$-té iterace $C$. Těch je,~jak již víme,~$2^k$,~tedy intervalů $I_1,I_2,\ldots,I_{N_\delta(C)}$ bude nejméně $2^k$ pro pokrytí $C$,~tzn. $N_\delta(C)\geqslant 2^k$. Tím dostáváme dolní odhad:
    \[\lowerdimB{C}=\liminf_{\delta\to 0}\dfrac{\ln{N_\delta(C)}}{-\ln{\delta}}\geqslant\liminf_{\delta\to 0}\dfrac{\ln{2^k}}{-\ln{3^{-k-1}}}=\liminf_{\delta\to 0}\dfrac{k\ln{2}}{(k+1)\ln{3}}=\dfrac{\ln{2}}{\ln{3}}.\]

    Protože $\lowerdimB{C}=\upperdimB{C}=\ln{2}/\ln{3}$,~tak box-counting dimenze Cantorova diskontinua je $\dimB{C}=\ln{2}/\ln{3}$. (Převzato z~\citep[str. 32]{Falconer2014})
\end{example}
Podobně bychom postupovali pro rovinné obrazce.
\begin{example}[Kochova křivka]\label{ex:kochova-krivka}
    Zde se zatím s~formální definicí nebudeme zatěžovat. Opět ukážene horní a~dolní odhad zvlášť. Kochovu křivku si označíme $K$.

    Obecně $k$-tá iterace Kochovy křivky bude obsahovat $4^n$ úseček,~každá o~délce $(1/3)^k$. Podobně jako v~předchozím příkladu \ref{ex:cantorovo-diskontinuum} zvolíme $3^{-k}<\delta\leqslant 3^{-k+1}$. Pokud si pro pokrytí zvolíme uzavřené koule
    \[K_\delta^1(x_1),K_\delta^2(x_2),\ldots,K_\delta^{N_\delta(K)}(x_{N_\delta(K)}),\;\text{kde}\;x_1,x_2,\ldots,x_{N_\delta(K)}\in\R^2\]
    pak $N_\delta(K)\leqslant 4^k$. Tedy
    \[\upperdimB{K}=\limsup_{\delta\to 0}\dfrac{\ln{N_\delta(K)}}{-\ln{K}}\leqslant\limsup_{k\to\infty}\dfrac{\ln{4^k}}{-\ln{3^{-k+1}}}=\limsup_{k\to\infty}\dfrac{k\ln{4}}{(k-1)\ln{3}}=\dfrac{\ln{4}}{\ln{3}}.\]

    Podobně pro dolní odhad uvážíme $3^{-k-1}\leqslant\delta<3^{-k}$. Vezmeme-li uzavřené koule $K_\delta^1(x_1),K_\delta^2(x_2),\ldots,K_\delta^{N_\delta(K)}(x_{N_\delta(K)})$,~pak žádný nemůže mít neprázdný průnik s~více než čtyřmi úsečkami,~tedy pro jejich pokrytí je zapotřebí alespoň $N_\delta(K)\geqslant 4^k/4=4^{k-1}$,~čímž dostáváme
    \[\lowerdimB{K}=\liminf_{\delta\to 0}\dfrac{\ln{N_\delta(K)}}{-\ln{K}}\geqslant\liminf_{k\to\infty}\dfrac{\ln{4^{k-1}}}{-\ln{3^{-k-1}}}=\liminf_{k\to\infty}\dfrac{(k-1)\ln{4}}{(k+1)\ln{3}}=\dfrac{\ln{4}}{\ln{3}}.\]
    Tzn.~$\dimB{K}=\ln{4}/\ln{3}$.
\end{example}
\begin{remark}
    Obecně množina $F$ skládající se z~$m$ disjunktních kopií sebe samotné v~měřítku $r$ má dimenzi $\dimB{F}=-\ln{m}/\ln{r}$.
\end{remark}
Nyní se podíváme ještě na jedno možné pojetí box-counting dimenze. Připomeňme,~že $\delta$-okolím množiny $F$ v~metrickém prostoru $(X,\rho)$ rozumíme
\[F_\delta=\set{x\in X\mid\exists y\in X: \rho(x,y)<\delta}.\]
Budeme nyní sledovat,~jak "rychle" se mění objem $F_\delta$ pro $\delta\to 0$. A~se zmínkou objemu nám zde do hry opět vstupuje Lebesgueova míra $\lambda_n$,~o níž jsme si povídali v~sekci \ref{sec:lebesgueova-mira}. Podívejme se nejdříve na několik příkladů.
\begin{itemize}
    \item Pro úsečku $u\subset\R^3$ o~délce $\ell$ lze objem jejího $\delta$-okolí stanovit jako
    \[\lambda_3(u)=\dfrac{4}{3}\pi \delta^3+\pi\delta^2\ell.\]
    Pokud však uvážíme $\delta$ dostatečně malé,~lze první člen zanedbat a psát
    \[\lambda_3(u)\approx\pi\ell\delta^2.\]
    \item V~případě uzavřené omezené množiny $F$ o~obsahu $A$ je objem $\lambda_3(F_\delta)\approx 2A\delta$.
    \item Pro kouli $B_r(x)\subset\R^3$,~kde $x\in\R^3$ a $r>0$ je objem
    \[\lambda_3((B_r(x))_\delta)=\dfrac{4}{3}\pi (r+\delta)^3=\dfrac{4}{3}\pi r^3+4\pi r^2\delta+4\pi r\delta^2+\dfrac{4}{3}\pi\delta^3\approx\dfrac{4}{3}\pi r^3.\]
    Změna objemu je v~tomto případě vzhledem k~$\delta$ zanedbatelná.
\end{itemize}
Můžeme si všimnout,~že v~každém případě odhad objemu vychází $\lambda_3(F)\approx c\delta^{3-s}$,~kde $c>0$ je závislé na původní míře $F$ a $s$ udává dimenzi. Obecněji pro množinu $F\subseteq\R^n$ bychom došli k~$\lambda_n(F)\approx c\delta^{n-s}$. Nyní,~podobně jako v~úvodu této sekce,~zkusme opět vyjádřit $s$:
\begin{align*}
    \ln{\lambda_n(F)}&\approx\ln{c}+(n-s)\ln{\delta}\\
    s\ln{\delta}&\approx n\ln{\delta}-\ln{\lambda_n(F)}+\ln{c}\\
    s&\approx n-\dfrac{\ln{\lambda_n(F)}}{\ln{\delta}}+\dfrac{\ln{c}}{\ln{\delta}}.
\end{align*}
Poslední člen bude v~limitě opět nulový.

Lze ukázat,~že $s$ není v~tomto případě nic jiného,~než již námi zkoumaná\linebreak{}box-counting dimenze. To si shrneme a dokážeme ve větě \ref{thm:bc-dimenze-lebesgueova-mira}.
\begin{theorem}\label{thm:bc-dimenze-lebesgueova-mira}
    Nechť $F\in\mathcal{L}^n$ je uzavřená a omezená. Pak platí: 
    \begin{enumerate}[label=(\roman*)]
        \item $\lowerdimB{F}=n-\liminf\limits_{\delta\to 0}\dfrac{\ln{\lambda_n(F_\delta)}}{\ln{\delta}}$,
        \item $\upperdimB{F}=n-\limsup\limits_{\delta\to 0}\dfrac{\ln{\lambda_n(F_\delta)}}{\ln{\delta}}$.
    \end{enumerate}
\end{theorem}
\begin{proof}
    V~rámci důkazu využijeme větu \ref{thm:ekvivalentni-def-box-counting-dimenze}.

    Mějme $F\in\mathcal{L}^n$ omezenou a uzavřenou. Označme $v_n$ objem jednotkové koule $K_1(x)$\footnote{Objem koule v~$\R^n$ lze vyjádřit vztahem
    \[V_n(r)=\dfrac{\pi^{n/2}}{\Gamma\left(\frac{n}{2}+1\right)}r^n,\]
    kde $\Gamma$ je tzv. \emph{gamma funkce}. Se vzorcem však dále v~textu pracovat nebudeme.
    } v~$\R^n$ pro $x\in\R^n$ libovolné. Dále mějmě pokrytí
    \[\mathcal{K}=\set{K_\delta^1(x_1),K_\delta^2(x_2),\ldots,K_\delta^{N_\delta(F)}(x_{N_\delta(F)})}\]
    množiny $F$,~kde $0<\delta<1$ a $x_j\in F$ pro každé $1\leqslant j\leqslant N_\delta(F)$. Pak lze zvolit pokrytí
    \[\mathcal{K}^\prime=\set{K_{2\delta}^1(x_1),K_{2\delta}^2(x_2),\ldots,K_{2\delta}^{N_\delta(F)}(x_{N_\delta(F)})},\]
    tzn.~$\mathcal{K}$ je zjemnění pokrytí $\mathcal{K}^\prime$. Zároveň však platí,~že $\mathcal{K}$ je i~pokrytím $F_\delta$. Pro libovolné $x\in F_\delta$ existuje totiž $y\in F$,~takové,~$\rho(x,y)<\delta$. Tedy pro dané $y$ existuje nějaká koule $K_\ell(x_\ell)\in\mathcal{K}$,~taková,~že $y\in K_\ell(x_\ell)$,~což znamená,~že
    \[\rho(x_\ell,y)\leqslant\rho(x_\ell,x)+\rho(x,y)\leqslant\delta+\delta=2\delta.\]
    Tzn.~míru $F$ lze zhora odhadnout jako
    \[\lambda_n(F_\delta)\leqslant N_\delta(F)v_n(2\delta)^n.\]
    Úpravou získáme:
    \begin{align*}
        \ln{\lambda_n(F_\delta)}&\leqslant n\ln{\delta}+\ln{N_\delta(F)}+\ln{2^nv_n}\\
        \dfrac{\ln{\lambda_n(F_\delta)}}{-\ln\delta}&\leqslant -n+\dfrac{\ln{N_\delta(F)}}{-\ln\delta}+\dfrac{\ln{2^nv_n}}{-\ln\delta},
    \end{align*}
    tedy v~limitě
    \[\liminf_{\delta\to 0}\dfrac{\ln\lambda_n(F_\delta)}{-\ln\delta}\leqslant -n+\lowerdimB{F}.\]
    K~odhadu $\upperdimB{F}$ lze dospět analogicky.

    Nyní uvažujme po dvou disjunktní otevřené koule $B_\delta^j(x_j)$,~kde $x_j\in F$ pro $1\leqslant j\leqslant N_\delta(F)$. Pak součtem jejich objemů získáme
    \[N_\delta(F)v_n\delta^n\leqslant\lambda_n(F_\delta).\]
    Obdobnou úpravou této nerovnosti získáme požadovanou nerovnost.
\end{proof}
Zkusme si aplikaci věty ilustrovat opět na příkladu fraktálu.
\begin{example}[Cantorovo diskontinuum potřetí]
    Pro Cantorovo diskontinuum $C$ v~$k$-té iteraci lze odhadnout délku $C_\delta$ pro $3^{-k}\leqslant\delta\leqslant 3^{-k+1}$ jako
    \[\lambda_1(C_\delta)\leqslant2^k\cdot 2\delta=2^{k+1}3^{-k+1}.\]
    Tedy podle věty \ref{thm:bc-dimenze-lebesgueova-mira}
    \begin{align*}
        \upperdimB{C}&=n-\limsup_{\delta\to 0}\dfrac{\ln{\lambda_1(C_\delta)}}{\ln{\delta}}\leqslant1-\limsup_{k\to\infty}\dfrac{\ln{2^{k+1}3^{-k+1}}}{\ln{3^{-k+1}}}\\
        &=\limsup_{k\to\infty}\dfrac{(k+1)\ln{2}}{(k-1)\ln{3}}=\dfrac{\ln{2}}{\ln{3}}.
    \end{align*}

    Podobně zvolíme-li $3^{-k-1}\leqslant\delta\leqslant 3^{-k}$,~pak
    \[\lambda_1(C_\delta)\geqslant 2^{k+1}3^{-k-1}\]
    a tedy
    \begin{align*}
        \lowerdimB{C}&=n-\liminf_{\delta\to 0}\dfrac{\ln{\lambda_1(C_\delta)}}{\ln{\delta}}\geqslant 1-\liminf_{k\to\infty}\dfrac{\ln{2^{k+1}3^{-k-1}}}{\ln{3^{-k-1}}}\\
        &=\liminf_{k\to\infty}\dfrac{(k+1)\ln{2}}{(k+1)\ln{3}}=\dfrac{\ln{2}}{\ln{3}}.
    \end{align*}
\end{example}

\subsection{Vlastnosti}\label{subsec:vlastnosti-bc-dimenze}

V minulé podsekci \ref{subsec:definice-a-vypocet-bc-dimenze} jsme se bavili o~možnostech pojetí box-counting dimenze. S~tím souvisely zejména pak věty \ref{thm:ekvivalentni-def-box-counting-dimenze} a \ref{thm:bc-dimenze-lebesgueova-mira}. Nyní trochu blíže ještě prozkoumáme některé její vlastnosti,~na něž se podíváme ve větě .
\begin{theorem}[Vlastnosti box-counting dimenze]\label{thm:vlastnosti-bc-dimenze}
    Nechť jsou dány $F,G\subseteq\R^n$.
    \begin{enumerate}[label=(\roman*)]
        \item\label{thm:monotonie-bc-dimenze} Pokud $G\subseteq F$,~pak $\lowerdimB{G}\leqslant\lowerdimB{F}$ a $\upperdimB{G}\leqslant\upperdimB{F}$.\rightnote{monotonie}
        \item\label{thm:rozsah-hodnot-bc-dimenze} Je-li $F\neq\emptyset$ omezená,~pak $0\leqslant\lowerdimB{F}\leqslant\upperdimB{F}\leqslant n$. \rightnote{rozsah hodnot}
        \item\label{thm:stabilita-bc-dimenze} $\upperdimB(F\cup G)=\max\set{\upperdimB{F},\upperdimB{G}}$.\rightnote{stabilita}
    \end{enumerate}
\end{theorem}
\begin{proof}
    \begin{enumerate}[label=\textit{(\roman*)}]
        \item Plyne triviálně z~faktu,~že pro libovolné $\delta>0$ je $N_\delta(G)\leqslant N_\delta(F)$,~neboť každé $\delta$-pokrytí $\mathcal{F}\supset F$ je zároveň $\delta$-pokrytím $G$.
        \item První dvojice nerovností je zjevná z~definice (viz \ref{def:box-counting-dimenze}). Pro třetí nerovnost zvolme kvádr $I$,~takový,~že $F\subset I$. Zvolíme-li $\delta>0$ a $\delta$-síť $\mathcal{D}$,~pak podle věty \ref{thm:ekvivalentni-def-box-counting-dimenze} je
        \[N_\delta(F)\leqslant N_\delta(I)=\left|\set{J\;\middle|\;J\cap I\neq\emptyset\;,\;J\in\mathcal{D}}\right|\leqslant c\delta^{-n},\]
        kde $c>0$. Tedy
        \[\upperdimB{F}\leqslant\upperdimB{I}=\limsup_{\delta\to 0}\dfrac{\ln{N_\delta(I)}}{-\ln{\delta}}\leqslant\limsup_{\delta\to 0}\dfrac{\ln{c\delta^{-n}}}{-\ln{\delta}}=n.\]
        \item Pro $\delta>0$ volme $\delta$-pokrytí $\mathcal{F}\supset F$ a $\mathcal{G}\supset G$. Je celkem zjevné,~že $N_\delta(F\cup G)\leqslant N_\delta(F)+N_\delta(G)$,~neboli
        \begin{align*}
            \ln(N_\delta(F)+N_\delta(G))&\leqslant \ln\left(2\max\set{N_\delta(F),N_\delta(G)}\right)\\
            &=\ln{2}+\ln\left(\max\set{N_\delta(F),N_\delta(G)}\right).
        \end{align*}
        Tedy
        \begin{align*}
            \upperdimB(F\cup G)&\leqslant\limsup_{\delta\to 0}\left(\dfrac{\ln{2}}{-\ln{\delta}}+\dfrac{\ln\left(\max\set{N_\delta(F),N_\delta(G)}\right)}{-\ln{\delta}}\right)\\
            &\leqslant\limsup_{\delta\to 0}\dfrac{\ln\left(\max\set{N_\delta(F),N_\delta(G)}\right)}{-\ln{\delta}}\\
            &=\limsup_{\delta\to 0}\left(\max\set{\dfrac{\ln{N_\delta(F)}}{-\ln{\delta}},\dfrac{\ln{N_\delta(G)}}{-\ln{\delta}}}\right)\\
            &\leqslant\max\set{\limsup_{\delta\to 0}\dfrac{\ln{N_\delta(F)}}{-\ln{\delta}},\limsup_{\delta\to 0}\dfrac{\ln{N_\delta(G)}}{-\ln{\delta}}}\\
            &=\max\set{\upperdimB{F},\upperdimB{G}}.
        \end{align*}
        Opačná nerovnost plyne z~faktu,~že $F\subset F\cup G$ a $G\subset F\cup G$,~tedy
        \[\upperdimB(F\cup G)\geqslant\upperdimB{F}\;\text{a}\;\upperdimB(F\cup G)\geqslant\upperdimB{G}\]
        podle bodu \ref{thm:monotonie-bc-dimenze},~neboli
        \[\upperdimB(F\cup G)=\max\set{\upperdimB{F},\upperdimB{G}}.\]
    \end{enumerate}
\end{proof}
(Převzato a upraveno z~\citep[str. 35]{Falconer2014}.)

Poslední bod \ref{thm:stabilita-bc-dimenze} tvrzení \ref{thm:vlastnosti-bc-dimenze} lze pochopitelně rozšířit indukcí. Čtenář se sám může přesvědčit,~že se jedná o~relativně jednoduché cvičení.
\begin{corollary}\label{cor:stabilita-bc-dimenze-obecne}
    Pro $F_1,F_2,\ldots,F_m\subseteq\R^n$ platí:
    \[\upperdimB\left(\bigcup_{i=1}^m F_i\right)=\max\set{\upperdimB{F_j}\mid 1\leqslant j\leqslant m}.\]
\end{corollary}
\begin{proof}
    Pro $m=1$ a $m=2$ víme,~že tvrzení platí. Pro $m+1$ lze psát:
    \begin{align*}
        \upperdimB\left(\bigcup_{i=1}^{m+1}F_i\right)&=\upperdimB\left(\left(\bigcup_{i=1}^{m}F_i\right)\cup F_{m+1}\right)\\
        &=\max\set{\upperdimB\left(\bigcup_{i=1}^{m+1}F_i\right),\upperdimB{F_{m+1}}}\\
        &\stackrel{\text{I.P.}}{=}\max\set{\max\set{\upperdimB{F_i}\mid 1\leqslant i\leqslant m},\upperdimB{F_{m+1}}}\\
        &=\max\set{\upperdimB{F_j}\mid 1\leqslant j\leqslant m+1}.
    \end{align*}
\end{proof}

Jako poslední se ještě nabízí otázka,~jak se bude dimenze $\dimB$ chovat vůči zobrazením. V~tomto kontextu pro nás budou relevantní především \emph{lipschitzovská}\index{lipschitzovské zobrazení} a \emph{bilipschitzovská zobrazení}\index{bilipschitzovské zobrazení}. Připomeňme,~že lipschitzovské zobrazení je takové zobrazení $\mapping{f}{X}{Y}$ mezi metrickými prostory $(X,\rho_X)$ a $(Y,\rho_Y)$,~že existuje konstanta $K>0$ a pro každé $x,y\in X$ platí
\[\rho_Y(f(x),f(y))\leqslant K\rho_X(x,y).\]
Pokud navíc platí,~že existují konstanty $K_1,K_2>0$,~takové,~že platí
\[K_1\rho_X(x,y)\leqslant\rho_Y(f(x),f(y))\leqslant K_2\rho_X(x,y),\]
pak $f$ nazýváme bilipschitzovské.

Než se však podíváme na samotný vztah box-counting dimenze a lipschitzovských,~resp. bilipschitzovských zobrazení,~dokážeme si jedno jednoduché\linebreak{}lemma,~které později využijeme.
\begin{lemma}\label{lem:lipschitzovska-zobrazeni-a-bijekce}
    Nechť $(X,\rho_X),(Y,\rho_Y)$ jsou metrické prostory a zobrazení $\mapping{f}{X}{Y}$ je bilipschitzovské. Pak $\mapping{f^\prime}{X}{f(X)}$ je bijekce.
\end{lemma}
\begin{proof}
    Podle předpokladu je $f$ bilipschitzovské zobrazení,~tedy i~$f^\prime$ je bilipschitzovské,~tedy existují pro konstanty $K_1,K_2>0$,~takové,~že
    \[K_1\rho_X(x,y)\leqslant\rho_Y(f^\prime(x),f^\prime(y))\leqslant K_2\rho_X(x,y),\;x,y\in X.\]
    Surjektivita zobrazení $f^\prime$ je zřejmá z~její definice. Pro spor předpokládejme,~že $f^\prime$ není injektivní,~tedy existují $x,y\in X$,~taková,~že $f^\prime(x)=f^\prime(y)$. Pak
    \[0<K_1\rho_X(x,y)\leqslant\rho_Y(f^\prime(x),f^\prime(y))=0,\]
    což je spor.
\end{proof}

V našem případě se nyní dále omezíme,~stejně jako předtím,~pouze na prostor $\R^n$.

\begin{theorem}\label{thm:bc-dimenze-bi-lipschitzovska-zobrazeni}
    Nechť jsou dány metrické prostory $(\R^n,\rho_n)$ a $(\R^m,\rho_m)$,~zobrazení $\mapping{f}{\R^n}{\R^m}$ a $F\subseteq\R^n$. Platí:
    \begin{enumerate}[label=(\roman*)]
        \item\label{thm:bc-dimenze-lipschitz} Je-li $f$ lipschitzovské,~pak
        \[\lowerdimB{f(F)}\leqslant\lowerdimB{F}\;\text{a}\;\upperdimB{f(F)}\leqslant\upperdimB{F}.\]
        \item\label{thm:bc-dimenze-bilipschitz} Je-li $f$ bilipschitzovské,~pak
        \[\lowerdimB{f(F)}=\lowerdimB{F}\;\text{a}\;\upperdimB{f(F)}=\upperdimB{F}.\]
    \end{enumerate}
\end{theorem}
\begin{proof}
    Máme tedy metrické prostory $(\R^n,\rho_n)$,~$(\R^m,\rho_m)$,~zobrazení $\mapping{f}{\R^n}{\R^m}$ a $F\subseteq\R^n$.
    \begin{enumerate}[label=\textit{(\roman*)}]
        \item Jako první si všimneme,~že je-li $\mathcal{F}=\set{F_1,F_2,\ldots}$ $\delta$-pokrytí,~množiny $F$,~kde $\delta>0$,~pak je jím i~systém
        \[\mathcal{F}^\prime=\set{F\cap F_1,F\cap F_2,\ldots}.\]
        Podle předpokladu je $f$ lipschitzovské,~tzn. pro každé $x,y\in\R^n$ je
        \[\rho_m(f(x),f(y))\leqslant K\rho_n(x,y),\;K>0.\]
        Speciálně tak platí i~$\diam(f(F\cap F_i))\leqslant K\diam(F\cap F_i)$ pro každé $i$,~a tedy
        \[\diam(f(F\cap F_i))\leqslant K\diam(F\cap F_i)\leqslant K\diam{F_i}\leqslant K\delta.\]
        Z~toho plyne,~že $\mathcal{G}=\set{f(F\cap F_1),f(F\cap F_2),\ldots}$ tvoří $K\delta$-pokrytí množiny $f(F)$. Tedy máme,~že $N_{K\delta}(f(F))\leqslant N_\delta(F)$. Po úpravě
        \[\dfrac{\ln{N_{K\delta}(f(F))}}{-\ln{\delta}}\leqslant\dfrac{\ln{N_\delta(F)}}{-\ln{\delta}}.\]
        Tedy celkově
        \begin{align*}
            \upperdimB{f(F)}&=\limsup_{\delta\to 0}\dfrac{\ln{N_{K\delta}(f(F))}}{-\ln{K\delta}}=\limsup_{\delta\to 0}\dfrac{\ln{N_{K\delta}(f(F))}}{-\ln{\delta}}\cdot\dfrac{\ln{\delta}}{\ln{K\delta}}\\
            &\leqslant\limsup_{\delta\to 0}\dfrac{\ln{N_{\delta}(f(F))}}{-\ln{\delta}}\cdot\dfrac{\ln{\delta}}{\ln{K\delta}}=\limsup_{\delta\to 0}\dfrac{\ln{N_{\delta}(f(F))}}{-\ln{\delta}}\\
            &=\upperdimB{F}.
        \end{align*}
        Nerovnost pro $\lowerdimB{f(F)}$ obdržíme analogicky.
        \item Je-li $f$ bilipschitzovské,~pak podle lemmatu \ref{lem:lipschitzovska-zobrazeni-a-bijekce} je $\mapping{g}{\R^n}{f(\R^n)}$ bijekce a existuje inverzní zobrazení $\mapping{g^{-1}}{f(\R^n)}{\R^n}$. Volme $u,v\in f(\R^n)$ libovolně a položme $x=g^{-1}(u),y=g^{-1}(v)$. Pak
        \[K_1\rho_n(x,y)=K_1\rho_n(g^{-1}(u),g^{-1}(v))\leqslant\rho_m(g(g^{-1}(u)),g(g^{-1}(v)))=\rho_m(u,v),\]
        neboli
        \[\rho_n(g^{-1}(u),g^{-1}(v))\leqslant\dfrac{1}{K_1}\rho_m(u,v),\]
        přičemž $K_1,K_2$ jsou konstanty z~definice. Tzn.~$g^{-1}$ je lipschitzovské.

        Podle bodu \ref{thm:bc-dimenze-lipschitz} tedy platí
        \begin{align*}
            \lowerdimB{F}&=\lowerdimB{g^{-1}(g(F))}\leqslant\lowerdimB{g(F)},\\
            \upperdimB{F}&=\upperdimB{g^{-1}(g(F))}\leqslant\upperdimB{g(F)}.
        \end{align*}
        Ovšem podle bodu \ref{thm:bc-dimenze-lipschitz} ovšem již víme,~že také platí
        \begin{align*}
            \lowerdimB{g(F)}&\leqslant\lowerdimB{F},\\
            \upperdimB{g(F)}&\leqslant\upperdimB{F}.
        \end{align*}
        Z~toho již plyne závěr tvrzení.
    \end{enumerate}
\end{proof}
(Převzato z~\citep[str. 36]{Falconer2014}.)

Právě dokázaná věta \ref{thm:bc-dimenze-bi-lipschitzovska-zobrazeni} (konkrétně bod \ref{thm:bc-dimenze-bilipschitz}) nám ve své podstatě říká,~že\linebreak{}box-counting dimenze nějakého útvaru $F$ je invariantní vůči libovolnému bilipschitzovskému zobrazení $f$. Tento výsledek se nám bude hodit dále v~kapitole \ref{chapter:klasifikace-fraktalu} u~tzv. \emph{systémů iterovaných funkcí}. 
\section{Hausdorffova míra a Hausdorffova dimenze}\label{sec:hausdorffova-mira-dimenze}

Způsobů,~jak definovat dimenzi je celá řada. Zatím jsme společně prozkoumali box-counting dimenzi (resp. některá její pojetí),~avšak lze najít více způsobů její definice\footnote{Některé další jsou sepsány např. v~\citep[str. 40]{Falconer2014}.}. Pravděpodobně však nejstarším exemplářem svého druhu je tzv. \emph{Hausdorffova dimenze} a s~ní související \emph{Hausdorffova míra},~které hrají ve fraktální geometrii velice podstatnou roli. Stále se však budeme zabývat pouze množinami v~$\R^n$. Jsou pojmenovány po německém matematikovi \name{Felixi Hausdorffovi} (1868--1942).
\begin{figure}[h]
    \centering
    \includegraphics[width=0.4\textwidth]{felix-hausdorff.jpg}
    \caption[Felix Hausdorff,~1868--1942]{Felix Hausdorff\footnote{Převzato z~\cite{OConnorHausdorff2025}},~1868--1942}
    \label{fig:felix-hausdorff}
\end{figure}

\subsection{Definice Hausdorffovy míry}\label{subsec:hd-mira-definice}

\begin{definition}\label{def:hd-mira-delta}
    Nechť je dána množina $F\subseteq\R^n$ a $s>0$. Pak pro každé $\delta>0$ definujeme zobrazení
    \[\hausdorffdeltameasure{s}{\delta}(F)=\inf\set{\sum_{i=1}^{\infty}(\diam{F_i})^s\;\middle|\;F\subseteq\bigcup_{i=1}^\infty F_i\;,\;\diam{F_j}\leqslant\delta\;\text{pro}\;j\in\N}.\]
\end{definition}
Na první pohled si lze všimnout,~že pro $0<\delta_1<\delta_2$ je $\hausdorffdeltameasure{s}{\delta_1}(F)\geqslant\hausdorffdeltameasure{s}{\delta_2}(F)$. Jinými slovy, funkce $\delta\mapsto\hausdorffdeltameasure{s}{\delta}(M)$ je nerostoucí. Toto není nikterak těžké si rozmyslet,~neboť pro $\delta_1<\delta_2$ existuje $\delta_1$-pokrytí $\mathcal{F}_1$,~takové,~že je podpokrytím $\delta_2$-pokrytí $\mathcal{F}_2$ množiny $F$,~tedy $\mathcal{F}_1\subseteq\mathcal{F}_2$. To znamená,~že
\begin{align*}
    \hausdorffdeltameasure{s}{\delta_1}(F)&=\inf\set{\sum_{U\in\mathcal{F}_1}(\diam{U})^s\;\middle|\;\text{$\mathcal{F}_1$ je $\delta_1$-pokrytí $\mathcal{F}$}}\\
    &\geqslant\inf\set{\sum_{U\in\mathcal{F}_2}(\diam{U})^s\;\middle|\;\text{$\mathcal{F}_2$ je $\delta_2$-pokrytí $\mathcal{F}$}}=\hausdorffdeltameasure{s}{\delta_2}(F).
\end{align*}
Zároveň je z~definice \ref{def:hd-mira-delta} zjevné,~že $\mathcal{H}_\delta^s(F)\geqslant 0$.
\begin{definition}[Hausdorffova míra]\label{def:hausdorffova-mira}
    Nechť $F\subseteq\R^n$. Pak pro množinu $F$ definujeme \emph{$s$-dimenzionální Hausdorffovu míru}\index{míra!Hausdorffova} jako
    \[\hausdorffmeasure{s}(F)=\lim_{\delta\to 0}\hausdorffdeltameasure{s}{\delta}(F).\]
\end{definition}
Z přechodzího je zjevné,~že limita v~definici \ref{def:hausdorffova-mira} vždy existuje.

Bude dobré se přesvědčit,~že je Hausdorffova míra mírou ve smyslu definice \ref{def:prostor-s-mirou}. Začneme však otázkou. \emph{Na jaké množině je potřeba Hausdorffovu míru $\hausdorffmeasure{s}$ uvažovat?} Odpověď nám poskytují tzv. \emph{borelovské množiny}\index{množina!borelovská},~které jsou pojmenovány po francouzském matematikovi \name{Émile Borelovi} (1871--1956).
\begin{figure}[h]
    \centering
    \includegraphics[width=0.4\textwidth]{Emile-Borel.jpeg}
    \caption[Émile Borel,~1871--1956]{Émile Borel\footnote{Převzato z~\cite{OConnorBorel2025}},~1871--1956}
\end{figure}
Borelovské množiny hrají podstatnou roli v~tzv. \emph{Deskriptivní teorii množin}. Nebudeme si zde vykládat všechny souvislosti,~vystačíme si se základem. Takto nazýváme všechny množiny,~které lze získat iteracemi operacemi spočetného sjednocení, průniku a doplňku otevřených množin z~$X$. Označme systém takových množin jako $\mathcal{G}$. Na tomto základě pak definujeme tzv. \emph{$\sigma$-algebru borelovských množin na $X$}:
\[\borelsigmaalgebra(X)=\bigcap_{\substack{\mathcal{F}\supseteq\mathcal{G}\\\text{$\mathcal{F}$ je $\sigma$-algebra}}}\mathcal{F}.\]
Jinými slovy,~$\borelsigmaalgebra(X)$ je nejmenší $\sigma$-algebra generovaná\footnote{Obecně $\sigma$-algebra $\mathcal{A}$ je generovaná množinou $X$,~když
\[\mathcal{A}=\bigcap_{\substack{\mathcal{F}\supseteq X\\\text{$\mathcal{F}$ je $\sigma$-algebra}}}\mathcal{F}.\]
Tento fakt se někdy značí $\mathcal{A}=\sigma(X)$.} všemi otevřenými množinami z~$X$.

Nás speciálně bude zajímat $\sigma$-algebra $\borelsigmaalgebra(\R^n)$. Nejdříve si však dokážeme dvě pomocná lemmata.
\begin{lemma}[$\sigma$-subaditivita Hausdorffovy míry]\label{lem:Hausdorffova-mira-subaditivita}
    Nechť jsou dány množiny $A_1,A_2,\ldots$,~kde $A_i\subseteq X$ pro každé $i\in\N$. Pak pro každé $s\geqslant 0$ platí
    \[\hausdorffmeasure{s}\left(\bigcup_{i=1}^\infty A_i\right)\leqslant\sum_{i=1}^{\infty}\hausdorffmeasure{s}(A_i).\]
\end{lemma}
\begin{proof}
    Nechť $s\geqslant 0$ a dále budiž dáno $\varepsilon>0$. Pro každé $i\in\N$ a $\delta>0$ mějme pokrytí
    \[\mathcal{F}_i=\set{F_{i,1},F_{i,2},\dots}\]
    množiny $A_i$,~takové,~že platí
    \[\sum_{j=1}^{\infty}(\diam{F_{i,j}})^s\leqslant\hausdorffdeltameasure{s}{\delta}(A_i)+\dfrac{\varepsilon}{2^i}.\]
    Systém $\bigcup_{i=1}^\infty\mathcal{F}_i$ tedy tvoří $\delta$-pokrytí množiny $A=\bigcup_{i=1}^\infty A_i$. Celkově
    \begin{align*}
        \hausdorffdeltameasure{s}{\delta}\left(\bigcup_{i=1}^\infty A_i\right)&\leqslant\sum_{i,j\in\N}(\diam{F_{i,j}})^s=\sum_{i=1}^{\infty}\sum_{j=1}^{\infty}(\diam{F_{i,j}})^s\leqslant\sum_{i=1}^{\infty}\left(\hausdorffdeltameasure{s}{\delta}(A_i)+\dfrac{\varepsilon}{2^i}\right)\\
        &=\sum_{i=1}^{\infty}\hausdorffdeltameasure{s}{\delta}(A_i)+\varepsilon.
    \end{align*}
    Limitním přechodem $\delta\to 0$ a aplikací Leviho věty\footnote{\emph{Leviho věta o~záměně pořadí limity a Lebesgueova integrálu} říká,~že je-li posloupnost nezáporných měřitelných funkcí $\set{f_n}_{n=1}^\infty$ neklesající,~tj.
    \[f_1\leqslant f_2\leqslant\dots\]
    na prostoru $(X,\mathcal{A},\mu)$ a zároveň $\lim_{n\to\infty}f_n(x)=f(x)$ pro každé $x\in X$,~pak
    \[\lim_{n\to\infty}\int_X f_n\dx[\mu]=\int_X \lim_{n\to\infty}f_n\dx[\mu].\]
    Zde je speciálně $\mu$ aritmetická míra, $X=\N$,~a $f_n(i)$ lze volit např. $\hausdorffdeltameasure{s}{1/n}(A_i)$. Z~Heineho věty víme,~že
    \[\lim_{n\to\infty}\hausdorffdeltameasure{s}{1/n}(A_i)=\lim_{\delta\to 0}\hausdorffdeltameasure{s}{\delta}(A_i)=\hausdorffmeasure{s}(A_i).\]
    }
    dostáváme
    \begin{align*}
        \hausdorffmeasure{s}\left(\bigcup_{i=1}^\infty A_i\right)&=\lim_{\delta\to 0}\hausdorffdeltameasure{s}{\delta}\left(\bigcup_{i=1}^\infty A_i\right)\leqslant\lim_{\delta\to 0}\sum_{i=1}^{\infty}\hausdorffdeltameasure{s}{\delta}(A_i)+\varepsilon=\sum_{i=1}^{\infty}\lim_{\delta\to 0}\hausdorffdeltameasure{s}{\delta}(A_i)+\varepsilon\\
        &=\sum_{i=1}^{\infty}\hausdorffmeasure{s}(A_i)+\varepsilon.
    \end{align*} 
\end{proof}
\begin{lemma}\label{lem:hausdorffova-mira-sigma-aditivita-kladna-vzdalenost}
    Nechť $(X,\varrho)$ je metrický prostor,~kde $X\subseteq\R^n$,~a $A,B\subseteq X$,~takové,~že pro jejich vzdálenost platí $\varrho(A,B)>0$. Pak pro každé $s\geqslant 0$ platí
    \[\hausdorffmeasure{s}(A\cup B)=\hausdorffmeasure{s}(A)+\hausdorffmeasure{s}(B).\]
\end{lemma}
\begin{proof}
    Nerovnost $\hausdorffmeasure{s}(A\cup B)\leqslant\hausdorffmeasure{s}(A)+\hausdorffmeasure{s}(B)$ je zřejmá ze $\sigma$-subaditivity Hausdorffovy míry (viz lemma \ref{lem:Hausdorffova-mira-subaditivita}).

    Bez újmy na obecnosti předpokládejme,~že $\hausdorffmeasure{s}(A\cup B)<\infty$. Mějme libovolné $\varepsilon>0$. Zvolme $\delta$-pokrytí $\mathcal{F}=\set{F_1,F_2,\ldots}$ množiny $A\cup B$,~takové,~že
    \[\sum_{i=1}^{\infty}(\diam{F_i})^s\leqslant\hausdorffdeltameasure{s}{\delta}(A\cup B)+\varepsilon.\]
    Opět bez újmy na obecnosti lze předpokládat,~že pro každé $i\in\N$ je $\diam{F_i}<\varrho(A,B)$. V~opačném případě bychom $F_i$ pokryli množnami s~menším průměrem. Z~toho pak plyne,~že každá z~množin $F_i$ má neprázdný průnik s~nejvýše jednou z~množin $A,B$,~tzn. z~pokrytí $\mathcal{F}$ lze vybrat dva disjunktní podsystémy $\mathcal{F}_A$ a~$\mathcal{F}_B$,~přičemž $\bigcup\mathcal{F}_A\supseteq A$ a $\bigcup\mathcal{F}_B\supseteq B$. Tedy celkově s~užitím předchozího lemmatu \ref{lem:Hausdorffova-mira-subaditivita} máme
    \begin{align*}
        \hausdorffdeltameasure{s}{\delta}(A)+\hausdorffdeltameasure{s}{\delta}(B)&\leqslant\hausdorffdeltameasure{s}{\delta}\left(\bigcup_{F\in\mathcal{F}_A}F\right)+\hausdorffdeltameasure{s}{\delta}\left(\bigcup_{F\in\mathcal{F}_B}F\right)\\
        &\leqslant\sum_{F\in\mathcal{F}_A}(\diam{F})^s+\sum_{F\in\mathcal{F}_B}(\diam{F})^s\\
        &\leqslant\sum_{i=1}^{\infty}(\diam{F_i})^s\leqslant\hausdorffdeltameasure{s}{\delta}(A\cup B)+\varepsilon.
    \end{align*}
    Pro $\delta\to 0$ dostáváme
    \[\hausdorffmeasure{s}(A)+\hausdorffmeasure{s}(B)\leqslant\hausdorffmeasure{s}(A\cup B)+\varepsilon\]
\end{proof}

\begin{definition}[Vnější míra]\label{def:vnejsi-mira}
    Nechť $(X,\mathcal{A})$ je měřitelný prostor. Zobrazení $\mapping{\mu^*}{\mathcal{A}}{\langle0,\infty\rangle}$ nazveme \emph{vnější mírou}\index{míra!vnější} na $\mathcal{A}$, pokud platí:
    \begin{enumerate}[label=(\alph*)]
        \item\label{def:vnejsi-mira-prazdna-mnozina} $\mu^*(\emptyset)=0$,
        \item\label{def:vnejsi-mira-monotonie} Pokud $A,B\in\mathcal{A}$ a $A\subseteq B$, pak $\mu^*(A)\leqslant\mu^*(B)$.
        \item\label{def:vnejsi-mira-sigma-subaditivita} Je-li $A_1,A_2,\ldots$ posloupnost množin, kde $A_i\in\mathcal{A}$ pro každé $i\in\N$, pak
        \[\mu^*\left(\bigcup_{i=1}^\infty A_i\right)\leqslant\sum_{i=1}^{\infty}\mu^*(A_i).\]
    \end{enumerate}
\end{definition}

Vnější míra představuje zobecnění toho, co jsme měli možnost vidět již v sekci \ref{sec:lebesgueova-mira} týkající Lebesgueovy míry\footnote{Jedná se slabší požadavek, tzn. každá míra je vnější mírou, opačné tvrzení však neplatí.}. Tu jsme definovali na základě tzv. \emph{vnější Lebesgueovy míry} (viz definice \ref{def:vnejsi-lebegueova-mira}), která sice sama o sobě míru nepředstavovala, nicméně při restrikci na "správný" systém množin jsme konstatovali, že se již jedná o míru. Lze se přesvědčit, že vnější Lebesgueova míra $\lebesgueoutermeasure{n}$ je vnější mírou ve smyslu definice \ref{def:vnejsi-mira} výše. Podobné pozorování lze učinit i pro Hausdorffovu míru $\hausdorffmeasure{s}$. Platnost podmínky \ref{def:vnejsi-mira-sigma-subaditivita} jsme dokázali v lemmatu \ref{lem:Hausdorffova-mira-subaditivita} a o platnosti \ref{def:vnejsi-mira-prazdna-mnozina} a \ref{def:vnejsi-mira-monotonie} se může čtenář velice snadno předvědčit. Tedy Hausdorffova míra na měřitelném prostoru $(X,\mathcal{A})$ je vnější mírou. Navíc pokud vnější míra $\mu^*$ splňuje závěr lemmatu \ref{lem:hausdorffova-mira-sigma-aditivita-kladna-vzdalenost}, pak ji nazýváme \emph{metrickou vnější mírou}\index{míra!metrická}\index{míra!metrická vnější}.

Carathéodoryho kritérium, které jsme si uváděli při zavádění \emph{lebesgueovské měřitelnosti}\index{měřitelnost!lebesgueovská} (viz definice \ref{def:lebesgueovska-meritelnost}). Ta jednoduše říkala, že rozdělením libovolné množiny $G$ pomocí pěvně zvolené množiny $A$ lze stanovit její míru jako součet měr dílčích částí, tzn. $G\cap A$ a $G\setminus A$. Tento koncept lze však rozšířit. Obecně jakákoliv množina je $\mu$-měřitelná, pokud splňuje Carathéodoryho kritérium.
\begin{definition}\label{def:meritelnost}
    Nechť $\mu$ je míra a $A\subseteq X$. Množina $A$ je $\mu$-měřitelná, pokud pro každé $G\subseteq X$ platí
    \[\mu(G)=\mu(G\cap A)+\mu(G\setminus A).\]
\end{definition}
Speciálně nyní ukážeme platnost následujícího tvrzení \ref{thm:hs-meritelnost-borel-mnozin}.
\begin{theorem}\label{thm:hs-meritelnost-borel-mnozin}
    Nechť $(X,\varrho)$ je metrický prostor. Pak každá množina $A\in\borelsigmaalgebra(X)$ je $\hausdorffmeasure{s}$-měřitelná pro každé $s\geqslant 0$.
\end{theorem}
\begin{proof}
    Není těžké si rozmyslet, že $\borelsigmaalgebra(X)$ vyjma otevřených množin obsahuje též všechny uzavřené\footnote{Plyne z uzavřenosti na doplněk.}. Volme tedy uzavřenou množinu $A\in\borelsigmaalgebra(X)$, $G\subseteq X$ a $s\geqslant 0$. Ze $\sigma$-subaditivity plyne nerovnost
    \[\hausdorffmeasure{s}(G)\leqslant\hausdorffmeasure{s}(G\cap A)+\hausdorffmeasure{s}(G\setminus A).\]
    Pro důkaz opačné nerovnosti definujeme posloupnost množin $P_0,P_1,P_2,\ldots$ následovně:
    \begin{align*}
        P_0&=\set{x\in G\mid\varrho(x,A)\geqslant 1},\\
        P_i&=\set{x\in G\;\middle|\;\dfrac{1}{i+1}\leqslant\varrho(x,A)\leqslant\dfrac{1}{i}}\;,\;i\geqslant 1.
    \end{align*}
    Pro libovolnou dvojici množin z podposloupnosti $P_0,P_2,P_4,\ldots$ platí, že jejich vzdálenosti jsou kladné. Z faktu, že $\hausdorffmeasure{s}$ je metrická (viz lemma \ref{lem:hausdorffova-mira-sigma-aditivita-kladna-vzdalenost}) a monotonie plyne
    \[\sum_{i=1}^{m}\hausdorffmeasure{s}(P_{2i})=\hausdorffmeasure{s}\left(\bigcup_{i=0}^m P_{2i}\right)\leqslant\hausdorffmeasure{s}(G)\]
    pro všechna $m\in\N$. Podobně pro liché členy $\sum_{i=0}^{m}\hausdorffmeasure{s}(P_{2i+1})\leqslant\hausdorffmeasure{s}(G)$. Tzn. řada $\sum_{i=0}^{\infty}\hausdorffmeasure{s}(P_i)$ je konvergentní. Zároveň platí
    \[\varrho\left(\bigcup_{i=0}^m P_i,G\cap A\right)>0,\]
    pro každé $m\in\N$, tedy lze psát
    \begin{align*}
        \hausdorffmeasure{s}(G\setminus A)&\leqslant\hausdorffmeasure{s}\left(\bigcup_{i=0}^m P_i\right)+\hausdorffmeasure{s}\left(\bigcup_{i=m+1}^\infty P_i\right)\\
        &\leqslant\hausdorffmeasure{s}(G)-\hausdorffmeasure{s}(G\cap A)+\sum_{i=m+1}^{\infty}\hausdorffmeasure{s}(P_i).
    \end{align*}
    Pro $m\to\infty$ dostáváme
    \[\hausdorffmeasure{s}(G\setminus A)\leqslant\hausdorffmeasure{s}(G)-\hausdorffmeasure{s}(G\cap A)\]
    z čehož již plyne požadovaná nerovnost.
\end{proof}
\begin{corollary}\label{cor:hausdorffova-mira-je-mira}
    Trojice $(X,\borelsigmaalgebra(X),\hausdorffmeasure{s})$,~kde $X$ je libovolná množina a $s\geqslant 0$,~tvoří prostor s~mírou.
\end{corollary}

Nyní již můžeme zobrazení $\hausdorffmeasure{s}$ nazývat mírou oprávněně. Pojďme se podívat na nějaké příklady.
\begin{example}
    Pro $s=0$ představuje zobrazení $\hausdorffmeasure{s}$ obyčejnou aritmetickou míru\index{míra!aritmetická},~tzn. pro konečnou množinu $A\subseteq\R^n$ je $\hausdorffmeasure{0}(A)=|A|$. Toto není těžké ukázat. Mějme množinu $A=\set{x_1,x_2,\ldots,x_n}$. Zvolíme-li
    \[\delta<\dfrac{1}{2}\cdot\min\set{\varrho_e(x_i,x_j)\mid 1\leqslant i,j\leqslant n},\]
    pak pro $\delta$-pokrytí $\mathcal{F}=\set{F_1,F_2,\ldots,F_n}$ takové,~že $x_i\in F_i$ pro každé $i$ máme
    \[\sum_{i=1}^{n}(\diam{F_i})^0=\sum_{i=1}^{n}1=n.\]
    Není těžké si rozmyslet,~že $n$ je nejmenší počet množin o~průměru nejvýše $\delta$,~takových,~aby pokrývaly množinu $A$. Zároveň pro libovolné $\varepsilon>0$ je potřeba nejvýše $n$-koulí o~poloměru $\varepsilon/2$ se středy v~$x_i$ pro pokrytí $A$. Tzn.~$\hausdorffmeasure{0}(A)=|A|=n$.
\end{example}

V rámci tohoto textu jsme se již zabývali jiným typem míry a to tzv. \emph{lebesgueovou mírou} (viz sekce \ref{sec:lebesgueova-mira}). Ta pro nás hrála důležitou roli v jednom možném pojetí \emph{box-counting dimenze} (viz sekce \ref{sec:box-counting-dimenze}). Lze ukázat, že pro množinu $F\subseteq\R^n$ je
\[\hausdorffmeasure{n}(F)=\dfrac{1}{v_n}\lebesguemeasure{n}(F),\]
kde $v_n$ je objem (míra) jednotkové koule v $\R^n$. Čtenář snad promine, že tento fakt zde ponecháme bez důkazu. \citep[str. 45]{Falconer2014}

\subsection{Stručně k vlastnostem Hausdorffovy míry}\label{subsec:vlastnosti-hausdorffovy-miry}

Na chvíli se ještě zastavíme u vlastností Hausdorffovy míry. Již jsme společně dokázali, že Hausdorffova míra je skutečně mírou, tzn. splňuje všechny základní vlastnosti, které jsme si představili ve větě \ref{thm:mira-vlastnosti} (viz sekce \ref{sec:prostory-s-mirou}). V tomto ohledu tedy netřeba již nic dalšího dokazovat. Nicméně podobně jako v případě \emph{box-counting dimenze} (viz podsekce \ref{subsec:vlastnosti-bc-dimenze}) se i zde podíváme, jak se Hausdorffova míra chová vůči \emph{lipschitzovským zobrazením}\index{zobrazení!lipschitzovské}.
\begin{theorem}\label{thm:hd-dimenze-lipschitzovske-zobrazeni}
    Nechť $F\subseteq\R^n$ v metrickém prostoru $(\R^n,\varrho)$ a zobrazení $\mapping{f}{F}{\R^n}$ je lipschitzovské\footnote{Tvrzení lze zformulovat obecněji pro tzv. \emph{hölderovská zobrazení}\index{zobrazení!hölderovské}, tedy zobrazení $f$ splnující
    \[\varrho(f(x),f(y))\leqslant K(\varrho(x,y))^\alpha.\]
    kde $\alpha>0$. Pak pro $F\subseteq\R^n$ platí
    \[\hausdorffmeasure{s/\alpha}(f(F))\leqslant K^{s/\alpha}\hausdorffmeasure{s}(F).\]
    My si však vystačíme se speciálním případem.} s konstantou $K>0$. Pak pro každé $s\geqslant 0$ platí
    \[\hausdorffmeasure{s}(f(F))\leqslant K^s\hausdorffmeasure{s}(F).\]
\end{theorem}
\begin{proof}
    Nechť $\mathcal{F}=\set{F_1,F_2,\ldots}$ je $\delta$-pokrytí $F$. Pak
    \[\diam(f(F\cap F_i))\leqslant K\diam(F\cap F_i)\leqslant K\diam{F_i},\]
    což znamená, že $\mathcal{G}=\set{F\cap F_1,F\cap F_2,\ldots}$ je $K\delta$-pokrytí $f(F)$. Z toho plyne, že
    \[\sum_{i=1}^{\infty}(\diam(f(F\cap F_i)))^s\leqslant K^s\sum_{i=1}^{\infty}(\diam{F_i})^s\]
    a tedy $\hausdorffdeltameasure{s}{K\delta}(f(F))\leqslant K^s\hausdorffdeltameasure{s}{\delta}(F)$. Pro $\delta\to 0$ máme požadovaný výsledek.
\end{proof}
(Převzato z \citep[str. 46]{Falconer2014}.)

Z toho speciálně plyne důsledek týkající se podobností.
\begin{corollary}\label{cor:hd-dimenze-podobnost}
    Nechť $F\subseteq\R^n$ v metrickém prostoru $(\R^n,\varrho)$ a zobrazení $\mapping{f}{F}{\R^n}$ je podobnost\index{podobnost}, tzn. existuje $K>0$ takové, že pro každé $x,y\in F$ platí
    \[\varrho(f(x),f(y))=K\varrho(x,y).\]
    Pak pro každé $s\geqslant 0$ platí
    \[\hausdorffmeasure{s}(f(F))=K^s\hausdorffmeasure{s}(F).\]
\end{corollary}
\begin{proof}
    K podobnosti $f$ existuje inverzní zobrazení $f^{-1}$ s koeficientem $L=1/K$. Z věty \ref{thm:hd-dimenze-lipschitzovske-zobrazeni} tedy plyne, že
    \[\hausdorffmeasure{s}(F)=\hausdorffmeasure{s}(f^{-1}(f(F)))\leqslant\dfrac{1}{K^s}\hausdorffmeasure{s}(f(F))\]
    nebo-li $\hausdorffmeasure{s}(f(F))\geqslant K^s\hausdorffmeasure{s}(F)$. Opačnou nerovnost získáme aplikací věty \ref{thm:hd-dimenze-lipschitzovske-zobrazeni} na zobrazení $f$.
\end{proof}

\subsection{Hausdorffova dimenze}\label{subsec:hausdorffova-dimenze}

Středobodem této sekce je tzv. \emph{Hausdorffova dimenze}\index{dimenze!Hausdorffova}. Na úvod si dokážeme jedno jednoduché tvrzení týkající se Hausdorffovy míry.
\begin{theorem}\label{thm:hodnoty-hausdorffovy-miry}
    Nechť $0\leqslant s<t<\infty$ a $F\subseteq X$. Pak platí:
    \begin{enumerate}[label=(\roman*)]
        \item\label{thm:hd-dimenze-konecna} $\hausdorffmeasure{s}(F)<\infty\implies\hausdorffmeasure{t}(F)=0$,
        \item\label{thm:hd-dimenze-nekonecno} $\hausdorffmeasure{t}(F)>0\implies\hausdorffmeasure{s}(F)=\infty$.
    \end{enumerate}
\end{theorem}
\begin{proof}
    Mějme $\delta$-pokrytí $\mathcal{F}=\set{F_1,F_2,\ldots}$,~takové,~že
    \[\sum_{i=1}^{\infty}(\diam{F_i})^s\leqslant\hausdorffdeltameasure{s}{\delta}(F)+\varepsilon\;,\;\varepsilon>0.\]
    Pak
    \[\hausdorffdeltameasure{t}{\delta}(F)\leqslant\sum_{i=1}^{\infty}(\diam{F_i})^t\leqslant\delta^{t-s}\sum_{i=1}^{\infty}(\diam{F_i})^s\leqslant\delta^{t-s}(\hausdorffdeltameasure{s}{\delta}(F)+\varepsilon).\]
    Tzn.~$\hausdorffdeltameasure{t}{\delta}(F)\leqslant\delta^{t-s}\hausdorffdeltameasure{s}{\delta}(F)$. Pro $\delta\to 0$ dostaneme body \ref{thm:hd-dimenze-konecna} a \ref{thm:hd-dimenze-nekonecno}.
\end{proof}
(Převzato z~\citep[str. 68]{Mattila1995}.)

Z věty \ref{thm:hodnoty-hausdorffovy-miry} lze vidět,~že Hausdorffova míra dává smysl jen pro určitou hodnotu $s$. Pro "příliš velké" $s$ bude hodnota vždy $0$,~naopak pro "moc malé" $s$ bude jeho hodnota rovna $\infty$ (viz obrázek \ref{fig:hausdorffova-dimenze-graf}).
\begin{figure}[h]
    \centering
    \includegraphics{ch02-hausdorffova-dimenze-graf.pdf}
    \caption{Graf funkce $f(s)=\hausdorffmeasure{s}(F)$,~kde $F\subseteq\R^n$.}
    \label{fig:hausdorffova-dimenze-graf}
\end{figure}
Této kritické hodnotě $s$ říkáme \emph{Hausdorffova dimenze}\index{dimenze!Hausdorffova}.
\begin{definition}[Hausdorffova dimenze]\label{def:hausdorffova-dimenze}
    Nechť $F\subseteq\R^n$. Hausdorffovou dimenzí\footnote{Též někdy nazývaná \emph{Hausdorffova-Bezikovičova dimenze}\index{dimenze!Hausdorffova-Bezikovičkova}. }\index{dimenze!Hausdorffova} množiny $F$ nazveme hodnotu
    \[\dimH{F}=\inf\set{s\geqslant 0\mid\hausdorffmeasure{s}(F)=0}=\sup\set{s\geqslant 0\mid\hausdorffmeasure{s}(F)=\infty}.\]
\end{definition}
Hodnota $\hausdorffmeasure{s}(F)$ pro $s=\dimH{F}$ může být různá,~tzn. může platit,~že $\hausdorffmeasure{s}(F)=\infty$,~$\hausdorffmeasure{s}(F)=0$ a nebo se může jednat o~konečné nenulové číslo,~tj. $0<\hausdorffmeasure{s}(F)<\infty$.

Nyní se podívejme na příklad výpočtu. Podobně, jako v případě box-counting dimenze, i zde budeme nezávisle určovat horní a dolní odhad.
\begin{example}[Sierpińského trojúhelník]\label{ex:sierpinskeho-trojuhelnik-hd-dimenze}
    V tomto případě se podíváme na dvě možnosti, jak dojít k výsledku. Celý Sierpińského trojúhelník si označme $S$ a obrazec po $k$-té iteraci si označíme $S_k$.
    \begin{itemize}
        \item V $k$-té iteraci, kde $k=0,1,2,\ldots$, vzniknou 3 nové trojúhelníky o obsahu $1/4$ obsahu původního trojúhelníka, tzn. jejich celkový počet je $t=3^k$. Uvažíme-li $\delta$-pokrytí
        \[\mathcal{K}=\set{K_\delta^1(x_1),K_\delta^2(x_2),\ldots,K_\delta^t(x_t),\emptyset,\emptyset,\ldots}\]
        kde $x_1,x_2,\ldots,x_t\in S_k$ a $\delta\leqslant 2^{-k}/2=2^{-k-1}$, pak
        \[\hausdorffdeltameasure{s}{2^{-k}}(S_k)\leqslant\sum_{i=1}^{3^k}(2^{-k})^s=3^k2^{-ks}=1.\]
        Pro $k\to\infty$ je $\hausdorffmeasure{s}(S)\leqslant 1$. Poslední rovnost nastává právě pro $s=\ln{3}/\ln{2}$.

        Nyní ukážeme, že $\hausdorffmeasure{s}(S)\geqslant 1/2$. Zvolme $\delta$-pokrytí $\mathcal{F}=\set{F_1,F_2,\ldots}$, takové, že
        \begin{equation}\label{eq:volba-delta-pokryti-F}
            2^{-k-1}\leqslant\diam{F_i}<2^{-k},
        \end{equation}
        kde $i\in\N$. Lze si rozmyslet, že každá z množin $F_i$ má neprázdný průnik s nejvýše dvěma dílčími trojúhelníky. Zvolíme-li $j\leqslant k$, pak každá z množin $F_i$ má průnik maximálně s $3^{j-k}$ trojúhelníky v $j$-té iteraci, resp.
        \[3^{j-k}=3^j2^{-ks}\leqslant2^j3^s(\diam{F_i})^s,\]
        jak plyne z volby pokrytí $\mathcal{F}$ v \eqref{eq:volba-delta-pokryti-F}.
    \end{itemize}
\end{example}