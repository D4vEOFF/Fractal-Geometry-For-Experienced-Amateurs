\section{Základní pojmy a značení}\label{sec:zakladni-pojmy-a-znaceni}

V tomto oddílu se v~krátkosti zaměříme na připomenutí některých pojmů a značení,~které budeme dále využívat. Související teorii týkající se mnoha záležitostí v~tomto případě vynecháme s~předpokladem,~že ji čtenář již zná. Pokud tomu však v~některých případech takto nebude,~lze tuto část textu považovat za výčet konceptů,~které pro zvládnutí nadcházející teorie budeme potřebovat. V tomto ohledu lze např. nahlédnout do knihy \cite{NetukaAnalyza2014}.

Výklad v této kapitole a dále v kapitole \ref{chapter:hausdorffuv-mp} se bude točit především okolo tzv. \emph{metrických prostorů} (viz definice \ref{def:metricky-prostor}).
\begin{definition}[Metrický prostor]\label{def:metricky-prostor}
    Metrickým prostorem nazýváme uspořádanou dvojici $(X,\varrho)$, kde $X\neq\emptyset$ a $\mapping{\varrho}{X\times X}{\R_0^+}$ je zobrazení splňující:
    \begin{enumerate}[label=(\alph*)]
        \item $\forall x,y\in X: \varrho(x,y)=0\iff x=y$,
        \item $\forall x,y\in X: x\neq y\implies \varrho(x,y)>0$,
        \item $\forall x,y\in X: \varrho(x,y)=\varrho(y,x)$,\rightnote{symetrie}
        \item $\forall x,y,z\in X: \varrho(x,z)\leqslant\varrho(x,y)+\varrho(y,z)$.\rightnote{trojúhelníková nerovnost}
    \end{enumerate}
\end{definition}

\subsection{Metrické pojmy}\label{subsec:metricke-pojmy}

Začneme některými základními pojmy sousvisejícími s metrickými prostory.
\begin{definition}[Otevřená/uzavřená koule]\label{def:koule-mp}
    Nechť $(X,\varrho)$ je metrický prostor. Pak definujeme
    \begin{itemize}
        \item \emph{otevřenou kouli} se středem v bodě $x\in X$ o poloměru $r\geqslant 0$
        \[B_r(x)=\set{y\in X\mid\varrho(y,x)<r},\]
        \item resp. \emph{uzavřenou kouli} se středem v bodě $x\in X$ o poloměru $r\geqslant 0$
        \[K_r(x)=\set{y\in X\mid\varrho(y,x)\leqslant r}.\]
    \end{itemize}
\end{definition}
Např. v $\R$ představuje otevřená koule $B_r(x)$ otevřený interval $(x-r,x+r)$ a $K_r(x)=\langle x-r,x+r\rangle$. V $\R^2$ se jedná o standardní kruh (s hranicí, nebo bez ní). Nebo v případě prostoru spojitých funkcí $X=\mathcal{C}(\langle a,b\rangle)$ se pro libovolnou funkci $f\in X$ jedná o pás šířky $2r$ sestrojený okolo grafu funkce $f$.

S otevřenými, resp. uzavřenými koulemi se pojí další terminologie.
\begin{definition}[Otevřená, uzavřená a omezená množina]\label{def:otevrena-uzavrena-omezena-mnozina}
     Množina $M$ v metrickém prostoru $(X,\varrho)$ se nazývá
     \begin{itemize}
        \item \emph{otevřená}, pokud pro každé $x\in M$ existuje $r>0$, takové, že $B_r(x)\subseteq M$.
        \item \emph{uzavřená}, pokud její doplněk $X\setminus M$ je otevřená množina.
        \item \emph{omezená}, pokud existuje $x\in X$ a $r>0$, takové, že $M\subseteq B_r(x)$.
     \end{itemize}
\end{definition}
Uzavřenost a otevřenost množiny závisí na volbě konkrétního metrického prostoru. Např. interval $(a,b)$ je v $\R$ otevřená množina, avšak v $X=(a,b)$ je to uzavřená množina, neboť její doplněk $\emptyset$ je otevřený. Není těžké dokázat, že $\R^n$ je
\[\diam{B_r(x)}=\diam{K_r(x)}=2r.\]
\begin{definition}[Průměr množiny]\label{def:prumer-mnoziny}
    Nechť $(X,\varrho)$ je metrický prostor. \emph{Průměr množiny $M\subseteq X$}, $M\neq\emptyset$ definujeme jako
    \[\diam{M}=\sup\set{\varrho(x,y)\mid x,y\in M}.\]
\end{definition}
Intuitivně $\diam{B_r(x)}=\diam{K_r(x)}=2r$, což není těžké dokázat.
\begin{definition}[Vzdálenost bodu od množiny, vzdálenost množin]\label{def:vzdalenost-bodu-od-mnoziny-vzdalenost-mnozin}
    Nechť $(X,\varrho)$ je metrický prostor.
    \begin{itemize}
        \item \emph{Vzdáleností bodu $x\in X$ od množiny $M\subseteq X$} rozumíme
        \[\varrho(x,M)=\inf\set{\varrho(x,y)\mid y\in X}.\]
        \item \emph{Vzdáleností množin $M,N\subseteq X$} rozumíme
        \[\varrho(M,N)=\inf\set{\varrho(x,y)\mid x\in M\;,\;y\in N}.\]
    \end{itemize}
\end{definition}

\subsection{Limity posloupností a funkcí}\label{subsec:limity-posl-mp}

S limitou posloupnosti a funkce jedné proměnné je čtenář nejspíše dobře seznámen. V kontextu metrických prostorů definujeme pojem limity následovně (viz definice \ref{def:limita-mp}).
\begin{definition}[Limita posloupnosti]\label{def:limita-mp}
    Mejme metrický prostor $(X,\varrho)$ a posloupnost $\set{x_n}_{n=1}^\infty$, kde $x_i\in X$ pro každé $i\in\N$. Pak posloupnost $\set{x_n}_{n=1}^\infty$ má limitu $x\in X$, píšeme
    \[\lim_{n\to\infty}x_n=x,\]
    nebo též $x_n\to x$, pokud
    \[\forall\varepsilon>0\;\exists n_0\in\N\;\forall n\geqslant n_0: \varrho(x_n,x)<\varepsilon.\]
\end{definition}
Limita posloupnosti je vždy určena jednoznačně (pokud existuje). Ve spojitosti s limitami pro nás bude dále relevantní i tzv. \emph{limes superior} a \emph{limes inferior}. Ty si však připomeneme mimo kontext metrických prostorů. Vystačíme si v reálných číslech.
\begin{definition}[Limes superior, limes inferior]\label{def:limsup-liminf-mp}
    Mějme posloupnost $\set{x_n}_{n=1}^\infty$, kde $x_n\in\R$ pro každé $n\in\N$. Pak definujeme:
    \begin{itemize}
        \item Limes superior
        \[\limsup_{n\to\infty}x_n=\lim_{n\to\infty}\sup\set{x_k\mid k\geqslant n}.\]
        \item Limes inferior
        \[\limsup_{n\to\infty}x_n=\lim_{n\to\infty}\inf\set{x_k\mid k\geqslant n}.\]
    \end{itemize}
\end{definition}
Jinak lze definovat limes superior, resp. limes inferior jako supremum, resp. infimum hromadných bodů posloupnosti. Speciálně platí, že posloupnost $\set{x_n}_{n=1}^\infty$ má limitu $L$, právě tehdy, když
\[\limsup_{n\to\infty}x_n=\liminf_{n\to\infty}x_n=L.\]

Od posloupností se přesuneme k limitám funkcí.
\begin{definition}[Limita funkce v bodě]\label{def:limita-fce-v-bode}
    Nechť $(X,\varrho_1),(Y,\varrho_2)$ a funkce $\mapping{f}{X}{Y}$. Řekneme, že $f$ má limitu $y\in Y$ v bodě $x_0\in X$, píšeme
    \[\lim_{x\to x_0}f(x)=y,\]
    pokud
    \[\forall \varepsilon >0\;\exists \delta >0\;\forall x\in X: x\in B_\delta(x_0)\setminus\set{x_0}\implies f(x)\in B_\varepsilon(y).\]
\end{definition}


\todo{Doplnit pojmy a značení podle dalšího textu
    \begin{itemize}
        \item Úplný MP
        \item Cauchyovská posloupnost
        \item Bodová a stejnoměrná konvergence
        \item vzdálenost bodu od množiny
        \item Otevřená/uzavřená koule
        \item Otevřená/uzavřená množina
        \item Kompaktní množina, věta o kompaktnosti v $\R^n$, věta o uzavřenosti kompaktní množiny.
        \item Kvádr a objem kvádru
        \item $\delta$-okolí množiny
        \item Pokrytí, zjemnění
        \item $\delta$-pokrytí
        \item $\delta$-mříž
        \item Vnitřek, hranice množiny
        \item (bi-)lipschitzovské zobrazení
    \end{itemize}
}