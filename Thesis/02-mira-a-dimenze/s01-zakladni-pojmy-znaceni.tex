\section{Základní pojmy a značení}\label{sec:zakladni-pojmy-a-znaceni}

V tomto oddílu se v~krátkosti zaměříme na připomenutí některých pojmů a značení,~které budeme dále využívat. Související teorii týkající se mnoha záležitostí v~tomto případě vynecháme s~předpokladem,~že ji čtenář již zná. Pokud tomu však v~některých případech takto nebude,~lze tuto část textu považovat za výčet konceptů,~které pro zvládnutí nadcházející teorie budeme potřebovat. V tomto ohledu lze např. nahlédnout do knihy \cite{NetukaAnalyza2014}.

Tuto část tedy vnímejte spíše jako referenční, než rigorózní výklad.

Výklad v této kapitole a dále v kapitole \ref{chapter:hausdorffuv-mp} se bude točit především okolo tzv. \emph{metrických prostorů} (viz definice \ref{def:metricky-prostor}).
\begin{definition}[Metrický prostor]\label{def:metricky-prostor}
    \emph{Metrickým prostorem}\index{metrický prostor} nazýváme uspořádanou dvojici $(X,\varrho)$, kde $X\neq\emptyset$ a $\mapping{\varrho}{X\times X}{\R_0^+}$ je zobrazení splňující:
    \begin{enumerate}[label=(\alph*)]
        \item $\forall x,y\in X: \varrho(x,y)=0\iff x=y$,
        \item $\forall x,y\in X: x\neq y\implies \varrho(x,y)>0$,
        \item $\forall x,y\in X: \varrho(x,y)=\varrho(y,x)$,\rightnote{symetrie}
        \item $\forall x,y,z\in X: \varrho(x,z)\leqslant\varrho(x,y)+\varrho(y,z)$.\rightnote{trojúhelníková nerovnost}
    \end{enumerate}
    Zobrazení $\varrho$ nazýváme \emph{metrika}\index{metrický prostor!metrika}.
\end{definition}

\subsection{Metrické pojmy}\label{subsec:metricke-pojmy}

Začneme některými základními pojmy sousvisejícími s metrickými prostory.
\begin{definition}[Otevřená/uzavřená koule]\label{def:koule-mp}
    Nechť $(X,\varrho)$ je metrický prostor. Pak definujeme
    \begin{itemize}
        \item \emph{otevřenou kouli} se středem v bodě $x\in X$ o poloměru $r\geqslant 0$
        \[B_r(x)=\set{y\in X\mid\varrho(y,x)<r},\]
        \item resp. \emph{uzavřenou kouli} se středem v bodě $x\in X$ o poloměru $r\geqslant 0$
        \[K_r(x)=\set{y\in X\mid\varrho(y,x)\leqslant r}.\]
    \end{itemize}
\end{definition}
Např. v $\R$ představuje otevřená koule $B_r(x)$ otevřený interval $(x-r,x+r)$ a $K_r(x)=\langle x-r,x+r\rangle$. V $\R^2$ se jedná o standardní kruh (s hranicí, nebo bez ní). Nebo v případě prostoru spojitých funkcí $X=\continuousfuncspace(\langle a,b\rangle)$ se pro libovolnou funkci $f\in X$ jedná o pás šířky $2r$ sestrojený okolo grafu funkce $f$.

S otevřenými, resp. uzavřenými koulemi se pojí další terminologie.
\begin{definition}[Otevřená, uzavřená a omezená množina]\label{def:otevrena-uzavrena-omezena-mnozina}
     Množina $M$ v metrickém prostoru $(X,\varrho)$ se nazývá
     \begin{itemize}
        \item \emph{otevřená}\index{množina!otevřená}, pokud pro každé $x\in M$ existuje $r>0$, takové, že $B_r(x)\subseteq M$.
        \item \emph{uzavřená}\index{množina!uzavřená}, pokud její doplněk $X\setminus M$ je otevřená množina.
        \item \emph{omezená}\index{množina!omezená}, pokud existuje $x\in X$ a $r>0$, takové, že $M\subseteq B_r(x)$.
     \end{itemize}
\end{definition}
Uzavřenost a otevřenost množiny závisí na volbě konkrétního metrického prostoru. Např. interval $(a,b)$ je v $\R$ otevřená množina, avšak v $X=(a,b)$ je to uzavřená množina, neboť její doplněk $\emptyset$ je otevřený.
\begin{definition}[Průměr množiny]\label{def:prumer-mnoziny}
    Nechť $(X,\varrho)$ je metrický prostor. \emph{Průměr množiny\index{průměr množiny} $M\subseteq X$}, $M\neq\emptyset$ definujeme jako
    \[\diam{M}=\sup\set{\varrho(x,y)\mid x,y\in M}.\]
\end{definition}
Není těžké dokázat, že $\R^n$ je
\[\diam{B_r(x)}=\diam{K_r(x)}=2r.\]
Obecně to však neplatí. Uvažme metrický prostor $(X,\varrho)$, kde je metrika $\varrho$
\[\varrho(x,y)=\begin{cases}
    0 & x=y,\\
    1 & x\neq y.
\end{cases}\]
Jedná se o takzvaný \emph{diskrétní metrický prostor}\index{metrický prostor!diskrétní} (o tom, že $\varrho$ je metrika se lze snadno přesvědčit). Pro $x\in X$ a $r>0$ platí, že
\[\diam{B_r(x)}=\begin{cases}
    0 & r\leqslant 1,\\
    1 & r>1.
\end{cases}\]
Tedy poloměr koule $B_r(x)$ může být větší než její průměr.

\begin{definition}[Vzdálenost bodu od množiny, vzdálenost množin]\label{def:vzdalenost-bodu-od-mnoziny-vzdalenost-mnozin}
    Nechť $(X,\varrho)$ je metrický prostor.
    \begin{itemize}
        \item \index{vzdálenost!bodu od množiny}\emph{Vzdáleností bodu $x\in X$ od množiny $M\subseteq X$} rozumíme
        \[\varrho(x,M)=\inf\set{\varrho(x,y)\mid y\in X}.\]
        \item \index{vzdálenost!množin}\emph{Vzdáleností množin $M,N\subseteq X$} rozumíme
        \[\varrho(M,N)=\inf\set{\varrho(x,y)\mid x\in M\;,\;y\in N}.\]
    \end{itemize}
\end{definition}

\begin{definition}[Vnitřek, hranice a uzávěr množiny]\label{def:vnitrek-hranice-uzaver}
    Nechť $(X,\varrho)$ je metrický prostor a $M\subseteq X$.
    \begin{itemize}
        \item \emph{Vnitřek množiny $M$}\index{vnitřek množiny} je množina
        \[\interior{M}=\set{x\in X\mid \exists r>0: B_r(x)\subseteq M}.\]
        \item \emph{Hranice množiny $M$}\index{hranice množiny} je množina
        \[\boundary{M}=\set{x\in X\mid \forall r>0\;\exists y,z\in B_r(x): y\in X\land z\in X\setminus M}.\]
        \item \emph{Uzávěr množiny $M$}\index{uzávěr množiny} je množina
        \[\closure{M}=M\cup\boundary{M}\]
    \end{itemize}
\end{definition}
V textu též budeme hodně pracovat s pojmem $\delta$-okolí.
\begin{definition}[$\delta$-okolí]\label{def:delta-okoli}
    Nechť $(X,\varrho)$ je metrický prostor a $M\subseteq X$. Pak $\delta$-okolím množiny $M$ rozumíme množinu
    \[(M)_\delta=\set{y\in X\mid \exists x\in X: \varrho(x,y)<\delta}.\]
\end{definition}
Zde se zvlášť hodí zmínit, že $(M)_\delta$ je v prostoru $\R^n$, s nímž budeme často pracovat, otevřená množina.

Jako poslední zde zmíníme termíny, které zejména využijeme v sekci věnující se tzv. Lebesgueově míře a posléze box-counting dimenzi (viz sekce \ref{sec:lebesgueova-mira} a \ref{sec:box-counting-dimenze}).
\begin{definition}[Kvádr]\label{def:kvadr}
    Kvádrem\index{kvádr} v $\R^n$ nazveme množinu
    \[I=\prod_{i=1}^{n}\langle a_i,b_i\rangle,\]
    kde $a_i,b_i\in\R$ pro každé $i\in\N$. Objem kvádru\index{objem kvádru} $I$ definujeme jako
    \[\vol_n(I)=\prod_{i=1}^{n}(b_i-a_i).\]
\end{definition}
\begin{definition}[$\delta$-mříž]\label{def:delta-mriz}
    \emph{$\delta$-mříž}\index{$\delta$-mříž} v $\R^n$ názýváme množinu
    \[\mathcal{Q}_\delta=\set{\prod_{i=1}^{n}\langle m_j\delta,(m_j+1)\delta\rangle\;\middle|\;m_1,\ldots,m_n\in\Z}.\]
\end{definition}
Definice \ref{def:delta-mriz} vlastně říká, že $\mathcal{Q}_\delta$ představuje rozdělení $\R^n$ na krychle o straně délky $\delta$ dotýkajících se pouze na hranici.

\subsection{Limity posloupností}\label{subsec:limity-posl-mp}

S limitou posloupnosti a funkce jedné proměnné je čtenář nejspíše dobře seznámen. V kontextu metrických prostorů definujeme pojem limity následovně (viz definice \ref{def:limita-mp}).
\begin{definition}[Limita posloupnosti]\label{def:limita-mp}
    Mejme metrický prostor $(X,\varrho)$ a posloupnost $\set{x_n}_{n=1}^\infty$, kde $x_i\in X$ pro každé $i\in\N$. Pak posloupnost\index{limita!posloupnosti} $\set{x_n}_{n=1}^\infty$ má limitu $x\in X$, píšeme
    \[\lim_{n\to\infty}x_n=x,\]
    nebo též $x_n\to x$, pokud
    \[\forall\varepsilon>0\;\exists n_0\in\N\;\forall n\geqslant n_0: \varrho(x_n,x)<\varepsilon.\]
\end{definition}
Limita posloupnosti je vždy určena jednoznačně (pokud existuje). Ve spojitosti s limitami pro nás bude dále relevantní i tzv. \emph{limes superior} a \emph{limes inferior}. Ty si však připomeneme mimo kontext metrických prostorů. Vystačíme si v reálných číslech.
\begin{definition}[Limes superior, limes inferior]\label{def:limsup-liminf-mp}
    Mějme posloupnost $\set{x_n}_{n=1}^\infty$, kde $x_n\in\R$ pro každé $n\in\N$. Pak definujeme:
    \begin{itemize}
        \item Limes superior\index{limes superior}
        \[\limsup_{n\to\infty}x_n=\lim_{n\to\infty}\sup\set{x_k\mid k\geqslant n}.\]
        \item Limes inferior\index{limes inferior}
        \[\limsup_{n\to\infty}x_n=\lim_{n\to\infty}\inf\set{x_k\mid k\geqslant n}.\]
    \end{itemize}
\end{definition}
Jinak lze definovat limes superior, resp. limes inferior jako supremum, resp. infimum hromadných bodů posloupnosti. Speciálně platí, že posloupnost $\set{x_n}_{n=1}^\infty$ má limitu $L$, právě tehdy, když
\[\limsup_{n\to\infty}x_n=\liminf_{n\to\infty}x_n=L.\]

Čtenář již nejspíše slyšel i o tzv. \emph{Bolzanově-Cauchyově podmínce}\index{Bolzanova-Cauchyova podmínka}. Posloupnosti ji splňující názýváme tzv. \emph{cauchyovské}.
\begin{definition}[Cauchyovská posloupnost]\label{def:cauchyovska-posloupnost}
    Nechť $(X,\varrho)$ je metrický prostor. Posloupnost $\set{x_n}_{n=1}^\infty$, kde $x_i\in X$ pro každé $i\in N$, nazveme \emph{cauchyovskou}, pokud splňuje:
    \[\forall\varepsilon>0\;\exists n_0\in\N\;\forall n,m\geqslant n_0:\varrho(x_n,x_m)<\varepsilon.\]
\end{definition}
Připomeňme rovněž, že Bolzanova-Cauchyova podmínka je nutná, nikoliv však postačující pro konvergenci posloupnosti v $X$. Uvažme např. prostor $(\Q,|\cdot|)$ se stardardní eukleidovskou metrikou. Posloupnost $\set{x_n}_{n=1}^\infty$ je definujeme následovně:
\[x_0>0\;\text{a}\;x_n=\dfrac{1}{2}\left(x_{n-1}+\dfrac{2}{x_{n-1}}\right)\;,\;n\in\N.\]
Všechny její členy jsou v $\Q$ a lze též ověřit, že je cauchyovská. Nicméně pro její limitu platí
\[\lim_{n\to\infty}x_n=\sqrt{2}\notin\Q.\]

V rámci metrických prostorů se tedy dává smysl mnohdy omezovat pouze na takové, kde je Bolzanova-Cauchyova podmínka nutná i postačující pro konvergenci. Těž říkáme úplné.
\begin{definition}[Úplný metrický prostor]\label{def:uplny-mp}
    Metrický prostor $(X,\varrho)$ se nazývá \emph{úplný}\index{metrický prostor!úplný}, pokud každá cauchyovská posloupnost v $X$ má limitu v $X$.
\end{definition}

\subsection{Limity funkcí}\label{subsec:limity-fci}

Od posloupností se přesuneme na chvíli k limitám funkcí.
\begin{definition}[Limita funkce v bodě]\label{def:limita-fce-v-bode}
    Nechť $(X,\varrho_1),(Y,\varrho_2)$ a funkce $\mapping{f}{X}{Y}$. Řekneme, že $f$ má limitu\index{limita!funkce v bodě} $y\in Y$ v bodě $x_0\in X$, píšeme
    \[\lim_{x\to x_0}f(x)=y,\]
    pokud
    \[\forall \varepsilon >0\;\exists \delta >0\;\forall x\in X: x\in B_\delta(x_0)\setminus\set{x_0}\implies f(x)\in B_\varepsilon(y).\]
\end{definition}
Podobně i zde lze definovat limes superior a limes inferior. Omezíme se opět jen na funkce $\mapping{f}{M}{\R}$.
\begin{definition}[Limes superior a limes inferior pro funkce]\label{def:limsup-liminf-funkce}
    Nechť $(X,\varrho)$ je metrický prostor a funkce $\mapping{f}{M}{\R}$, kde $M\subseteq X$. Pak definujeme
    \begin{itemize}
        \item Limes superior v bodě $x_0$:
        \[\limsup _{x\to x_0}f(x)=\lim _{\varepsilon \to 0}\left(\sup\set{f(x):x\in M\cap B_\varepsilon(x_0)\setminus \{x_0\}}\right)\]
        \item Limes inferior v bodě $x_0$:
        \[\limsup _{x\to x_0}f(x)=\lim _{\varepsilon \to 0}\left(\inf\set{f(x):x\in M\cap B_\varepsilon(x_0)\setminus \{x_0\}}\right)\]
    \end{itemize}
\end{definition}

\subsection{Bodová a stejnoměrná konvergence}\label{subsec:bodova-stejnomerna-konvergence}

V rámci posloupností se v matematické analýze často pracuje s posloupnostmi funkcí. K jejich limitám se váže dvojice důležitých termínů: \emph{bodová} a \emph{stejnoměrná konvergence}.
\begin{definition}[Bodová a stejnoměrná konvergence]\label{def:bodova-stejnomerna-konvergence}
    Nechť $\set{f_n}_{n=1}^\infty$ je posloupnost funkcí z metrického prostoru $(\continuousfuncspace(\langle a,b\rangle),\varrho)$, kde $\varrho$ je tzv. \emph{suprémová metrika}, tj. pro každou $f,g\in \continuousfuncspace(\langle a,b\rangle)$
    \[\varrho(f,g)=\sup_{x\in\langle a,b\rangle}|f_n(x)-f(x)|,\]
    přičemž $\set{f_n}_{n=1}^\infty$ konverguje k funkci $f\in \continuousfuncspace(\langle a,b\rangle)$.
    Pak říkáme, že posloupnost $\set{f_n}_{n=1}^\infty$ konverguje k $f$
    \begin{itemize}
        \item \emph{bodově}\index{konvergence!bodová}, pokud $\lim_{n\to\infty}f_n(x)=f(x)$.
        \item \emph{stejnoměrně}\index{konvergence!stejnoměrná}, pokud
        \[\forall\varepsilon>0\;\exists n_0\in\N\;\forall n\geqslant n_0\;\forall x\in\langle a,b\rangle: \varrho(f_n,f)<\varepsilon.\]
        Píšeme $f_n\rightrightarrows f$.
    \end{itemize}
\end{definition}
Stejnoměrná konvergence implikuje konvergenci bodovou, opačné tvezení však neplatí. Např. posloupnost funkcí $f_n(x)=x^n$, kde $x\in\langle 0,1\rangle$, konverguje k funkci 
\[f(x)=\begin{cases}
    0 & x < 1\\
    1 & x = 1
\end{cases}\]
pouze bodově, nikoliv stejnoměrně, protože
\[\sup_{x\in\langle 0,1\rangle}|f_n(x)-f(x)|=\sup_{x\in\langle 0,1\rangle}|x^n-0|=1.\]

\subsection{Topologické pojmy}\label{subsec:topologicke-pojmy}

Poměrně důležitým pojmem v teorii metrických prostorů jsou tzv. kompaktní množiny. S tím se pojí následující termíny (viz definice ).
\begin{definition}[Pokrytí, $\delta$-pokrytí a zjemnění]\label{def:delta-pokryti-zjemneni}
    Nechť je dán metrický prostor $(X,\varrho)$, $M\subseteq X$ a systém množin $\mathcal{U}=\set{U_1,U_2,\ldots}\subseteq\powset{X}$.
    \begin{itemize}
        \item $\mathcal{U}$ tvoří tzv. \emph{pokrytí}\index{pokrytí} množiny $M$, pokud
        \[M\subseteq\bigcup_{i=1}^\infty U_i.\]
        Navíc o pokrytí $\mathcal{U}$ říkáme, že je \emph{otevřené}\index{pokrytí!otevřené pokrytí}, pokud pro každé $i\in\N$ je množina $U_i$ otevřená.
        \item Pokud existuje $\delta>0$, takové, že pro každé $i\in\N$ platí $\diam{U_i}<\delta$, pak $\mathcal{U}$ nazýváme \emph{$\delta$-pokrytím}\index{$\delta$-pokrytí}.
        \item Platí-li pro systém $\mathcal{G}=\set{G_1,G_2,\ldots}\subseteq\powset{X}$, že
        \[M\subseteq\bigcup_{i=1}^\infty G_i\subseteq\bigcup_{i=1}^\infty U_i,\]
        pak $\mathcal{G}$ je tzv. \emph{podpokrytí}\index{podpokrytí} pokrytí $\mathcal{U}$.
        \item Pokud pro každou $G\in\mathcal{G}$ existuje množina $U\in\mathcal{U}$, taková, že $G\subseteq U$, pak $\mathcal{G}$ nazýváme \emph{zjemněním}\index{zjemnění} pokrytí $\mathcal{U}$. 
    \end{itemize}
\end{definition}
\begin{definition}[Kompaktní množina]\label{def:kompaktni-mnozina}
    Nechť $(X,\varrho)$ je metrický prostor. Množinu $M\subseteq X$ nazveme \emph{kompaktní}\index{množina!kompaktní}, jestliže pro každé otevřené pokrytí $\mathcal{U}=\set{U_1,U_2,\ldots}$ existují indexy $n_1,\ldots,n_k\in\N$, takové, že
    \[M\subseteq\bigcup_{i=1}^k U_{n_i}.\]
\end{definition}
Volněji řečeno, pro kompaktní množinu lze z jejího libovolného otevřeného pokrytí vybrat konečné podpokrytí. Dokazovat kompaktnost množiny z definice může být mnohdy nepraktické, nicméně vzhledem k tomu, že budeme často pracovat s prostorem $\R^n$, záležitost se nám v tomto ohledu značně zjednodušuje.
\begin{theorem}[Heineho-Borelova]\label{thm:heine-borel}
    Nechť $(\R^n,\varrho)$ je metrický prostor a $M\subseteq\R^n$. Následující výroky jsou ekvivalentní:
    \begin{enumerate}[label=(\roman*)]
        \item Množina $M$ je kompaktní.
        \item Množina $M$ je uzavřená a omezená.
    \end{enumerate}
\end{theorem}
Důkaz věty lze nálézt např. v \citep[str. 166]{NetukaAnalyza2014}. Pro obecný metrický prostor toto tvrzení již neplatí. Např. v diskrétním metrickém prostoru je každá množina uzavřená i omezená (stačí zvolit $r>1$, tzn. $B_r(x)=X$). Nicméně uvážíme-li prostor $X=(0,1)$ s dikrétní metrikou $\varrho$, pak z otevřeného pokrytí
\[(0,1)\subseteq\bigcup_{n=1}^\infty\left\langle\dfrac{1}{n+1},\dfrac{1}{n}\right)\]
nelze vybrat žádné konečné, navzdory faktu, že interval $(0,1)$ je v $X$ uzavřený a omezený.

\todo{Jak je to s předpokladem pro metriku $\varrho$?}

\subsection{Lipschitzovská zobrazení}\label{subsec:lipschitzovska-zobrazeni}

\begin{definition}[Lipschitzovské zobrazení]\label{def:bilipschitzovske-zobrazeni}
    Nechť $(X,\varrho_1),(Y,\varrho_2)$ jsou metrické prostory. Pak zobrazení $\mapping{f}{X}{Y}$ nazveme 
    \begin{itemize}
        \item \emph{lipschitzovské}\index{zobrazení!lipschitzovské}, pokud existuje konstanta $K>0$ taková, že pro každé $x,y\in X$ platí
        \[\varrho_2(f(x),f(y))\leqslant K\varrho_1(x,y).\]
        Navíc pokud lze volit $K<1$, pak $f$ nazýváme \emph{kontrakcí}\index{kontrakce}
        \item \emph{bilipschitzovské}\index{zobrazení!bilipschitzovské}, pokud existují konstanty $K_1,K_2>0$ takové, že pro každé $x,y\in X$ platí
        \[K_1\varrho_1(x,y)\leqslant\varrho_2(f(x),f(y))\leqslant K_2\varrho_1(x,y).\]
    \end{itemize}
\end{definition}
Je celkem zjevné, že lipschitzovská zobrazení jsou vždy spojitá. Pro každé $\varepsilon>0$ stačí zvolit kouli o poloměru $\varepsilon/K$, tedy
\[\varrho_2(f(x_0),f(x))\leqslant K\varrho_1(x_0,x)\leqslant K\cdot\dfrac{\varepsilon}{K}=\varepsilon\]
implikující spojitost $f$ v bodě $x_0$.