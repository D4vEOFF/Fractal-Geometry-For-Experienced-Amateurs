\section{Základní pojmy a~značení}\label{sec:zakladni-pojmy-a-znaceni}

V tomto oddílu se v~krátkosti zaměříme na připomenutí některých pojmů a~značení,~které budeme dále využívat. Související teorii týkající se mnoha záležitostí v~tomto případě vynecháme s~předpokladem,~že ji čtenář již zná. Pokud tomu však v~některých případech takto nebude,~lze tuto část textu považovat za výčet konceptů,~které pro zvládnutí nadcházející teorie budeme potřebovat. V~tomto ohledu lze např. nahlédnout do knihy \cite{NetukaAnalyza2014}.

Tuto část tedy vnímejte spíše jako referenční,~než rigorózní výklad.

Výklad v~této kapitole a~dále v~kapitole~\ref{chapter:hausdorffuv-mp} se bude točit především okolo tzv. \emph{metrických prostorů} (viz definice~\ref{def:metricky-prostor}).
\begin{definition}[Metrický prostor]\label{def:metricky-prostor}
    \emph{Metrickým prostorem}\index{metrický prostor}\index{prostor!metrický} nazýváme uspořádanou dvojici $(X,\varrho)$,~kde $X\neq\emptyset$ a~$\mapping{\varrho}{X\times X}{\langle0,\infty)}$ je zobrazení splňující:
    \begin{enumerate}[label=(\alph*)]
        \item $\forall x,y\in X: \varrho(x,y)=0\iff x=y$,
        \item $\forall x,y\in X: \varrho(x,y)=\varrho(y,x)$,\rightnote{symetrie}
        \item $\forall x,y,z\in X: \varrho(x,z)\leqslant\varrho(x,y)+\varrho(y,z)$.\rightnote{trojúhelníková nerovnost}
    \end{enumerate}
    Zobrazení $\varrho$ nazýváme \emph{metrika}\index{metrika}.
\end{definition}
Zobrazení $\varrho$ v definici \ref{def:metricky-prostor} pro nás představuje způsob měření vzdálenosti mezi prvky množiny $X$. 

\subsection{Metrické pojmy}\label{subsec:metricke-pojmy}

Začneme některými základními pojmy sousvisejícími s~metrickými prostory.
\begin{definition}[Otevřená/uzavřená koule]\label{def:koule-mp}
    Nechť $(X,\varrho)$ je metrický prostor. Pak definujeme
    \begin{itemize}
        \item \emph{otevřenou kouli} se středem v~bodě $x\in X$ o~poloměru $r>0$
        \[B_r(x)=\set{y\in X\mid\varrho(y,x)<r},\]
        \item resp. \emph{uzavřenou kouli} se středem v~bodě $x\in X$ o~poloměru $r>0$
        \[K_r(x)=\set{y\in X\mid\varrho(y,x)\leqslant r}.\]
    \end{itemize}
\end{definition}
Např. v~$\R$ představuje otevřená koule $B_r(x)$ otevřený interval $(x-r,x+r)$ a~$K_r(x)=\langle x-r,x+r\rangle$. V~$\R^2$ s eukleidovskou metrikou se jedná o~standardní kruh. Nebo jiným příkladem může být prostor všech spojitých funkcí na zadaném intervalu $\langle a,b\rangle$,~značíme $X=\continuousfuncspace(\langle a,b\rangle)$,~přičemž jako metriku lze zvolit např.
\[\varrho(f,g)=\sup_{x\in\langle a,b\rangle}|f(x)-g(x)|.\]
Pak pro libovolnou funkci $f\in X$ představuje $B_r(f)$ a $K_r(f)$ pás šířky $2r$ sestrojený okolo grafu funkce $f$.

S otevřenými,~resp. uzavřenými koulemi se pojí další terminologie.
\begin{definition}[Otevřená,~uzavřená a~omezená množina]\label{def:otevrena-uzavrena-omezena-mnozina}
     Množina $M$ v~metrickém prostoru $(X,\varrho)$ se nazývá
     \begin{itemize}
        \item \emph{otevřená}\index{množina!otevřená},~pokud pro každé $x\in M$ existuje $r>0$,~takové,~že $B_r(x)\subseteq M$.
        \item \emph{uzavřená}\index{množina!uzavřená},~pokud její doplněk $X\setminus M$ je otevřená množina.
        \item \emph{omezená}\index{množina!omezená},~pokud existuje $x\in X$ a~$r>0$,~takové,~že $M\subseteq B_r(x)$.
     \end{itemize}
\end{definition}
Uzavřenost a~otevřenost množiny závisí na volbě konkrétního metrického prostoru. Např. interval $(a,b)$ je v~$\R$ otevřená množina,~avšak v~$X=(a,b)$ je to otevřená i~uzavřená množina,~neboť její doplněk $\emptyset$ je otevřený.
\begin{definition}[Průměr množiny]\label{def:prumer-mnoziny}
    Nechť $(X,\varrho)$ je metrický prostor. \emph{Průměr množiny\index{průměr množiny} $M\subseteq X$},~$M\neq\emptyset$ definujeme jako
    \[\diam{M}=\sup\set{\varrho(x,y)\mid x,y\in M}.\]
\end{definition}
Není těžké dokázat,~že v~$\R^n$ je
\[\diam{B_r(x)}=\diam{K_r(x)}=2r.\]
Obecně to však neplatí. Uvažme metrický prostor $(X,\varrho)$,~kde je metrika $\varrho$
\[\varrho(x,y)=\begin{cases}
    0 & x=y,\\
    1 & x\neq y.
\end{cases}\]
Jedná se o~takzvaný \emph{diskrétní metrický prostor}\index{metrický prostor!diskrétní}\index{diskrétní metrický prostor} (o tom,~že $\varrho$ je metrika se lze snadno přesvědčit). Pro $x\in X$ a~$r>0$ platí,~že
\[\diam{B_r(x)}=\begin{cases}
    0 & r\leqslant 1,\\
    1 & r>1.
\end{cases}\]
Koule $B_r(x)$ v diskrétním metrickém prostoru totiž představuje buď singleton $\set{x}$ pro $r\leqslant 1$, nebo celý prostor $X$. Tedy poloměr koule $B_r(x)$ může být větší než její průměr.

\begin{definition}[Vzdálenost bodu od množiny,~vzdálenost množin]\label{def:vzdalenost-bodu-od-mnoziny-vzdalenost-mnozin}
    Nechť $(X,\varrho)$ je metrický prostor.
    \begin{itemize}
        \item \index{vzdálenost!bodu od množiny}\emph{Vzdáleností bodu $x\in X$ od množiny $M\subseteq X$} rozumíme
        \[\varrho(x,M)=\inf\set{\varrho(x,y)\mid y\in M}.\]
        \item \index{vzdálenost!množin}\emph{Vzdáleností množin $M,N\subseteq X$} rozumíme
        \[\varrho(M,N)=\inf\set{\varrho(x,y)\mid x\in M,\;y\in N}.\]
    \end{itemize}
\end{definition}

\begin{definition}[Vnitřek,~hranice a~uzávěr množiny]\label{def:vnitrek-hranice-uzaver}
    Nechť $(X,\varrho)$ je metrický prostor a~$M\subseteq X$.
    \begin{itemize}
        \item \emph{Vnitřek množiny $M$}\index{vnitřek množiny} je množina
        \[\interior{M}=\set{x\in X\mid \exists r>0: B_r(x)\subseteq M}.\]
        \item \emph{Hranice množiny $M$}\index{hranice množiny} je množina
        \[\boundary{M}=\set{x\in X\mid \forall r>0\;\exists y,z\in B_r(x): y\in X\land z\in X\setminus M}.\]
        \item \emph{Uzávěr množiny $M$}\index{uzávěr množiny} je množina
        \[\closure{M}=M\cup\boundary{M}\]
    \end{itemize}
\end{definition}
V textu též budeme hodně pracovat s~pojmem $\delta$-okolí.
\begin{definition}[$\delta$-okolí]\label{def:delta-okoli}
    Nechť $(X,\varrho)$ je metrický prostor a~$M\subseteq X$. Pak $\delta$-okolím množiny $M$ rozumíme množinu
    \[(M)_\delta=\set{x\in X\mid\varrho(x,M)<\delta}.\]
\end{definition}
Zde se zvlášť hodí zmínit,~že $(M)_\delta$ je v~prostoru $\R^n$,~s nímž budeme často pracovat,~otevřená množina, je-li $M$ neprázdná.

Jako poslední zde zmíníme termíny,~které zejména využijeme v~sekci věnující se Lebesgueově míře a~posléze box-counting dimenzi (viz sekce~\ref{sec:lebesgueova-mira} a~\ref{sec:box-counting-dimenze}).
\begin{definition}[Kvádr]\label{def:kvadr}
    Kvádrem\index{kvádr} v~$\R^n$ nazveme množinu
    \[I=\prod_{i=1}^{n}\langle a_i,b_i\rangle,\]
    kde $a_i,b_i\in\R$ a $a_i\leqslant b_i$ pro každé $i\in\N$. Objem kvádru\index{objem kvádru} $I$ definujeme jako
    \[\vol_n(I)=\prod_{i=1}^{n}(b_i-a_i).\]
\end{definition}
\begin{definition}[$\delta$-mříž]\label{def:delta-mriz}
    \emph{$\delta$-mříž}\index{$\delta$-mříž} v~$\R^n$ názýváme množinu
    \[\mathcal{Q}_\delta=\set{\prod_{i=1}^{n}\langle m_j\delta,(m_j+1)\delta\rangle\;\middle|\;m_1,\ldots,m_n\in\Z}.\]
\end{definition}
Množinu $\mathcal{Q}_\delta$ si můžeme představit jako rozdělení $\R^n$ na krychle o~straně délky $\delta$ dotýkajících se pouze na hranici.

\subsection{Limity posloupností}\label{subsec:limity-posl-mp}

S limitou posloupnosti a~funkce jedné proměnné je čtenář nejspíše dobře seznámen. V~kontextu metrických prostorů definujeme pojem limity posloupnosti následovně.
\begin{definition}[Limita posloupnosti]\label{def:limita-mp}
    Mějme metrický prostor $(X,\varrho)$ a~posloupnost $\set{x_n}_{n=1}^\infty$,~kde $x_i\in X$ pro každé $i\in\N$. Pak posloupnost\index{limita!posloupnosti}\index{limita posloupnosti} $\set{x_n}_{n=1}^\infty$ má limitu $x\in X$,~píšeme
    \[\lim_{n\to\infty}x_n=x,\]
    nebo též $x_n\to x$,~pokud
    \[\forall\varepsilon>0\;\exists n_0\in\N\;\forall n\geqslant n_0: \varrho(x_n,x)<\varepsilon,\quad\text{tj.}\quad\lim_{n\to\infty}\varrho(x_n,x)=0.\]
\end{definition}
Limita posloupnosti je vždy určena jednoznačně (pokud existuje). Ve spojitosti s~limitami pro nás bude dále relevantní i~tzv. \emph{limes superior} a~\emph{limes inferior}.
\begin{definition}[Limes superior,~limes inferior]\label{def:limsup-liminf-mp}
    Nechť je dána posloupnost $\set{x_n}_{n=1}^\infty$,~kde $x_n\in\R$ pro každé $n\in\N$. Pak definujeme:
    \begin{itemize}
        \item Limes superior\index{limes superior} posloupnosti $\set{x_n}_{n=1}^\infty$ jako
        \[\limsup_{n\to\infty}x_n=\lim_{n\to\infty}(\sup\set{x_k\mid k\geqslant n}).\]
        \item Limes inferior\index{limes inferior} posloupnosti $\set{x_n}_{n=1}^\infty$ jako
        \[\limsup_{n\to\infty}x_n=\lim_{n\to\infty}(\inf\set{x_k\mid k\geqslant n}).\]
    \end{itemize}
\end{definition}
Jinak lze definovat limes superior,~resp. limes inferior jako supremum,~resp. infimum hromadných bodů posloupnosti. Speciálně platí,~že posloupnost $\set{x_n}_{n=1}^\infty$ má limitu $L$,~právě tehdy,~když
\[\limsup_{n\to\infty}x_n=\liminf_{n\to\infty}x_n=L.\]

Čtenář již nejspíše slyšel i~o tzv. \emph{Bolzanově-Cauchyově podmínce}\index{Bolzanova-Cauchyova podmínka}. Posloupnosti ji splňující nazýváme \emph{cauchyovské}.
\begin{definition}[Cauchyovská posloupnost]\label{def:cauchyovska-posloupnost}
    Nechť $(X,\varrho)$ je metrický prostor. Posloupnost $\set{x_n}_{n=1}^\infty$,~kde $x_i\in X$ pro každé $i\in N$,~nazveme \emph{cauchyovskou},~pokud splňuje:
    \[\forall\varepsilon>0\;\exists n_0\in\N\;\forall n,m\geqslant n_0:\varrho(x_n,x_m)<\varepsilon.\]
\end{definition}
Připomeňme rovněž,~že Bolzanova-Cauchyova podmínka je nutná,~nikoliv však postačující pro konvergenci posloupnosti v~$X$. Uvažme např. prostor $(\Q,|\cdot|)$ se stardardní eukleidovskou metrikou. Posloupnost $\set{x_n}_{n=1}^\infty$ je definujeme následovně:
\[x_0>0\quad\text{a}\quad x_n=\dfrac{1}{2}\left(x_{n-1}+\dfrac{2}{x_{n-1}}\right)\;,\;n\in\N.\]
Všechny její členy jsou v~$\Q$ a~lze též ověřit,~že je cauchyovská (např. proto, že má limitu v $\R$). Nicméně pro její limitu platí
\[\lim_{n\to\infty}x_n=\sqrt{2}\notin\Q.\]

V rámci metrických prostorů tedy mnohdy dává smysl omezovat pouze na takové prostory,~kde je Bolzanova-Cauchyova podmínka nutná i~postačující pro konvergenci. Též jim říkáme úplné.
\begin{definition}[Úplný metrický prostor]\label{def:uplny-mp}
    Metrický prostor $(X,\varrho)$ se nazývá \emph{úplný}\index{metrický prostor!úplný},~pokud každá cauchyovská posloupnost v~$X$ má limitu v~$X$.
\end{definition}
Příklady úplných metrických prostorů jsou např. uzavřený interval $\langle0,1\rangle$ se standardní eukleidovskou metrikou,~nebo prostor $\continuousfuncspace(\langle0,1\rangle)$ s~již zmíněnou supremovou metrikou.

\subsection{Limity funkcí}\label{subsec:limity-fci}

Od posloupností se přesuneme na chvíli k~limitám funkcí.
\begin{definition}[Limita funkce/zobrazení v~bodě]\label{def:limita-fce-v-bode}
    Nechť $(X,\varrho_1),(Y,\varrho_2)$ jsou metrické prostory a~$\mapping{f}{X}{Y}$ je zobrazení. Řekneme,~že $f$ má limitu\index{limita!funkce v~bodě} $y\in Y$ v~bodě $x_0\in X$,~píšeme
    \[\lim_{x\to x_0}f(x)=y,\]
    pokud
    \[\forall \varepsilon >0\;\exists \delta >0\;\forall x\in X: x\in B_\delta(x_0)\setminus\set{x_0}\implies f(x)\in B_\varepsilon(y).\]
    Navíc pokud platí
    \[\lim_{x\to x_0}f(x)=f(x_0),\]
    říkáme, že funkce $f$ je spojitá v bodě $x_0$. Je-li funkce $f$ spojitá v každém bodě množiny $M\subseteq X$, pak říkáme, že $f$ je spojitá na množině $M$.
\end{definition}
V případě, že $Y=\R$, lze definovat limes superior a~limes inferior podobně jako pro posloupnosti výše. Omezíme se opět jen na funkce $\mapping{f}{M}{\R}$.
\begin{definition}[Limes superior a~limes inferior pro funkce]\label{def:limsup-liminf-funkce}
    Nechť $(X,\varrho)$ je metrický prostor a~funkce $\mapping{f}{M}{\R}$,~kde $M\subseteq X$. Pak definujeme
    \begin{itemize}
        \item Limes superior funkce $f$ v~bodě $x_0$:
        \[\limsup _{x\to x_0}f(x)=\lim _{\varepsilon \to 0}\left(\sup\set{f(x):x\in M\cap B_\varepsilon(x_0)\setminus \{x_0\}}\right)\]
        \item Limes inferior funkce $f$ v~bodě $x_0$:
        \[\limsup _{x\to x_0}f(x)=\lim _{\varepsilon \to 0}\left(\inf\set{f(x):x\in M\cap B_\varepsilon(x_0)\setminus \{x_0\}}\right)\]
    \end{itemize}
\end{definition}

\subsection{Bodová a~stejnoměrná konvergence}\label{subsec:bodova-stejnomerna-konvergence}

V~matematické analýze často pracuje s~posloupnostmi funkcí. K~jejich limitám se váže dvojice důležitých termínů: \emph{bodová} a~\emph{stejnoměrná konvergence}.
\begin{definition}[Bodová a~stejnoměrná konvergence]\label{def:bodova-stejnomerna-konvergence}
    Nechť $\set{f_n}_{n=1}^\infty$ je posloupnost funkcí z~metrického prostoru $(\continuousfuncspace(\langle a,b\rangle),\varrho)$,~kde $\varrho$ je \emph{suprémová metrika},~tj. pro každou $f,g\in \continuousfuncspace(\langle a,b\rangle)$
    \[\varrho(f,g)=\sup_{x\in\langle a,b\rangle}|f_n(x)-f(x)|,\]
    přičemž $\set{f_n}_{n=1}^\infty$ konverguje k~funkci $f\in \continuousfuncspace(\langle a,b\rangle)$.
    Pak říkáme,~že posloupnost $\set{f_n}_{n=1}^\infty$ konverguje k~$f$
    \begin{itemize}
        \item \emph{bodově}\index{konvergence!bodová},~pokud pro každé $x$ platí $\lim_{n\to\infty}f_n(x)=f(x)$.
        \item \emph{stejnoměrně}\index{konvergence!stejnoměrná},~pokud
        \[\forall\varepsilon>0\;\exists n_0\in\N\;\forall n\geqslant n_0\;\forall x\in\langle a,b\rangle: \varrho(f_n,f)<\varepsilon.\]
        Píšeme $f_n\rightrightarrows f$.
    \end{itemize}
\end{definition}
Stejnoměrná konvergence implikuje konvergenci bodovou,~opačné tvezení však neplatí. Např. posloupnost funkcí $f_n(x)=x^n$,~kde $x\in\langle 0,1\rangle$,~konverguje k~funkci 
\[f(x)=\begin{cases}
    0 & x < 1,\\
    1 & x = 1.
\end{cases}\]
pouze bodově,~nikoliv stejnoměrně,~protože
\[\sup_{x\in\langle 0,1\rangle}|f_n(x)-f(x)|=\sup_{x\in\langle 0,1\rangle}|x^n-0|=1.\]

\subsection{Topologické pojmy}\label{subsec:topologicke-pojmy}

Důležitým pojmem v~teorii metrických prostorů jsou tzv. kompaktní množiny. S~tím se pojí následující termíny.
\begin{definition}[Pokrytí,~$\delta$-pokrytí a~zjemnění]\label{def:delta-pokryti-zjemneni}
    Nechť je dán metrický prostor $(X,\varrho)$,~$M\subseteq X$ a~systém množin $\mathcal{U}\subseteq\powset{X}$.
    \begin{itemize}
        \item $\mathcal{U}$ tvoří tzv. \emph{pokrytí}\index{pokrytí} množiny $M$,~pokud
        \[M\subseteq\bigcup_{U\in\mathcal{U}} U.\]
        Navíc o~pokrytí $\mathcal{U}$ říkáme,~že je \emph{otevřené}\index{pokrytí!otevřené},~pokud každá množina $U\in\mathcal{U}$ je otevřená.
        \item Pokud existuje $\delta>0$,~takové,~že pro každé $i\in\N$ platí $\diam{U_i}<\delta$,~pak $\mathcal{U}$ nazýváme \emph{$\delta$-pokrytím}\index{$\delta$-pokrytí}.
        \item Platí-li pro systém $\mathcal{G}\subseteq\powset{X}$,~že
        \[M\subseteq\bigcup_{G\in\mathcal{G}} G,\]
        a~zároveň $\mathcal{G}\subseteq\mathcal{U}$,~pak $\mathcal{G}$ je tzv. \emph{podpokrytí}\index{podpokrytí} pokrytí $\mathcal{U}$.
        \item Je-li $\mathcal{G}$ pokrytí $M$ a~pro každou $G\in\mathcal{G}$ existuje množina $U\in\mathcal{U}$,~taková,~že $G\subseteq U$,~pak $\mathcal{G}$ nazýváme \emph{zjemněním}\index{zjemnění} pokrytí $\mathcal{U}$. 
    \end{itemize}
\end{definition}
\begin{definition}[Kompaktní množina]\label{def:kompaktni-mnozina}
    Nechť $(X,\varrho)$ je metrický prostor. Množinu $M\subseteq X$ nazveme \emph{kompaktní}\index{množina!kompaktní},~jestliže pro každé otevřené pokrytí $\mathcal{U}$ existují indexy $n_1,\ldots,n_k\in I$, kde $I$ je indexová množina,~přičemž
    \[M\subseteq\bigcup_{i=1}^k U_{n_i}.\]
\end{definition}
Volněji řečeno,~pro kompaktní množinu lze z~jejího libovolného otevřeného pokrytí vybrat konečné podpokrytí. Dokazovat kompaktnost množiny z~definice může být mnohdy nepraktické,~nicméně vzhledem k~tomu,~že budeme často pracovat s~prostorem $\R^n$,~záležitost se nám v~tomto ohledu značně zjednodušuje.
\begin{theorem}[Heineho-Borelova]\label{thm:heine-borel}
    Nechť $\varrho_e$ je obvyklá eukleidovská metrika na $\R^n$~a $M\subseteq\R^n$. Následující výroky jsou ekvivalentní:
    \begin{enumerate}[label=(\roman*)]
        \item Množina $M$ je kompaktní.
        \item Množina $M$ je uzavřená a~omezená.
    \end{enumerate}
\end{theorem}
Důkaz věty lze nálézt např. v~\citep[str. 166]{NetukaAnalyza2014}. Pro obecný metrický prostor toto tvrzení již neplatí. Např. v~diskrétním metrickém prostoru je každá množina uzavřená i~omezená (stačí zvolit $r>1$,~tzn. $B_r(x)=X$). Nicméně uvážíme-li prostor $X=(0,1)$ s~dikrétní metrikou $\varrho$,~pak z~otevřeného pokrytí
\[(0,1)\subseteq\bigcup_{n=1}^\infty\left\langle\dfrac{1}{n+1},\dfrac{1}{n}\right)\]
nelze vybrat žádné konečné,~navzdory faktu,~že interval $(0,1)$ je v~$X$ uzavřený a~omezený.
\begin{theorem}\label{thm:kompakt-implikuje-omezenost}
    Nechť $M$ je kompaktní množina v~metrickém prostoru $(X,\varrho)$. Pak $M$ je omezená.
\end{theorem}
K důkazu věty~\ref{thm:kompakt-implikuje-omezenost} lze využít faktu,~že každá spojitá funkce definovaná na kompaktní množině nabývá svého maxima.

\subsection{Lipschitzovská zobrazení}\label{subsec:lipschitzovska-zobrazeni}

\begin{definition}[Lipschitzovské zobrazení]\label{def:bilipschitzovske-zobrazeni}
    Nechť $(X,\varrho_1),(Y,\varrho_2)$ jsou metrické prostory a $\mapping{f}{X}{Y}$ je zobrazení.
    \begin{itemize}
        \item Zobrazení $f$ nazveme \emph{lipschitzovské}\index{zobrazení!lipschitzovské},~pokud existuje konstanta $K>0$ taková,~že pro každé $x,y\in X$ platí
        \[\varrho_2(f(x),f(y))\leqslant K\varrho_1(x,y).\]
        Navíc pokud lze volit $K<1$,~pak $f$ nazýváme \emph{kontrakcí}\index{kontrakce}
        \item Zobrazení $f$ nazveme \emph{bilipschitzovské}\index{zobrazení!bilipschitzovské},~pokud existují konstanty $K_1,K_2>0$ takové,~že pro každé $x,y\in X$ platí
        \[K_1\varrho_1(x,y)\leqslant\varrho_2(f(x),f(y))\leqslant K_2\varrho_1(x,y).\]
    \end{itemize}
\end{definition}
Je celkem zjevné,~že lipschitzovská zobrazení jsou vždy spojitá. Pro každé $\varepsilon>0$ stačí zvolit kouli o~poloměru $\varepsilon/K$,~tedy
\[\varrho_2(f(x_0),f(x))\leqslant K\varrho_1(x_0,x)\leqslant K\cdot\dfrac{\varepsilon}{K}=\varepsilon,\]
což implikuje spojitost $f$ v~bodě $x_0$.