\section{Lebesgueova míra}\label{sec:lebesgueova-mira}

Jedněmi z~nejznámějších průkopníků v~oblasti teorie míry byli francouzský matematik \name{Camille Jordan}\footnote{1832--1922} a~italský matematik \name{Giuseppe Peano}\footnote{1958--1932}, kteří ve svých publikacích popsali způsob měření "velikosti" množin dnes známý jako \emph{Jordanova-Peanova míra}\index{Jordanova-Peanova míra}. Rovinu $\R^2$ si rozdělíme na čtvercovou síť, přičemž $S$ bude představovat součet obsahů všech čtverců síťě, jež jsou obsaženy ve vnitřku množiny $M$ a~$S^\prime$ je součet obsahů všech čtverců sítě mající společný alespoň jeden bod s~hranicí množiny $M$. Součet $S+S^\prime$ pak představuje obsah všech čtverců sítě, které obsahují body uzávěru množiny $M$ (viz obrázek~\ref{fig:jordan-peano-mira}).
\begin{figure}[h]
    \centering
    \begin{subfigure}{0.4\textwidth}
        \centering
        \includegraphics{ch02-jordan-peano-vnitrni-mira.pdf}
        \caption{Součet $S$}
    \end{subfigure}
    \qquad
    \begin{subfigure}{0.4\textwidth}
        \centering
        \includegraphics{ch02-jordan-peano-vnejsi-mira.pdf}
        \caption{Součet $S+S^\prime$}
    \end{subfigure}
    \caption{Vnitřní a~vnější Jordanova-Peanova míra množiny $M$}
    \label{fig:jordan-peano-mira}
\end{figure}
Při zjemňování čtvercové sítě konvergují součty $S$ a~$S+S^\prime$ k~limitám, které po řadě nazýváme \emph{vnitřní} a~\emph{vnější Jordanova-Peanova míra}\index{Jordanova-Peanova míra}\index{míra!Jordanova-Peanova}. Pokud se obě tyto limity shodují, mluvíme zkrátka \emph{Jordanově-Peanově míře} množiny $M$. Tato úvaha položila základy pro vznik teorie míry. \cite{Sarmanova1996}

V předešlé sekci~\ref{sec:prostory-s-mirou} jsme si přiblížili pojem \emph{"míra"}\index{míra} obecně a~podívali jsme se na několik příkladů. Obecnou ideu měření "velikosti" lze založit např. na aproximaci obecné množiny pomocí \emph{spočetných sjednocení útvarů}, jejichž "velikost" umíme jednoduše určit. V~dalším textu se omezíme pouze na množinu $\R^n$.

Na zmíněné myšlence je postavena definice \emph{$n$-rozměrné Lebesgueovy míry}, kdy obecnou množinu budeme pokrývat pomocí \emph{kvádrů}. Připomeňme, že obecně\linebreak\mbox{$n$-rozměrným} kvádrem\index{$n$-rozměrný kvádr} $I$ rozumíme kartézský součin \emph{intervalů}
\[\langle a_1,b_1\rangle,\ldots,\langle a_n,b_n\rangle\subseteq\R,\]
tj.
\[I=\prod_{i=1}^{n}\langle a_i,b_i\rangle=\langle a_1,b_1\rangle\times\langle a_2,b_2\rangle\times\dots\times\langle a_n,b_n\rangle,\]
a jeho objem\index{kvádr!objem kvádru} definujeme jako
\[\vol_n(I)=\prod_{i=1}^{n}(b_i-a_i).\]
Lze nejspíše ihned vidět, že objem $\vol_n(I)$ je \emph{aditivní} i~\textit{subaditivní}.

Nyní si definujeme tzv. \emph{vnější Lebesgueovu míru}.
\begin{definition}[Vnější Lebesgueova míra]\label{def:vnejsi-lebegueova-mira}
    Nechť $A\subseteq\R^n$. Pak vnější $n$-rozměr\-nou Lebesgueovou mírou\index{Lebesgueova míra!vnější}\index{Lebesgueova míra!$n$-rozměrná} $A$ je
    \[\lebesgueoutermeasure{n}(A)=\inf\set{\sum_{j=1}^{\infty}\vol_n(I_j)\;\middle|\;\text{$I_j$ je kvádr pro každé $j\in\N$}\;,\;A\subseteq\bigcup_{j=1}^\infty I_j}.\]
\end{definition}
Vnější Lebesgueova míra množiny intuitivně zachycuje informaci o~"velikosti" dané množiny. Zjevně platí pro libovolnou množinu $A\subseteq\R^n$, že $\lebesgueoutermeasure{n}(A)\in\R_0^+$, protože $\vol_n(I_j)\geqslant0$ pro každé $j\in\N$.
\begin{example}\label{ex:lebegueova-mira-trivialni-priklady}
    Ukažme si některé triviální příklady výpočtů vnější Lebesgueovy míry z~definice (viz~\ref{def:vnejsi-lebegueova-mira}). Budeme tedy hledat příslušné pokrytí dané množiny.
    \begin{itemize}
        \item Pro prázdnou množinu $\emptyset$ je $\lebesgueoutermeasure{n}(\emptyset)=0$, neboť $\emptyset\subseteq\prod_{i=1}^{n}\langle0,0\rangle$ (prázdnou množinu lze pokrýt jakýmkoliv kvádrem) a
        \[\vol_n\left(\prod_{i=1}^{n}\langle0,0\rangle\right)=0.\]
        \item Mějme libovolnou konečnou množinu $A=\set{x_1,x_2,\dots,x_n}\subseteq\R^n$. Pro každé $x_j$ stačí položit $I_j=\set{x_j}$ pro každé $1\leqslant j\leqslant n$, což je degenerovaný interval splňující $\vol_n(I_j)=0$. Poznamenejme, že i~singleton představuje kvádr.
        \item Pro libovolnou spočetnou množinu $A=\set{x_i\mid i\in\N}\subseteq\R^n$ je $\lebesgueoutermeasure{n}(A)=0$. Pokrytí volíme stejně jako v~předešlém bodě. Tedy např. pro $\Q$ je $\lebesgueoutermeasure{1}(\Q)=0$, neboť $\Q$ je spočetná.
        \item Pro množinu $\R$ je $\lebesgueoutermeasure{1}(\R)=\infty$, avšak pro 
        \[A=\set{(x,0)\mid x\in\R}\subset\R^2\]
        (osa $x$ v~$\R^2$) je $\lebesgueoutermeasure{2}(A)=0$. Stačí definovat kvádr
        \[I_j=\langle j,j+1\rangle\times\langle 0,0\rangle\]
        pro každé $j\in\Z$.
    \end{itemize}
\end{example}
Jako poslední si ukážeme, že vnější Lebesgueova míra $n$-rozměrného kvádru je rovna jeho objemu.
\begin{proposition}\label{prop:lebegueova-mira-objem-kvadru}
    Je-li $I\subset\R^n$ kvádr, pak $\lebesgueoutermeasure{n}(I)=\vol_n(I)$.
\end{proposition}
\begin{proof}
    Ukážeme zvlášť, že $\lebesgueoutermeasure{n}(I)\leqslant\vol_n(I)$ a~$\lebesgueoutermeasure{n}(I)\geqslant\vol_n(I)$.
    \begin{itemize}
        \item Zvolme pokrytí $\mathcal{I}=\set{I_1,I_2,\ldots}$ kvádru $I$, tzn.~$I\subseteq\bigcup_{i=1}^\infty I_i$, tak, aby platilo
        \[\sum_{i=1}^{\infty}\vol_n(I_i)\leqslant(1+\varepsilon)\lebesgueoutermeasure{n}(I)\]
        pro nějaké $\varepsilon>0$. Nyní si zvolíme nové kvádry\footnote{Formálně $\mathcal{I}$ tvoří zjemnění pokrytí $\mathcal{J}$.} $\mathcal{J}=\set{J_1,J_2,\ldots}$ tak, aby pro každé $i\in\N$ bylo
        \[I\subset\interior{J}\land\vol_n(J_i)\leqslant(1+\varepsilon)\vol_n(I_i).\]
        To není nikterak složité, stačí např. pro každé $i\in\N$ položit
        \[J_i=\prod_{j=1}^{n}\left\langle x_j-r_j\sqrt[n]{1+\varepsilon},x_j+r_j\sqrt[n]{1+\varepsilon}\right\rangle,\]
        kde
        \[I_i=\langle x_1-r_1,x_1+r_1\rangle\times\ldots\times\langle x_n-r_n,x_n+r_n\rangle.\]
        Protože však $I$ je uzavřená a~omezená množina, je podle věty~\ref{thm:heine-borel} kompaktní, tedy z~otevřeného pokrytí $\interior{J_1},\interior{J_2},\ldots$ lze vybrat konečné podpokrytí. Existuje tedy $m\in\N$ takové, že
        \[I\subseteq\bigcup_{i=1}^m J_i.\]
        Celkově tedy dostáváme
        \begin{align*}
            \vol_n(I)&\leqslant\sum_{i=1}^{m}\vol_n(J_i)\leqslant(1+\varepsilon)\sum_{i=1}^{m}\vol_n(I_i)\leqslant(1+\varepsilon)\sum_{i=1}^{\infty}\vol_n(I_i)\\
            &\leqslant (1+\varepsilon)^2\lebesgueoutermeasure{n}(I)
        \end{align*}
        pro každé $\varepsilon>0$, což dokazuje požadovanou nerovnost.
        \item Důkaz opačné nerovnosti je velmi jednoduchý. Kvádr $I$ totiž reprezentuje pokrytí sebe samotného, tzn. lze volit $I_1=I$ a~zbylé kvádry $I_j$, kde $j\geqslant 2$, mohou být libovolné s~nulovým objemem.
    \end{itemize}
\end{proof}
\begin{remark}
    Vraťme se ještě k~větě~\ref{thm:mira-vlastnosti} o~vlastnostech míry, konkrétně bod~\ref{thm:mira-nerost-posl}. Předpoklad $\mu(A_1)<\infty$ zde vynechat nelze. Snadno si rozmyslíme, že pokud uvážíme množiny $A_j=\langle j,\infty)$, pak $\lebesgueoutermeasure{n}(A_j)=\lebesgueoutermeasure{n}(\langle j,\infty))=\infty$ pro každé $j\in\N$ a~tedy
    \[\bigcap_{i=1}^\infty A_i=\emptyset.\]
    Lze vidět, že zatímco $\lim_{j\to\infty}\mu(A_j)=\infty$, tak $\mu\left(\bigcap_{i=1}^\infty A_i\right)=0$.
\end{remark}

Z příkladů~\ref{ex:lebegueova-mira-trivialni-priklady} a~\ref{prop:lebegueova-mira-objem-kvadru} můžeme tušit, že pro rozumně zvolené množiny zachycuje vnější Lebesgueova míra jejich intuitivní "velikost". V~případě intervalu odpovídá jeho délce, v~případě diskrétní množiny je nulová a~podobně např.~pro obdélník odpovídá jeho obsahu, pro kvádr jeho objemu atd.

Nyní se však nabízí jedna otázka. Čtenář by mohl již od chvíle, kdy jsme zavedli pojem vnější Lebesgueovy míry (opět viz definice~\ref{def:vnejsi-lebegueova-mira}), namítat, co nás opravňuje nazývat zobrazení $\lebesgueoutermeasure{n}$ mírou ve smyslu definice~\ref{def:prostor-s-mirou}. Jak víme, že splňuje podmínku $\sigma$-aditivity? Odpověď na tuto otázku není zcela přímočará ani jednoduchá.

Bohužel v~případě vnější Lebesgueovy míry pro obecnou množinu neplatí vlastnost aditivity, tedy existují množiny $A,B\in\mathcal{A}$ takové, že
\[\lebesgueoutermeasure{n}(A\cup B)\neq\lebesgueoutermeasure{n}(A)+\lebesgueoutermeasure{n}(B).\]
Příklad takové množiny poskytuje např. takzvaná \emph{Vitaliho konstrukce}\index{Vitaliho konstrukce}, se kterou přišel italský matematik \name{Giuseppe Vitali}\footnote{1875--1932}\index{Vitali}\index{Giuseppe Vitali} roku 1905, využívající invariance vnější Lebesgueovy míry vůči posunutí, tzn. $\lebesgueoutermeasure{n}(x+A)=\lebesgueoutermeasure{n}(A)$. \cite{OConnor2025} V~rámci tohoto textu se jí zde zabývat nebudeme, avšak pro zájemce doporučujeme zdroje \citep[str. 3]{Lukes2013} a~\cite{Verner2025}, kde je Vitaliho množina podrobněji rozepsána.

Je tedy potřeba se omezit na takové množiny, kde je $\lebesgueoutermeasure{n}$ aditivní. Existuje více způsobů jejich charakterizace, avšak my si zde uvedeme ten, se kterým přišel řecký matematik \name{Constantin Carathéodory}\index{Constantin Carathéodory}\footnote{1873--1950}.
\begin{definition}[Lebesgueovská měřitelnost]\label{def:lebesgueovska-meritelnost}
    Množinu $A\subseteq\R^n$ nazveme (lebesgueovsky) měřitelnou\index{množina!lebesgueovsky měřitelná}, pokud pro každou množinu $G$ platí
    \[\lebesgueoutermeasure{n}(G)=\lebesgueoutermeasure{n}(G\cap A)+\lebesgueoutermeasure{n}(G\setminus A).\]
    Systém všech měřitelných množin v~$\R^n$ značíme $\mathcal{L}(\R^n)$.  Pokud $A\in\mathcal{L}(\R^n)$, pak číslo $\lebesguemeasure{n}(A)=\lebesgueoutermeasure{n}(A)$ nazýváme $n$-rozměrnou Lebesgueovou mírou množiny $A$.
\end{definition}
Podmínka v~definici~\ref{def:lebesgueovska-meritelnost} se někdy nazývá \emph{Carathéodoryho kritérium}. Zjednodušeně říká, že množina $A$ je lebesgueovsky měřitelná, když při "rozdělení" \emph{libovolně} zvolené množiny $G$ na dvě části pomocí $A$ lze míru $G$ stanovit součtem měr daných částí (viz obrázek~\ref{fig:caratheodoryho-kriterium}). Zároveň je dobré (a snadné) si rozmyslet, že kvádry, které figurují v~definici vnější Lebesgueovy míry, jsou měřitelné.
\begin{figure}[h]
    \centering
    \includegraphics{ch02-caratheodoryho-kriterium.pdf}
    \caption{Ilustrace měřitelnosti množiny $A$}
    \label{fig:caratheodoryho-kriterium}
\end{figure}
O systému $\mathcal{L}(\R^n)$ a~Lebesgueově míře $\lebesguemeasure{n}$ lze dokázat následující tvrzení.
\begin{theorem}\label{thm:prostor-s-Lebesgueovou-mirou}
    Platí:
    \begin{enumerate}[label=(\roman*)]
        \item $(\R^n,\mathcal{L}(\R^n))$ je měřitelný prostor.
        \item $(\R^n,\mathcal{L}(\R^n),\lebesguemeasure{n})$ je prostor s~mírou.
    \end{enumerate}
\end{theorem}
Čtenář snad promine, že formální důkaz v~případě tohoto tvrzení v~zájmu zachování stručnosti textu zcela vynecháme, nicméně zvídavý jedinec jej může nalézt např. v~knize \citep[str. 347]{Royden2010}, kde jsou příslušné záležitosti rozepsány.