\section{Hausdorffova míra a Hausdorffova dimenze}\label{sec:hausdorffova-mira-dimenze}

Způsobů,~jak definovat dimenzi je celá řada. Zatím jsme společně prozkoumali box-counting dimenzi (resp. některá její pojetí),~avšak lze najít více způsobů její definice\footnote{Některé další jsou sepsány např. v~\citep[str. 40]{Falconer2014}.}. Pravděpodobně však nejstarším exemplářem svého druhu je tzv. \emph{Hausdorffova dimenze} a s~ní související \emph{Hausdorffova míra},~které hrají ve fraktální geometrii velice podstatnou roli. Stále se však budeme zabývat pouze množinami v~$\R^n$. Jsou pojmenovány po německém matematikovi \name{Felixi Hausdorffovi} (1868--1942).
\begin{figure}[h]
    \centering
    \includegraphics[width=0.4\textwidth]{felix-hausdorff.jpg}
    \caption[Felix Hausdorff,~1868--1942]{Felix Hausdorff\footnote{Převzato z~\cite{OConnorHausdorff2025}},~1868--1942}
    \label{fig:felix-hausdorff}
\end{figure}

\subsection{Definice Hausdorffovy míry}\label{subsec:hd-mira-definice}

\begin{definition}\label{def:hd-mira-delta}
    Nechť je dána množina $F\subseteq\R^n$ a $s>0$. Pak pro každé $\delta>0$ definujeme zobrazení
    \[\mathcal{H}_\delta^s(F)=\inf\set{\sum_{i=1}^{\infty}(\diam{F_i})^s\;\middle|\;F\subseteq\bigcup_{i=1}^\infty F_i\;,\;\diam{F_j}\leqslant\delta\;\text{pro}\;1\leqslant j\leqslant n}.\]
\end{definition}
Na první pohled si lze všimnout,~že pro $0<\delta_1<\delta_2$ je $\mathcal{H}_{\delta_1}^s(F)\leqslant\mathcal{H}_{\delta_2}^s(F)$. Jinými slovy,~pro $\delta\to 0$ hodnota $\mathcal{H}_\delta^s(F)$ klesá. Toto není nikterak těžké si rozmyslet,~neboť pro $\delta_1<\delta_2$ existuje $\delta_1$-pokrytí $\mathcal{F}_1$,~takové,~že je podpokrytím $\delta_2$-pokrytí $\mathcal{F}_2$ množiny $F$,~tedy $\mathcal{F}_1\subseteq\mathcal{F}_2$. To znamená,~že
\begin{align*}
    \mathcal{H}_{\delta_1}^s(F)&=\inf\set{\sum_{U\in\mathcal{F}_1}(\diam{U})^s\;\middle|\;\text{$\mathcal{F}_1$ je $\delta_1$-pokrytí}}\\
    &\leqslant\inf\set{\sum_{U\in\mathcal{F}_2}(\diam{U})^s\;\middle|\;\text{$\mathcal{F}_2$ je $\delta_2$-pokrytí}}=\mathcal{H}_{\delta_2}^s(F).
\end{align*}
Zároveň je z~definice \ref{def:hd-mira-delta} zjevné,~že $\mathcal{H}_\delta^s(F)\geqslant 0$.

\todo{Doplnit důkaz,~že Hausdorffova míra je mírou}