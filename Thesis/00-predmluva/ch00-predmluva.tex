\chapwithtoc{Předmluva}\label{chapter:predmluva}

\section*{Co?}

Text poskytuje přehled základních aspektů fraktální geometrie, která v současné době tvoří nejen rozsáhlou, ale též poměrně oblíbenou disciplínu, a to jak v řadách matematiky znalejších jedinců (především studentů matematiky a technických oborů), tak i v rámci širší matematické veřejnosti. Lze tak nalézt velké množství literatury, které se jí zabývají z různých úhlů pohledu. Některé publikace cílí více na matematickou podstatu, jiné zase na praktické aspekty. V rámci tohoto textu se pokusíme pojmout jak povrchní podstatu fraktální geometrie, tak i její formální stránku.

\section*{Kdo?}

Relevance jednotlivých kapitol a sekcí zavisí na konkrétním očekávání a zájmu čtenáře. Kapitola \ref{chapter:uvod_do_fraktalu} věnovaná úvodnímu povídání o fraktální geometrii a výkladu základních konceptů nepředpokládá téměř žádné předešlé znalosti problematiky. Naopak pro lehce znalejšího čtenáře stojí za zmínku kapitola \ref{chapter:klasifikace-fraktalu} věnovaná klasifikaci základních typů fraktálů, kterou si lze také přečíst bez hlubší znalosti související teorie (ač se na ni budeme na vhodných místech odkazovat). Výjimku tvoří akorát sekce \ref{subsec:formalni-jazyky-a-gramatiky} a \ref{subsec:hausdorffuv-mp-kontrakce}. Dále pak pro programátory je vhodná kapitola \ref{chapter:generovani-fraktalu} věnovaná rozboru a implementaci algoritmů pro generování fraktálů. Z didaktických důvodů jsme zvolili jazyk Python, který i přes leckteré jiné nedostatky představuje vhodný nástroj pro názornost implementace (především díky jeho jednoduché syntaxi). Avšak i znalci jiných jazyků si mohou přijít na své, neboť každý algoritmus uvádíme i s jeho pseudokódem. Ten sám o sobě představuje pouze abstraktní popis algoritmu, avšak bez potřeby většího úsilí jej lze převést do libovolného programovacího jazyka.

Naopak pro náročnějšího čtenáře jsou připraveny kapitoly \ref{chapter:teorie-miry-a-dimenze} a \ref{chapter:hausdorffuv-mp} zabývající se teorií míry, konceptu dimenze a Hausdorffovým metrickým prostorem, na nichž je následně vystavěna další teorie.

\section*{Znalosti?}

Ač jsou kapitoly \ref{chapter:uvod_do_fraktalu}, \ref{chapter:klasifikace-fraktalu} a \ref{chapter:generovani-fraktalu} určeny pro méně náročné, přesto je nejspíše jasné, že některé znalosti budou přecijen třeba. Celkově se od čtenáře očekává znalost středoškolské matematiky (především funkce, posloupnosti a základy teorie množin). Naopak kapitoly \ref{chapter:teorie-miry-a-dimenze} a \ref{chapter:hausdorffuv-mp} předpokládají základní znalosti z oblasti matematické analýzy, především metrických prostorů (shrnutí potřebné teorie lze nalézt v sekci \ref{sec:zakladni-pojmy-a-znaceni}).

\section*{Kam dál?}

Jak jsme již na začátku zmínili, knih o fraktální geometrii existuje opravdu mnoho. Na poli české literatury je jednou z nejznámějších publikací kniha \cite{Zelinka2006}, která se zabývá především praktickými aplikacemi. Též stojí za zmínění i kniha \cite{Voracova2022}, která obsahuje kapitolu věnovanou zmíněné problematice.

Z cizojazyčných zdrojů doporučuji pro zdatnější čtenáře především knihu \cite{Falconer1989}, kde lze nalézt mnohé informace, často silně nad rámec tohoto textu. Též se lze dále podívat do knih \cite{Prusinkiewicz1990}, \cite{Edgar2008} a \cite{Mattila1995}.

(Inspirováno knihou \cite{Hladik2019}.)