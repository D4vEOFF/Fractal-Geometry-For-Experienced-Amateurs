\chapter{Úvod do fraktálů}\label{chapter:uvod_do_fraktalu}

Pod pojmem "geometrie" si čtenář pravděpodobně vybaví rovinnou či prostorovou geometrii pracující s~jednoduchými útvary jako trojúhelník, obdélník, kruh, kvádr, jehlan apod.~a~s~útvary z~nich složenými. V~reálném světě tak lze nalézt mnoho uplatnění této "standardní" geometrie, kupříkladu ve strojírenství, stavebnictví, i~jinde. Často tak máme o~světě představu právě ve smyslu eukleidovské geometrie\index{geometrie!eukleidovská}\index{eukleidovská geometrie}. Lze však nalézt řadu objektů, pro jejichž popis jsou tyto představy limitující. Například v~přírodě mrak nelze popsat jako kouli, horu nelze popsat jako jehlan a~ani pobřeží nelze určitě popsat jako kružnici.\par

Mnohé přírodní obrazce již nelze jednoduše modelovat pomocí nástrojů "standardní" eukleidovské geometrie, s~níž jsme seznámeni již od základní školy a~která byla po mnoho století základním nástrojem pro popis a~porozumění matematickému prostoru. Často hraje roli i~jistá nahodilost projevující se v~charakteru takových obrazců. \emph{Fraktální geometrie}\index{fraktální geometrie} se zabývá nepravidelnými a~často se opakujícími vzory, které se vyskytují v~přírodě i~umění. Tyto vzory jsou často složité a~zdánlivě chaotické, ale fraktální geometrie nám umožňuje je analyzovat a~pochopit.\par

Vznik fraktální geometrie se datuje od roku \emph{1975} a~za jejího zakladatele je považován francouzsko-americký matematik \name{Benoît~Mandelbrot} \mbox{(1924--2010)}\index{Mandelbrot}\index{Benoît Mandelbrot}. Historicky za jejím vznikem stály objevy matematických struktur, které nezapadaly do~"představ" v~rámci tehdy známé eukleidovské geometrie\index{geometrie!eukleidovská}\index{eukleidovská geometrie}. Byly často považovány za "patologické", nicméně matematici, kteří je vytvořili, je považovali za důležité kvůli ukázce bohatých možností, jež nabízí svět matematiky překračující možnosti jednoduchých struktur, s~nimiž se setkávali. \citep[str. 33]{Mandelbrot1983}

\section{Jak dlouhé je pobřeží Velké Británie?}\label{sec:pobrezi_velke_britanie}
Položme si na začátek trochu jinou otázku,~kterou si z~počátku položil i~Benoît Mandelbrot: \emph{Jaká je podstata tvaru pobřeží?} Ta se stala podstatnou v~jeho práci \emph{"How Long Is the Coast of Britain?"}. Uvažme část pobřeží s~počátečním a koncovým bodem (viz obrázek~\ref{fig:pobrezi}).
\begin{figure}[h]
    \centering
    \includegraphics[scale=\normalipe]{ch01-pobrezi.pdf}
    \caption{Příklad mapy pobřeží se spojnicí bodů $A$ a $B$.}
    \label{fig:pobrezi}
\end{figure}
Zjevně jeho délka je zdola omezena délkou spojnice koncových bodů $A$ a $B$,~nicméně typické pobřeží je velmi nepravidelné a klikaté,~a jeho skutečná délka je tak často mnohem delší. Existují různé metody,~které nám umožňují určit přesněji jeho délku. Několik z~nich je popsáno v~knize \citep[str. 79]{Mandelbrot1983},~pro uvedení do problematiky si zde však vystačíme s~tou nejednodušší.\par
Předpokládejme,~že pobřeží,~které zkoumáme,~má pevné hranice (tj. zanedbáváme např. přílivy a odlivy nastávající během roku),~a dále jsme schopni rozlišovat libovolně krátké vzdálenosti.\par
Mějme zadané libovolné $\varepsilon>0$. Podél pobřeží začneme umisťovat tyče tak,~že po každém umístění provedeme na mapě krok délky \textbf{nejvýše} $\varepsilon$,~přičemž začínáme v~bodě $A$ a postupujeme až k~bodu $B$ (popř. pokud měříme délku pobřeží ostrova,~pokračujeme dokud se nevrátíme tam,~kde jsme začali). Předpokládejme,~že jsme provedli celkově $n(\varepsilon)$ kroků (jejich počet je závislý na zvolené délce kroku). \emph{Přibližnou délku pobřeží} $\ell(\varepsilon)$ pak stanovíme jako
\begin{equation*}
    \ell(\varepsilon)=n(\varepsilon)\cdot\varepsilon.
\end{equation*}
\begin{figure}[h]
    \centering
    \includegraphics[scale=\normalipe]{ch01-pobrezi-aproximace.pdf}
    \caption{Odhad délky pobřeží,~kde $n=10$ při zvoleném $\varepsilon$.}
    \label{fig:pobrezi_aproximace}
\end{figure}
Nyní by nás mohlo napadnout,~že pro zmenšující se $\varepsilon$,~tj. $\varepsilon\to0^+$,~bude hodnota $\ell(\varepsilon)$ konvergovat ke skutečné délce pobřeží. Tzn.~označíme-li skutečnou délku pobřeží $L$,~pak bychom mohli očekávat,~že platí
\begin{equation}\label{eq:aproximace_limita}
    L=\lim_{\varepsilon\to0^+}{\ell(\varepsilon)}.
\end{equation}
Jenže,~bohužel,~limita \eqref{eq:aproximace_limita} ve skutečnosti bude rovna $\infty$. Proč? Je třeba si uvědomit,~že zde pracujeme s~\emph{mapou} pobřeží,~která má určité \emph{měřítko}. Pokud bychom měli pobřeží na mapě s~měřítkem $1\,:\,100\;000$,~uvidíme méně detailů,~než kdybychom jej zkoumali na mapě s~měřítkem $1\,:\,1\;000$. (Viz obrázek~\ref{fig:pobrezi_zoom}.)\par
\begin{figure}[h]
    \centering
    \includegraphics[scale=\normalipe]{ch01-pobrezi-zoom.pdf}
    \caption{Část pobřeží od bodu $A$ v~menším měřítku.}
    \label{fig:pobrezi_zoom}
\end{figure}
Nově odhalené detaily (menší poloostrůvky apod.) zde přispívají k~celkové délce pobřeží $\ell(\varepsilon)$. Postupným zvětšováním měřítka mapy bychom tak odhalili další detaily. Naše původní idea tak selhává,~neboť (v "klasickém" pojetí délky) pro $\varepsilon\to0^+$ hodnota $\ell(\varepsilon)$ poroste nade všechny meze,~tj. $\lim_{\varepsilon\to0^+}{\ell(\varepsilon)}=\infty$.\par
Nabízí se otázka: Proč se toto děje? Pokud se ohlédneme zpět za eukleidovskou geometrií\index{geometrie!eukleidovská}\index{eukleidovská geometrie},~tento problém zde nenastává. Např. u~kružnice v~eukleidovské rovině $\mathbb{E}_2$ změnou měřítka žádné další detaily křivky neobjevíme (podobně u~jiných geometrických útvarů,~viz obrázky~\ref{subfig:kruznice} a~\ref{subfig:kruznice_zoom}). 
\begin{figure}[h]
    \centering
    \begin{subfigure}{\subfigwidth}
        \centering
        \includegraphics[scale=\normalipe]{ch01-kruznice.pdf}
        \caption{Kružnice v~menším měřítku.}
        \label{subfig:kruznice}
    \end{subfigure}
    \quad
    \begin{subfigure}{\subfigwidth}
        \centering
        \includegraphics[scale=\normalipe]{ch01-kruznice-zoom.pdf}
        \caption{Část kružnice ve větším měřítku.}
        \label{subfig:kruznice_zoom}
    \end{subfigure}
\end{figure}
Díky tomu můžeme v~případě počítání obvodu kružnice použít např. Archimédovu metodu.\par
Máme-li kružnici o~poloměru $r>0$,~pak jí můžeme vepsat libovolný pravidelný $n$-úhelník (viz obrázek~\ref{subfig:archimedova_metoda}).
\begin{figure}[h]
    \centering
    \begin{subfigure}{\subfigwidth}
        \centering
        \includegraphics[scale=\normalipe]{ch01-archimedova-metoda.pdf}
        \caption{Pravidelný osmiúhelník vepsaný kružnici.}
        \label{subfig:archimedova_metoda}
    \end{subfigure}
    \quad
    \begin{subfigure}{\subfigwidth}
        \centering
        \includegraphics[scale=\normalipe]{ch01-archimedova-metoda-cast-nuhelniku.pdf}
        \caption{Část vepsaného pravidelného $n$-úhelníku.}
        \label{subfig:archimedova_metoda_cast_nuhelniku}
    \end{subfigure}
    \caption{Princip Archimédovy metody.}
    \label{fig:princip_archimedovy_metody}
\end{figure}
Obvod pravidelného $n$-úhelníku is označíme $O_n$ a délku jeho strany $x$ (viz obrázek~\ref{subfig:archimedova_metoda_cast_nuhelniku}). Tu jsme schopni stanovit užitím elementární goniometrie,~tj.
\begin{equation*}
    x=2r\cdot\sin{\dfrac{\pi}{n}},
\end{equation*}
a tedy obvod
\begin{equation*}
    O_n=2rn\cdot\sin{\dfrac{\pi}{n}}.
\end{equation*}
Pro rostoucí $n$ bude obvod pravidelného $n$-úhelníku stále lépe aproximovat obvod původní kružnice (viz obrázek~\ref{fig:archimedova_metoda_presnejsi}).
\begin{figure}[h]
    \centering
    \includegraphics[scale=\normalipe]{ch01-pobrezi-aproximace-presnejsi.pdf}
    \caption{Aproximace obvodu kružnice pomocí pravidelného šestnáctiúhelníku.}
    \label{fig:archimedova_metoda_presnejsi}
\end{figure}
Limitním přechodem (tj. pro $n\to\infty$) tak můžeme odvodit vzorec pro obvod kružnice:
\begin{align*}
    O&=\lim_{n\to\infty}{2rn\cdot\sin{\dfrac{\pi}{n}}}=2r\cdot\lim_{n\to\infty}{n\cdot\sin{\dfrac{\pi}{n}}}=2\pi r\cdot\lim_{n\to\infty}{\cdot\dfrac{\sin{\dfrac{\pi}{n}}}{\dfrac{\pi}{n}}}=2\pi r.
\end{align*}
Idea aproximace pomocí "zjemňování" zde skutečně funguje a délka ve standardním pojetí tak dává smysl,~jak bychom mohli očekávat. Křivka,~kterou tvoří pobřeží,~má však oproti kružnici jiný geometrický charakter. Délka pobřeží $\infty$,~k~níž Mandelbrot došel,~tak dává smysl \emph{z geometrického pohledu},~avšak výsledek to není moc užitečný.
\input{01-uvod/s02-sobepodobnost.tex}
\section{Fraktální dimenze}\label{sec:fraktalni_dimenze}

\subsection{Chápání konceptu dimenze}\label{subsec:koncept-dimenze}

Útvary uvedené v~sekci~\ref{sec:sobepodobnost} ilustrují vlastnost soběpodobnosti, kterou jsme si (zatím neformálně) popsali. V~eukleidovské geometrii lze však u~mnohých základních objektů pozorovat stejnou vlastnost. Např. čtverec lze určitě prohlásit v~jistém smyslu za soběpodobný, neboť jej lze rozdělit na podobné útvary (viz obrázek~\ref{fig:sobepodobnost-ctverce}).
\begin{figure}[h]
    \centering
    \begin{subfigure}[b]{\subfigwidth}
        \centering
        \includegraphics[scale=\normalipe]{ch01-ctverec-sobepodobnost.pdf}
        \caption{Rozdělení na čtyři menší čtverce}
        \label{subfig:sobepodobnost-ctverce-1}
    \end{subfigure}
    \begin{subfigure}[b]{\subfigwidth}
        \centering
        \includegraphics[scale=\normalipe]{ch01-ctverec-sobepodobnost-2.pdf}
        \caption{Jiná možnost rozdělení čtverce}
        \label{subfig:sobepodobnost-ctverce-2}
    \end{subfigure}
    \caption{Soběpodobnost čtverce}
    \label{fig:sobepodobnost-ctverce}
\end{figure}
Podobně např. i~obyčejná úsečka je taktéž soběpodobná, protože ji můžeme rozdělit na $k$ stejných částí (viz obrázek~\ref*{fig:sobepodobnost-usecky}).\par
\begin{figure}[h]
    \centering
    \includegraphics[scale=\normalipe]{ch01-usecka-sobepodobnost.pdf}
    \caption{Úsečka rozdělená na šest stejných částí}
    \label{fig:sobepodobnost-usecky}
\end{figure}
K čemu nám uvědomění takové skutečnosti vlastně je? Zmenšíme-li úsečku $k$-krát, pak budeme potřebovat přesně $k$ těchto částí, abychom dostali úsečku původní délky. U~čtverce (nebo obdélníku obecně) při změnšení délky strany $k$-krát budeme potřebovat $k^2$ daných útvarů pro obdržení čtverce s~původním obsahem.\footnote{Obdélník zmenšený $k$-krát bude mít strany délek $a/k,\,b/k$, tedy jeho obsah bude
\[\dfrac{ab}{k^2}=\dfrac{S}{k^2},\]
kde $S$ je obsah původního obdélníka.}
Pro krychli bude situace podobná, $k$-krát zmenšená kopie bude potřeba $k^3$-krát, abychom dostali krychli o~původním objemu (viz obrázek~\ref{fig:krychle-sobepodobnost}).
\begin{figure}[h]
    \centering
    \includegraphics[scale=\normalipe]{ch01-krychle-sobepodobnost.pdf}
    \caption{Krychle rozdělená na 27 stejných částí}
    \label{fig:krychle-sobepodobnost}
\end{figure}
Lze si všimnout, že v~závislosti na \emph{dimenzi} objektu se mění daný exponent. Vztah lze tak zobecnit na
\begin{equation}\label{eq:pocet-utvaru}
    N(k)=k^d,
\end{equation}
kde $N(k)$ je počet nových útvarů v~závislosti na faktoru $k$ a~číslu $d$. Intuitivně je nejspíše jasné, že dimenze útvaru je takové číslo $d$, pro které platí rovnost \eqref{eq:pocet-utvaru}.

Toto je jeden z~možných způsobů, jak lze chápat koncept dimenze. Jednoduchou úpravou rovnosti \eqref{eq:pocet-utvaru} dostaneme
\[d=\log_k{N(k)}=\dfrac{\ln{N(k)}}{\ln{k}}.\]
(Obecně lze volit jakýkoliv přípustný základ logaritmů, tj $d=\log_b{N(k)}/\log_b{k}$ pro $b\in\R_+\setminus\set{1}$.)

Dimenze v~tomto pojetí skutečně dává dobrý smysl. Pro "klasické" geometrické objekty vychází dimenze vždy celočíselně (viz tabulka \ref{table:eukleides-dimenze}).
\begin{table}[h]
    \centering
    \begin{tabular}{r|cc}
    Útvar    & $N(k)$ & $d=\ln{N(k)}/\ln{k}$ \\ \hline
    Úsečka   & $3$      & $1$                          \\
    Čtverec  & $9$      & $2$                          \\
    Krychle  & $27$     & $3$                          \\
    Teserakt & $81$     & $4$                          \\
    \end{tabular}
    \caption{Hodnoty dimenze $d$ pro různé útvary}
    \label{table:eukleides-dimenze}
\end{table}

Na této myšlence je založen pojem tzv. \emph{fraktální dimenze}\index{fraktální dimenze}\index{dimenze!fraktální}. Existuje více neekvivalentních způsobů její definice. Jedna z~nich, které se dále nyní v~této sekci budeme držet, se v~anglicky psané literatuře nazývá \emph{"box-counting dimension"}\index{box-counting dimenze}\footnote{Převzato z~\cite[str. 93]{Zelinka2006} a~\cite[str. 28]{Falconer2014}.}\index{dimenze!box-counting}, odkud plyne i~značení $\dimB$. Pro útvar $F$ (formálně vzato množinu bodů) definujeme 
\begin{equation}\label{eq:fraktalni-dimenze}
    \dimB{F}=\lim_{\varepsilon\to 0_+}{\dfrac{\ln{N_\varepsilon(F)}}{\ln{\left(\frac{1}{\varepsilon}\right)}}}.
\end{equation}
Výraz $1/\varepsilon$ zde představuje faktor podobnosti jako původní $k$ (samotné $\varepsilon$ tak hraje roli měřítka). Číslo $N_\varepsilon(F)$ je počet soběpodobných útvarů, na které jsme rozdělili původní útvar $F$ v měřítku $\varepsilon$. Největší rozdíl zde však představuje zkoumání "limitního chování" daného výrazu.

Nejdříve poznamenejme, že pro "klasické" geometrické útvary hodnota $\dimB{F}$ vychází skutečně tak, jak jsme zvyklí. Toto si ilustrujeme na příkladech~\ref{ex:fraktalni-dimenze-usecka}, \ref{ex:fraktalni-dimenze-ctverec} a~\ref{ex:fraktalni-dimenze-trojuhelnik}.
\begin{example}[Fraktální dimenze úsečky]\label{ex:fraktalni-dimenze-usecka}
    Začněme asi nejednodušším příkladem výpočtu fraktální dimenze, a to u~úsečky (označme $\ell$). Představme si, že úsečku \emph{jednotkové délky} rozdělíme na $N_\varepsilon(\ell)=n$ shodných dílů. Pak měřítko libovolného dílu je
    \[\varepsilon=\dfrac{1}{n}=n^{-1}.\]
    (Zde je dobré si uvědomit, že pro $n\to\infty$, tedy zjemňování dělení úsečky, platí, že $\varepsilon\to 0_+$.) Fraktální dimenzi úsečky vypočteme z~definice jako
    \[\dimB{\ell}=\lim_{\varepsilon\to 0_+}{\dfrac{\ln{N_\varepsilon(\ell)}}{\ln{\left(\frac{1}{\varepsilon}\right)}}}=\lim_{n\to\infty}{\dfrac{\ln{n}}{\ln{n}}}=1.\]
\end{example}
\begin{example}[Fraktální dimenze čtverce]\label{ex:fraktalni-dimenze-ctverec}
    Podobně jako v~příkladu~\ref{ex:fraktalni-dimenze-usecka}\linebreak{}výše můžeme stanovit i~fraktální dimenzi čtverce (označme $S$). Uvažujme tedy čtverec o~jednotkovém obsahu, který rozdělíme $N_\varepsilon(S)=n$ shodných útvarů. Přitom víme, že obsah mění kvadraticky vůči délce strany. Měřítko nového čtverce tak bude
    \[\varepsilon=\sqrt{\dfrac{1}{n}}=n^{-1/2}\]
    a~fraktální dimenze vychází
    \[\dimB{S}=\lim_{n\to\infty}{\dfrac{\ln{n}}{\ln{n^{1/2}}}}=\lim_{n\to\infty}{\dfrac{\ln{n}}{\frac{1}{2}\ln{n}}}=2.\]
\end{example}
Pro krychli bude výpočet naprosto analogický (viz příklad~\ref{ex:fraktalni-dimenze-ctverec}). Obecně pro $d$-rozměrnou krychli bude její fraktální dimenze\footnote{Obdobnou úvahou dojdeme k~měřítku $\varepsilon=n^{-1/d}$.} rovna $d$.\par
Zkusme se nyní oprostit od krychle k~dalšímu útvaru.
\begin{example}[Fraktální dimenze trojúhelníka]\label{ex:fraktalni-dimenze-trojuhelnik}
    Podívejme se, jak to dopadne s~fraktální dimenzí \emph{obecného trojúhelníku}. Každý trojúhelník $T$ lze rozdělit na čtveřici \emph{vzájemně shodných trojúhelníků $T_1,\dots,T_4$}, které vzniknou sestrojením středních příček v~původním trojúhelníka (viz obrázek~\ref{fig:trojuhelnik-sobepodobnost}).
    \begin{figure}[h]
        \centering
        \includegraphics[scale=\normalipe]{ch01-trojuhelnik-sobepodobnost.pdf}
        \caption{Trojúhelník $T$ rozdělený na trojúhelníky $T_1,\dots,T_4$}
        \label{fig:trojuhelnik-sobepodobnost}
    \end{figure}
    Délka každé střední příčky odpovídá polovině délky strany, s~níž je rovnoběžná, tedy obsah každého z~menších trojúhelníků je \emph{čtvrtina obsahu původního} trojúhelníka $T$. Tento postup můžeme opakovat pro každý z menších trojúhelníků, čímž dostaneme $4^2$ trojúhelníků. Takto můžeme postupovat libovolně dlouho, přičemž po $n$ krocích bude počet\footnote{Výpočet bychom mohli i~zde provést ve stejném duchu jako u~úsečky, čtverce nebo krychle. Počet částí, na něž rozdělíme trojúhelník, označíme $N_\varepsilon(T)=n$, přičemž měřítko pak bude $\varepsilon=n^{-1/2}$.} vzniklých trojúhelníků $N_\varepsilon(T)=4^n$ a~měřítko každého z~nich bude $\varepsilon=(1/2)^n$. Fraktální dimenze tak vychází:
    \[\dimB{T}=\lim_{n\to\infty}{\dfrac{\ln{4^n}}{\ln{2^n}}}=\lim_{n\to\infty}{\dfrac{2n\ln{2}}{n\ln{2}}}=2.\]
\end{example}
Pro jednoduché útvary vychází dimenze tak, jak bychom mohli očekávat. K~zajímavějším výsledkům však dospějeme u~fraktálů, na něž se blíže podíváme v~následující podsekci~\ref{subsec:dimenze-fraktalu}.

\subsection{Dimenze fraktálů}\label{subsec:dimenze-fraktalu}

Co kdybychom však zkusili podobnou myšlenku aplikovat i~na \emph{fraktální objekty} jako např. Kochovu křivku, Kochovu vločku, Sierpińského trojúhelník nebo Cantorovo diskontinuum? Zkusme to. Pro připomenutí jednotlivých křivek a~výsledků k~nim si dovolujeme čtenáře opětovně odkázat na sekci~\ref{sec:sobepodobnost}, kde jsou podrobněji rozebrány.

V tomto případě uvidíme, že dochází na první pohled k~docela zvláštnímu jevu. Fraktální dimenze již totiž nemusí vycházet celočíselně, jak jsme zvyklí.
\begin{itemize}
    \item \textbf{Kochova křivka $F_{KC}$.} V~každé iteraci nahrazujeme každou úsečku čtyřmi novými. Kompletní Kochova křivka tak obsahuje právě \emph{čtyři} kopie sebe sama zmenšené na třetinu, tj. v~$n$-té iteraci je $N_\varepsilon(F_{KC})=4^n$, jak jsme již odvodili (viz podsekce~\ref{subsec:kochova_krivka}).\footnote{Lze však zvolit i~jiné dělení. Např. lze na Kochovu křivku nahlížet tak, že obsahuje \emph{16 kopií} sebe sama zmenšených na \emph{devítinu}.} Měřítko nově vzniklých částí je tak $\varepsilon=(1/3)^n$.
    \begin{equation}\label{eq:kochova-krivka-dimenze}
        \dimB{F_{KC}}=\lim_{\varepsilon\to 0_+}{\dfrac{\ln{N_\varepsilon(F_{KC})}}{\ln{\left(\frac{1}{\varepsilon}\right)}}}=\lim_{n\to\infty}{\dfrac{\ln{4^n}}{\ln{3^n}}}=\dfrac{\ln{4}}{\ln{3}}\approx 1{,}2618595\dots
    \end{equation}
    \item \textbf{Kochova vločka $F_{KS}$.} Začínáme s~rovnostranným trojúhelníkem o~straně délky $1$, na jehož stranách postupně vznikne Kochova křivka. V~$n$-té iteraci je obvod Kochovy vločky $o_n$ roven $3\cdot 4^n$, tj. i~$N_\varepsilon(F_{KS})=3\cdot 4^n$, kde měřítko\footnote{Měřítko se ve srovnání s~Kochovou křivkou liší v~mocnině, neboť délku nových úseků porovnáváme s~obvodem celého trojúhelníka, nikoliv pouze délkou jedné jeho strany. Nicméně ve výpočtu bychom se mohli omezit i~jen na jednu ze stran, výpočet by byl tak zcela identický jako u~Kochovy křivky.} nově vzniklých úseček je $\varepsilon=1/3\cdot(1/3)^n=(1/3)^{n+1}$. Není těžké se přesvědčit, že fraktální dimenze vychází stejně jako u~Kochovy křivky:
    \begin{equation}\label{eq:kochova-vlocka-dimenze}
        \dimB{F_{KS}}=\lim_{n\to\infty}{\dfrac{\ln{3\cdot 4^n}}{\ln{3^{n+1}}}}=\lim_{n\to\infty}{\dfrac{\ln{4^n}\overbrace{\left(1+\frac{\ln{3}}{\ln{4^n}}\right)}^{\to 1}}{\ln{3^n}\underbrace{\left(1+\frac{\ln{3}}{\ln{3^n}}\right)}_{\to 1}}}=\dfrac{\ln{4}}{\ln{3}}.
    \end{equation}
    \item \textbf{Sierpińského trojúhelník $F_{ST}$.} V~každé iteraci vynecháme prostřední trojúhelník, čímž vznikne \emph{trojice} nových trojúhelníků s~\emph{polovičním} měřítkem. Tzn.~$N_\varepsilon(F_{ST})=3^n$ pro $\varepsilon=(1/2)^n$, a tedy
    \begin{equation}\label{eq:sierpinskeho-trojuhelnik-dimenze}
        \dimB{F_{ST}}=\lim_{n\to\infty}{\dfrac{\ln{3^n}}{\ln{2^{n}}}}=\dfrac{\ln{3}}{\ln{2}}\approx 1{,}5849625\dots
    \end{equation}
    \item \textbf{Cantorovo diskontinuum $F_{CD}$.} Vždy vyjmeme prostřední třetinu\linebreak{}úsečky, čímž obdržíme \emph{dvojici} úseček \emph{třetinové} délky, tj. $N_\varepsilon(F_{CD})=2^n$ pro $\varepsilon=(1/3)^n$. Fraktální dimenze tak vychází
    \begin{equation}\label{eq:cantorovo-diskontinuum-dimenze}
        \dimB{F_{CD}}=\lim_{n\to\infty}{\dfrac{\ln{2^n}}{\ln{3^n}}}=\dfrac{\ln{2}}{\ln{3}}\approx 0{,}6309297\dots
    \end{equation}
\end{itemize}
Udělejme si nyní menší souhrn a~porovnání dosud získaných výsledků (viz tabulka~\ref{table:fraktaly-eukleides-dimenze}).
\begin{table}[h]
    \centering
    \begin{tabular}{r|ccc}
        Útvar $F$                & $\varepsilon$ & $N_\varepsilon(F)$ & $\dimB{F}$         \\ \hline
        Úsečka                   & $n^{-1}$      & $n$                & 1                  \\
        Čtverec                  & $n^{-1/2}$    & $n$                & 2                  \\
        Krychle                  & $n^{-1/3}$    & $n$                & 3                  \\
        Teserakt                 & $n^{-1/4}$    & $n$                & 4                  \\
        $d$-rozměrná krychle     & $n^{-1/d}$    & $n$                & $d$                \\
        Obecný trojúhelník       & $(1/2)^n$     & $4^n$              & 2                  \\
        Kochova křivka           & $(1/3)^n$     & $4^n$              & $1{,}2618595\dots$ \\
        Kochova vločka           & $(1/3)^{n+1}$ & $3\cdot 4^n$       & $1{,}2618595\dots$ \\
        Sierpińského trojúhelník & $(1/2)^n$     & $3^n$              & $1{,}5849625\dots$ \\
        Cantorovo diskontinuum   & $(1/3)^n$     & $2^n$              & $0{,}6309297\dots$ \\
    \end{tabular}
    \caption{Porovnání fraktálních dimenzí $d_k$ různých objektů}
    \label{table:fraktaly-eukleides-dimenze}
\end{table}
Můžeme si všimnout, že zatímco u~"klasických" objektů vychází fraktální dimenze \emph{celočíselná}, u~(zmíněných) fraktálů vychází \emph{neceločíselně}, ba dokonce i~iracionálně.

\subsection{Topologická dimenze}\label{subsec:topologicka-dimenze}

Výsledky z~předešlé části~\ref{subsec:dimenze-fraktalu} se můžou zdát poněkud překvapující. Jak je vůbec možné, že dimenze nemusí vycházet nutně celočíselná? Ač se to možná zdá jako nesmyslný výsledek, je třeba si uvědomit, jak vlastně koncept dimenze chápeme. Na jednu stranu na ni lze nahlížet jako na mocninu "s níž se zvyšuje" obsah/objem tělesa, měníme-li měřítko. Naopak čtenář znalý lineární algebry si možná vzpomene, že v~této matematické disciplíně se na dimenzi nahlíží jako na \emph{mohutnost libovolné báze daného vektorového prostoru}. Ta naopak vychází vždy pouze celočíselně, případně nekonečná, avšak nelze s~ní dobře zachytit hlubší detail geometrie u~objektů, jako jsou právě fraktály.

Další možné pojetí pojmu dimenze nám dává tzv. \emph{topologická dimenze}\index{dimenze!topologická}\index{topologická dimenze}. Ta totiž daleko více odpovídají našemu intuitivnímu chápání tohoto pojmu, neboť se vždy jedná o~celé číslo, jak ho známe ze školní geometrie. Existuje více topologických dimenzí\footnote{Jiným příkladem takové dimenze je \emph{induktivní dimenze}.}, které co do definice nejsou ekvivalentní, ačkoliv ve většině standardních případů splývají. My se zde pro ilustraci podíváme na tzv. \emph{Lebesgueovu pokrývací dimenzi}\index{dimenze!Lebesgueova pokrývací}\index{Lebesgueova pokrývací dimenze} (dále jen již "topologickou dimenzi") pojmenovanou po francouzském matematikovi \name{Henri Lebesgueovi}\footnote{1875--1941}\index{Henri Lebesgue}. Myšlenka definice je založena na pokrývání objektu (formálně vzato \emph{množiny bodů}) tzv. \emph{otevřenými množinami}\index{množina!otevřená}\index{otevřená množina}.\footnote{Otevřená množina je zobecnění pojmu otevřeného intervalu reálných čísel. Neformálně řečeno je to taková množina $X$, kde pro každý její bod $x\in X$ patří do této množiny i~nějaké $\varepsilon$-okolí tohoto bodu (patří do ní i~body, které jsou "dostatečně blízko").} Formální definici si zde v~rámci zachování jednoduchosti odpustíme, avšak pro hlubší matematický základ si dovolíme čtenáře odkázat např. na knihu \cite{Engelking1989}. Avšak pokusíme se ji alespoň nastínit.

Množina $X$ má topologickou dimenzi $\dimL{X}=n$, pokud $n$ je nejmenší číslo takové, že pro každé pokrytí otevřenými množinami\footnote{Formálněji to znamená, že $A_1,\dots,A_n$ jsou otevřené množiny takové, že platí $X\subseteq\bigcup_{i=1}^n{A_i}$.} $\mathcal{U}$ existuje zjemnění\footnote{Zjemněním pokrytí $\mathcal{U}$ nazýváme takové pokrytí $\mathcal{U}^\prime$ množiny $X$, kde každá množina $A_i^\prime\in\mathcal{U}^\prime$ je \emph{podmnožinou} nějaké množiny $A_j$ původního pokrytí $\mathcal{U}$.}\index{zjemnění} $\mathcal{U}^\prime$ takové, že každý bod $x\in X$ náleží nejvýše $n+1$ množin pokrytí $\mathcal{U}$.

Tuto ideu si zkusíme přiblížit na příkladu topologické dimenze úsečky (viz obrázek~\ref{fig:usecka-zjemneni}).
\begin{figure}[h]
    \centering
    \includegraphics[scale=\normalipe]{ch01-usecka-pokryti-1.pdf}\\\qquad\\
    \includegraphics[scale=\normalipe]{ch01-usecka-pokryti-2.pdf}\\\qquad\\
    \includegraphics[scale=\normalipe]{ch01-usecka-pokryti-3.pdf}\\\qquad\\
    \caption{Různé možnosti (pod)pokrytí úsečky}
    \label{fig:usecka-zjemneni}
\end{figure}
Pro libovolné pokrytí lze ukázat, že každý bod je obsažen maximálně ve \emph{dvou množinách} vhodně zvoleného zjemnění, tedy topologická dimenze úsečky je $1$. Podobně např. pro čtverec lze dojít k~závěru, že pro každé pokrytí existuje zjemnění takové, že každý bod je obsažen maximálně ve \emph{třech množinách}, tedy jeho topologická dimenze je 2, jak bychom očekávali (viz obrázek~\ref{fig:ctverec-zjemneni}).
\begin{figure}[h]
    \centering
    \includegraphics[scale=\normalipe]{ch01-ctverec-pokryti.pdf}
    \caption{Možné (pod)pokrytí čtverce}
    \label{fig:ctverec-zjemneni}
\end{figure}
Porovnejme nyní topologickou dimenzi vůči dimenzi fraktální. Jak jsme se již přesvědčili v~příkladech~\ref{ex:fraktalni-dimenze-usecka}, \ref{ex:fraktalni-dimenze-ctverec} a~\ref{ex:fraktalni-dimenze-trojuhelnik}, pro "standardní" útvary je fraktální dimenze celočíselná (ač jsou i~další, které jsme neuvedli), zatímco v~podsekci~\ref{subsec:dimenze-fraktalu} jsme zjistili, že u~fraktální dimenze fraktálů tomu tak být nemusí. Přitom však topologická dimenze fraktálních útvarů je (a dokonce musí být) celočíselná (viz tabulka~\ref{table:fraktalni-topologicka-dimenze}).
\begin{table}[h]
    \centering
    \begin{tabular}{r|cc}
    Útvar $F$                & $\dimB{F}$            & $\dimL{F}$ \\\hline
    Úsečka                   & 1                     & 1          \\
    Čtverec                  & 2                     & 2          \\
    Krychle                  & 3                     & 3          \\
    Teserakt                 & 4                     & 4          \\
    $d$-rozměrná krychle     & $d$                   & $d$        \\
    Kochova křivka           & $1{,}2618595\dots$    & 1          \\
    Kochova vločka           & $1{,}2618595\dots$    & 1          \\
    Sierpińského trojúhelník & $1{,}5849625\dots$    & 2          \\
    Cantorovo diskontinuum   & $0{,}6309297\dots$    & 0      
    \end{tabular}
    \caption{Porovnání fraktální a~topologické dimenze útvarů}
    \label{table:fraktalni-topologicka-dimenze}
\end{table}
Fraktální dimenze tak oproti té topologické daleko lépe zachycuje informaci o~detailní geometrii daných objektů.
\section{Co je to fraktál?}\label{sec:co-je-to-fraktal}
\emph{Tak co je to tedy ten "fraktál"?}\index{fraktál} Odpovědi na tuto otázku jsme se poměrně dlouhou dobu vyhýbali a~onen termín, popř. jeho přídavnou variantu \emph{"fraktální"}, jsme používali čistě na intuitivní úrovni. Ač jsme se zatím obešli bez jeho formálnějšího upřesnění, bylo by možná přinejmenším slušné se o~to alespoň pokusit. V~předešlé sekci~\ref{sec:fraktalni_dimenze} jsme si (alespoň částečně) ujasnili \emph{fraktální} a~\emph{topologickou dimenzi}, které jsme následně použili na příkladech konkrétních útvarů (konkrétně viz tabulky~\ref{table:fraktaly-eukleides-dimenze} a~\ref{table:fraktalni-topologicka-dimenze}).

Již jsme si všimli, že u~fraktálních útvarů vychází fraktální dimenze $\dimH$ neceločíselně, zatímco jejich topologická dimenze $\dimL$ je vždy celočíselná. To by se mohlo zdát jako dobrá charakteristika fraktálů. Avšak existují útvary, jejichž fraktální a~topologická dimenze se shodují, přestože také mají "fraktální charakter". Pro příklad nemusíme chodit nikterak daleko, pravděpodobně nejznámějším útvarem je v~tomto ohledu \emph{Mandelbrotova množina}\index{množina!Mandelbrotova}\index{Mandelbrotova mnozina}, jejíž fraktální i~topologická dimenze je rovna $2$ (blíže se na ni podíváme v~podsekci~\ref{subsec:mandebrotova-mnozina}). Jiná definice zase naopak popisuje fraktál jako útvar, jehož Hausdorffova dimenze (na tu se blíže podíváme v~sekci~\ref{sec:hausdorffova-mira-dimenze}) je ostře větší než dimenze topologická. To bychom však ale nemohli považovat za fraktál např. již zmíněný Sierpińského trojúhelník. Problém (a také důvod, proč jsme se definici toho pojmu vyhýbali) je však zkrátka ten, že dodnes \textbf{není známá} žádná univerzální definice fraktálu. \cite[str. 226]{Voracova2022}

Je to možná trochu zklamání, nicméně dobrou zprávou je, že ani pro další výklad ona absence formální definice fraktálu nebude překážkou. V~dalším textu se zaměříme (mj.) především na klasifikaci fraktálů (viz kapitola~\ref{chapter:klasifikace-fraktalu}) a~další jejich vlastnosti.