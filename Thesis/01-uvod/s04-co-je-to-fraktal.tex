\section{Co je to fraktál?}\label{sec:co-je-to-fraktal}
\emph{Tak co je to tedy ten "fraktál"?}\index{fraktál} Odpovědi na tuto otázku jsme se poměrně dlouhou dobu vyhýbali a~onen termín, popř. jeho přídavnou variantu \emph{"fraktální"}, jsme používali čistě na intuitivní úrovni. Ač jsme se zatím obešli bez jeho formálnějšího upřesnění, bylo by možná přinejmenším slušné se o~to alespoň pokusit. V~předešlé sekci~\ref{sec:fraktalni_dimenze} jsme si (alespoň částečně) ujasnili \emph{fraktální} a~\emph{topologickou dimenzi}, které jsme následně použili na příkladech konkrétních útvarů (konkrétně viz tabulky~\ref{table:fraktaly-eukleides-dimenze} a~\ref{table:fraktalni-topologicka-dimenze}).

Již jsme si všimli, že u~fraktálních útvarů vychází fraktální dimenze $\dimH$ neceločíselně, zatímco jejich topologická dimenze $\dimL$ je vždy celočíselná. To by se mohlo zdát jako dobrá charakteristika fraktálů. Avšak existují útvary, jejichž fraktální a~topologická dimenze se shodují, přestože také mají "fraktální charakter". Pro příklad nemusíme chodit nikterak daleko, pravděpodobně nejznámějším útvarem je v~tomto ohledu \emph{Mandelbrotova množina}\index{množina!Mandelbrotova}\index{Mandelbrotova mnozina}, jejíž fraktální i~topologická dimenze je rovna $2$ (blíže se na ni podíváme v~podsekci~\ref{subsec:mandebrotova-mnozina}). Jiná definice zase naopak popisuje fraktál jako útvar, jehož Hausdorffova dimenze (na tu se blíže podíváme v~sekci~\ref{sec:hausdorffova-mira-dimenze}) je ostře větší než dimenze topologická. To bychom však ale nemohli považovat za fraktál např. již zmíněný Sierpińského trojúhelník. Problém (a také důvod, proč jsme se definici toho pojmu vyhýbali) je však zkrátka ten, že dodnes \textbf{není známá} žádná univerzální definice fraktálu. \cite[str. 226]{Voracova2022}

Je to možná trochu zklamání, nicméně dobrou zprávou je, že ani pro další výklad ona absence formální definice fraktálu nebude překážkou. V~dalším textu se zaměříme (mj.) především na klasifikaci fraktálů (viz kapitola~\ref{chapter:klasifikace-fraktalu}) a~další jejich vlastnosti.