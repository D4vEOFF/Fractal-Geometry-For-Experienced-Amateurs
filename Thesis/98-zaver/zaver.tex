\chapwithtoc{Závěrem}

Tento text celkově shrnuje některé základní poznatky z~oblasti fraktální geometrie\index{fraktální geometrie}\index{geometrie!fraktální}. Za zmínku stojí především koncepty soběpodobnosti\index{soběpodobnost} a~fraktální dimenze\index{dimenze!fraktální}\index{fraktální dimenze}, základy teorie míry\index{teorie míry}, L-systémy\index{L-systém}, systémy iterovaných funkcí (IFS)\index{systém iterovaných funkcí}\index{IFS} a~Time Escape algoritmy\index{algoritmus!Time Escape}\index{Time Escape algoritmus}, především se zaměřením na Juliovy\index{Juliova množina}\index{množina!Juliova} množiny a~jako poslední programová ukázka implementace některých algoritmů\index{algoritmus} pro generování zmíněných typů fraktálů.

Text byl psán s~cílem, aby se v~něm orientovali jak čtenáři bez hlubšího matematického vzdělání -- hlavně díky intuitivněji zaměřenému vysvětlení v~kapitolách~\ref{chapter:uvod_do_fraktalu}, \ref{chapter:klasifikace-fraktalu} a~\ref{chapter:generovani-fraktalu}, tak i~pokročilejší zájemci, kteří ocení matematicky náročnější obsah v~kapitole~\ref{chapter:teorie-miry-a-dimenze} věnované teorii míry a~dimenzím. V~kapitole~\ref{chapter:generovani-fraktalu} byla vyvinuta snaha, co nejvíce vyjít vstříc širokému okruhu programátorů, a~to důrazem na obecnost prezentovaných algoritmů. Tím je eliminována nutnost přímé znalosti jazyka Python, a~čtenář si i~tak může odnést něco užitečného pro vlastní projekty.

Velkou roli v~práci hrají ilustrace, z nichž většina byla vytvořena ve vektorovém programu Ipe nebo pomocí programu na generování fraktálů, který je přiložen k~práci. Lze jej nalézt na portálu GitHub\footnote{viz odkaz \url{https://github.com/D4vEOFF/Py-Fractal-Generator}}, kde je možné si přímo vyzkoušet generování fraktálních obrazců a~seznámit se s~praktickou implementací algoritmů uvedených v~kapitole~\ref{chapter:generovani-fraktalu}.

Text tedy poskytuje přehled hlavně teoretických aspektů fraktálů a~jejich popisu. Je dobré podotknout, že matematická rozmanitost této problamatiky je opravdu široká a~většinu teorie tak ani nebylo možné zde ve smysluplné míře obsáhnout\footnote{Pro hlubší studium se čtenář může podívat do seznamu citované literatury}. Tím spíš by se však slušelo přinejmenším zmínit její praktický aspekt. Fraktály se v~našem světě vyskytují až překvapivě často a~jsou všude okolo nás. Ukazuje se, že jejich praktické využití je velmi rozsáhlé. Například v~oblasti kódování dat se fraktální metody uplatňují při efektivní kompresi obrázků, neboť dokáží jednoduchým způsobem popsat složité struktury. Ve sféře neuronových sítí se fraktální principy využívají jak v~návrhu specifických architektur, tak při tvorbě optimalizačních algoritmů. V~počítačové grafice jsou fraktály používány pro realistickou simulaci přírodních jevů, jako jsou například horské útvary, mraky nebo složité povrchy. V~tomto ohledu si ještě naposledy dovolíme čtenáře odkázat na další literaturu, konkrétně knihu \cite{Zelinka2006}, která je přednostně věnována využití této oblasti geometrie.

Fraktály přitom nejsou pouze matematickými abstrakcemi -- jejich struktury jsou hojně pozorovatelné i~v přírodě, např. ve větvení stromů, ve struktuře rostlin, ve formacích horských hřebenů, v říčních sítích nebo v~dynamice proudění kapalin. Tato všudypřítomnost fraktálních vzorů v~přírodě potvrzuje hluboké propojení matematiky se světem kolem nás, což činí fraktální geometrii nejen atraktivní, ale také vysoce relevantní oblastí výzkumu a~poznání.