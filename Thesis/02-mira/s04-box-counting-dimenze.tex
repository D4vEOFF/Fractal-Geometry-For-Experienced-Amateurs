\section{Box-counting dimenze}\label{sec:box-counting-dimenze}

Tomuto typu dimenze jsme se již v základu věnovali v kapitole \ref{chapter:uvod_do_fraktalu}, specificky sekci \ref{sec:fraktalni_dimenze}, kde jsme rozebrali jeho způsob jejího výpočtu a ukázali jsme si jej několika příkladech. V této části si blíže rozebereme některé další vlastnosti týkající se právě \emph{box-counting dimenze}\index{box-counting dimenze}\footnote{V kapitole \ref{chapter:uvod_do_fraktalu} jsme pro jednoduchost používali obecnější termín \emph{fraktální dimenze}. Ten však zahrnuje daleko širší škálu možných definic, než jen tu, kterou jsme si představili. Avšak dále v tomto textu budeme používat výhradně její skutečný název, tj. box-counting dimenze.} a pokusíme se ji lépe zasadit do kontextu teorie míry, které jsme se samostatně až do této chvíle věnovali.

Jako první se podíváme na myšlenku box-counting trochu blíže a maličko si ji zobecníme. Původně jsme nahlíželi na dimenzi jako na exponent, s nímž roste "velikost" zkoumaného útvaru. Tato myšlenka nás se ukázala jako rozumná, neboť pro "klasické" geometrické útvary vycházela tato dimenze vždy celočíselně, nicméně už tomu tak nebylo v případě fraktálních útvarů. Myšlenka byla taková, že jsme útvar $F$ rozdělili na určitý počet stejně "velkých částí", označme $F_1,F_2,\ldots,F_m$, v nějakém měřítku $\varepsilon>0$. Zkusme nyní požadavek na striktně stejnou velikost (formálně vzato míru) trochu rozvolnit. Bude nám stačit, když pro každé $i$ je $\diam{F_i}\leqslant\delta$, kde $\delta>0$. Zároveň nebudeme požadovat, aby množiny $F_1,F_2,\ldots,F_m$ byly všechny striktně po dvou téměř disjunktními\footnote{Množiny $M,N$ jsou \emph{téměř disjunktní}, pokud $\interior{M}\cap\interior{N}=\emptyset$, tedy může nastat, že se na hranici mohou "dotýkat", tzn. $\boundary{M}\cap\boundary{N}\neq\emptyset$.} podmnožinami $F$, ale stačí, když budou tvořit pokrytí $F$.

Mějme tedy nějakou neprázdnoou omezenou množinu $F\subset\R^n$, kde pro každé $\delta>0$ budeme hledat \emph{nejmenší počet} množin, takových, že pokrývají $F$. Toto číslo si označíme $N_\delta(F)$. Dimenze množiny $F$ by tedy měla odrážet "rychlost" růstu $N_\delta(F)$ pro $\delta\to 0^+$. Je-li splněna aproximace
\begin{equation}\label{eq:odhad-n-delta}
    N_\delta(F)\approx c\delta^{-s}
\end{equation}
pro $c>0$, pak řekneme, že množina $F$ má box-counting dimenzi $s$. (Převzato z \citep[str. 27]{Falconer2014}.)
\begin{remark}
    V dalším textu budeme místo $\delta\to 0^+$ psát pro jednoduchost pouze $\delta\to 0$, byť by se slušilo používat první variantu. Čtenáři je však nejspíše jasné, že uvažovat záporný průměr množiny nemá smysl.
\end{remark}
Logaritmováním a úpravou výrazu \eqref{eq:odhad-n-delta} dostaneme:
\begin{align}\label{eq:odvozeni-box-counting-dimenze}
    \ln{N_\delta(F)}&\approx\ln{c}+\ln{\delta^{-s}}\\
    \ln{N_\delta(F)}&\approx\ln{c}-s\ln{\delta}\\
    s&\approx\dfrac{\ln{N_\delta(F)}}{-\ln{\delta}}+\dfrac{\ln{c}}{\ln{\delta}}.
\end{align}
Když porovnáme výsledek v \eqref{eq:odvozeni-box-counting-dimenze} s rovností \eqref{eq:fraktalni-dimenze} z minulé kapitoly, můžeme si všimnout, že zde navíc figuruje člen $\ln{c}/\ln{\delta}$. Když však uvážíme limitu daného výrazu pro $\delta\to 0$, dostaneme původní vzorec, který jsme již viděli, tj.
\[\lim_{\delta\to 0}\left(\dfrac{\ln{N_\delta(F)}}{-\ln{\delta}}+\dfrac{\ln{c}}{\ln{\delta}}\right)=\lim_{\delta\to 0}\dfrac{\ln{N_\delta(F)}}{-\ln{\delta}}+\lim_{\delta\to 0}\dfrac{\ln{c}}{\ln{\delta}}=\lim_{\delta\to 0}\dfrac{\ln{N_\delta(F)}}{-\ln{\delta}}.\]
% \begin{definition}[$\delta$-pokrytí]\label{def:delta-pokryti}
%     Nechť $(X,\rho)$ je metrický prostor, $F\subseteq X$ a $\delta>0$. Jestliže $\mathcal{F}=\set{F_1,F_2,\ldots}\subseteq\powset{X}$ je pokrytí $F$ a zároveň $\diam{F_j}\leqslant\delta$ pro každé $j$, pak $\mathcal{F}$ nazýváme $\delta$-pokrytí\index{$\delta$-pokrytí} množiny $F$.
% \end{definition}
Předchozí úvahu můžeme shrnout do následující definice.
\begin{definition}[Box-counting dimenze]\label{def:box-counting-dimenze}
    Nechť $F\subset \R^n$ je neprázdná omezená množina. Pak definujeme následující:
    \begin{itemize}
        \item \emph{Nejmenší počet množin v $\delta$-pokrytí množiny $F$} značíme $N_\delta(F)$, tj.
        \[N_\delta(F)=\inf\set{m\;\middle|\;F\subseteq\bigcup_{i=1}^n F_i\;,\;\diam{F_j}\leqslant\delta\;\text{pro}\;1\leqslant j\leqslant m}.\]
        \item \emph{Horní box-counting dimenze} množiny $F$ je
        \[\upperdimB{F}=\limsup_{\delta\to 0}\dfrac{\ln{N_\delta(F)}}{-\ln{\delta}}.\]
        \item \emph{Dolní box-counting dimenze} množiny $F$ je
        \[\lowerdimB{F}=\liminf_{\delta\to 0}\dfrac{\ln{N_\delta(F)}}{-\ln{\delta}}.\]
    \end{itemize}
    V případě, že $\lowerdimB{F}=\upperdimB{F}$, pak společnou hodnotu nazýváme \emph{box-counting dimenzí} množiny $F$, značíme $\dimB{F}$, přičemž platí
    \[\dimB{F}=\lim_{\delta\to 0}\dfrac{\ln{N_\delta(F)}}{-\ln{\delta}}.\]
\end{definition}
\begin{remark}
    Zde je důležité zmínit, že pro v dalším textu budeme uvažovat $\delta$ dostatečně malé, takové, že hodnota $-\ln{\delta}$ je vždy kladná. Dále též budeme pracovat (podle definice \ref{def:box-counting-dimenze}) pouze s neprázdnými omezenými množinami, abychom se vyhnuli problémům s případy, kdy $N_\delta(F)=\infty$ nebo $N_\delta(F)=0$.
\end{remark}
Abychom uvedli vše na pravou míru, box-counting dimenzi lze taktéž definovat více způsoby. V tuto uvažujeme obecně $\delta$-pokrytí dané množiny $F$, tj. pokrytí \emph{obecnými} množinami o průměru maximálně $\delta>0$. Lze se však zaměřit i na konkrétní útvary, jak ukazuje následující věta.
\begin{theorem}[Ekvivalentní definice box-counting dimenze]\label{thm:ekvivalentni-def-box-counting-dimenze}
    Nechť $F\subset \R^n$ je neprázdná omezená množina. Pak
    \begin{align*}
        \lowerdimB{F}&=\liminf_{\delta\to 0}\dfrac{\ln{M_\delta(F)}}{-\ln{\delta}},\\
        \upperdimB{F}&=\limsup_{\delta\to 0}\dfrac{\ln{M_\delta(F)}}{-\ln{\delta}},\\
        \dimB{F}&=\lim_{\delta\to 0}\dfrac{\ln{M_\delta(F)}}{-\ln{\delta}},
    \end{align*}
    kde pro $M_\delta(F)$ platí
    \begin{enumerate}[label=(\roman*)]
        \item\label{thm:pokryti-delta-uz-koulemi} $\displaystyle M_\delta(F)=\inf\set{m\;\middle|\;F\subseteq\bigcup\limits_{i=1}^m K_\delta(x_i)\;,\;x_j\in\R^n\;\text{pro}\;1\leqslant j\leqslant m}$,
        \item\label{thm:pokryti-delta-kvadry} $\displaystyle M_\delta(F)=\inf\set{m\;\middle|\;F\subseteq\bigcup\limits_{i=1}^m I_i\;,\;I_j=\prod_{k=1}^{n}\langle a_k,a_k+\delta\rangle\;\text{pro}\;1\leqslant j\leqslant m}$,
        \item\label{thm:pokryti-delta-sit} $\displaystyle M_\delta(F)=\left|\set{I\;\middle|\;I\cap F\neq\emptyset\;,\;I\in\mathcal{D}}\right|$, kde $\mathcal{D}$ je $\delta$-síť.
        \item\label{thm:pokryti-delta-dis-ot-koulemi} $\displaystyle M_\delta(F)=\sup\set{m\;\middle|\;B_\delta(x_i)\cap B_\delta(x_j)=\emptyset\;;\;x_i,x_j\in\R^n\;\text{pro}\;1\leqslant i,j\leqslant m}$.
    \end{enumerate}
\end{theorem}
(Převzato z \citep[str. 30]{Falconer2014}.)

Pojďme si větu \ref{thm:ekvivalentni-def-box-counting-dimenze} nyní trochu rozebrat.
\begin{itemize}
    \item Body \ref{thm:pokryti-delta-uz-koulemi} a \ref{thm:pokryti-delta-dis-ot-koulemi} říkájí, že $N_\delta(F)$ je rovno \emph{nejmenšímu počtu uzavřených koulí o poloměru $\delta$, které pokrývají $F$}, resp. \emph{nejvyšší počet disjunktních otevřených koulí, které mají střed v $F$}.
    \item Podobně body \ref{thm:pokryti-delta-kvadry} a \ref{thm:pokryti-delta-sit} tvrdí, že $N_\delta(F)$ lze ekvivalentně definovat jako pokrytí kvádry o stranách délky $\delta$, resp. počet všech kvádrů z $\delta$-sítě, které mají s $F$ neprázdný průnik.
\end{itemize}
\todo{Doplnit obrázek pro ilustraci věty}

Zároveň body \ref{thm:pokryti-delta-kvadry} a \ref{thm:pokryti-delta-sit} nám dávají dobré opodstatnění názvu tohoto typu dimenze, neboť zkoumáme pokrývání daného obrazce "kostkami". Při aproximacích box-counting dimenze obrazce $F\subset\R^2$ tak lze pracovat s mřížkou čtverců o libovolné straně $\delta>0$, kdy $N_\delta(F)$ stanovíme jako počet čtverců, které se překrývají se zkoumaným obrazcem $F$.

\todo{Doplnit obrázek z Wikipedie a text z Falconera str. 31.}