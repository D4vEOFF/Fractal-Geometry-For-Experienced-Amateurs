\section{Box-counting dimenze}\label{sec:box-counting-dimenze}

Tomuto typu dimenze jsme se již v~základu věnovali v~kapitole \ref{chapter:uvod_do_fraktalu},~specificky sekci \ref{sec:fraktalni_dimenze},~kde jsme rozebrali jeho způsob jejího výpočtu a ukázali jsme si jej několika příkladech. V~této části si blíže rozebereme některé další vlastnosti týkající se právě \emph{box-counting dimenze}\index{box-counting dimenze}\footnote{V kapitole \ref{chapter:uvod_do_fraktalu} jsme pro jednoduchost používali obecnější termín \emph{fraktální dimenze}. Ten však zahrnuje daleko širší škálu možných definic,~než jen tu,~kterou jsme si představili. Avšak dále v~tomto textu budeme používat výhradně její skutečný název,~tj. box-counting dimenze.} a pokusíme se ji lépe zasadit do kontextu teorie míry,~které jsme se samostatně až do této chvíle věnovali.

Jako první se podíváme na myšlenku box-counting trochu blíže a maličko si ji zobecníme. Původně jsme nahlíželi na dimenzi jako na exponent,~s nímž roste "velikost" zkoumaného útvaru. Tato myšlenka nás se ukázala jako rozumná,~neboť pro "klasické" geometrické útvary vycházela tato dimenze vždy celočíselně,~nicméně už tomu tak nebylo v~případě fraktálních útvarů. Myšlenka byla taková,~že jsme útvar $F$ rozdělili na určitý počet stejně "velkých částí",~označme $F_1,F_2,\ldots,F_m$,~v nějakém měřítku $\varepsilon>0$. Zkusme nyní požadavek na striktně stejnou velikost (formálně vzato míru) trochu rozvolnit. Bude nám stačit,~když pro každé $i$ je $\diam{F_i}\leqslant\delta$,~kde $\delta>0$. Zároveň nebudeme požadovat,~aby množiny\linebreak{}$F_1,F_2,\ldots,F_m$ byly všechny striktně po dvou téměř disjunktními\footnote{Množiny $M,N$ jsou \emph{téměř disjunktní},~pokud $\interior{M}\cap\interior{N}=\emptyset$,~tedy může nastat,~že se na hranici mohou "dotýkat",~tzn. $\boundary{M}\cap\boundary{N}\neq\emptyset$.} podmnožinami $F$,~ale stačí,~když budou tvořit pokrytí $F$.

Mějme tedy nějakou neprázdnoou omezenou množinu $F\subset\R^n$,~kde pro každé $\delta>0$ budeme hledat \emph{nejmenší počet} množin,~takových,~že pokrývají $F$. Toto číslo si označíme $N_\delta(F)$. Dimenze množiny $F$ by tedy měla odrážet "rychlost" růstu $N_\delta(F)$ pro $\delta\to 0^+$. Je-li splněna aproximace
\begin{equation}\label{eq:odhad-n-delta}
    N_\delta(F)\approx c\delta^{-s}
\end{equation}
pro $c>0$,~pak řekneme,~že množina $F$ má box-counting dimenzi $s$. (Převzato z \citep[str. 27]{Falconer2014}.)
\begin{remark}
    V~dalším textu budeme místo $\delta\to 0^+$ psát pro jednoduchost pouze $\delta\to 0$,~byť by se slušilo používat první variantu. Čtenáři je však nejspíše jasné,~že uvažovat záporný průměr množiny nemá smysl.
\end{remark}
Logaritmováním a úpravou výrazu \eqref{eq:odhad-n-delta} dostaneme:
\begin{align}\label{eq:odvozeni-box-counting-dimenze}
    \ln{N_\delta(F)}&\approx\ln{c}+\ln{\delta^{-s}}\\
    \ln{N_\delta(F)}&\approx\ln{c}-s\ln{\delta}\\
    s&\approx\dfrac{\ln{N_\delta(F)}}{-\ln{\delta}}+\dfrac{\ln{c}}{\ln{\delta}}.
\end{align}
Když porovnáme výsledek v~\eqref{eq:odvozeni-box-counting-dimenze} s~rovností \eqref{eq:fraktalni-dimenze} z minulé kapitoly,~můžeme si všimnout,~že zde navíc figuruje člen $\ln{c}/\ln{\delta}$. Když však uvážíme limitu daného výrazu pro $\delta\to 0$,~dostaneme původní vzorec,~který jsme již viděli,~tj.
\[\lim_{\delta\to 0}\left(\dfrac{\ln{N_\delta(F)}}{-\ln{\delta}}+\dfrac{\ln{c}}{\ln{\delta}}\right)=\lim_{\delta\to 0}\dfrac{\ln{N_\delta(F)}}{-\ln{\delta}}+\lim_{\delta\to 0}\dfrac{\ln{c}}{\ln{\delta}}=\lim_{\delta\to 0}\dfrac{\ln{N_\delta(F)}}{-\ln{\delta}}.\]
% \begin{definition}[$\delta$-pokrytí]\label{def:delta-pokryti}
%     Nechť $(X,\rho)$ je metrický prostor,~$F\subseteq X$ a $\delta>0$. Jestliže $\mathcal{F}=\set{F_1,F_2,\ldots}\subseteq\powset{X}$ je pokrytí $F$ a zároveň $\diam{F_j}\leqslant\delta$ pro každé $j$,~pak $\mathcal{F}$ nazýváme $\delta$-pokrytí\index{$\delta$-pokrytí} množiny $F$.
% \end{definition}
Předchozí úvahu můžeme shrnout do následující definice.
\begin{definition}[Box-counting dimenze]\label{def:box-counting-dimenze}
    Nechť $F\subset \R^n$ je neprázdná omezená množina. Pak definujeme následující:
    \begin{enumerate}[label=(\alph*)]
        \item \emph{Nejmenší počet množin v~$\delta$-pokrytí množiny $F$} značíme $N_\delta(F)$,~tj.
        \[N_\delta(F)=\inf\set{m\in\N_0\;\middle|\;F\subseteq\bigcup_{i=1}^n F_i\;,\;\diam{F_j}\leqslant\delta\;\text{pro}\;1\leqslant j\leqslant m}.\]
        \item \emph{Horní box-counting dimenze} množiny $F$ je
        \[\upperdimB{F}=\limsup_{\delta\to 0}\dfrac{\ln{N_\delta(F)}}{-\ln{\delta}}.\]
        \item \emph{Dolní box-counting dimenze} množiny $F$ je
        \[\lowerdimB{F}=\liminf_{\delta\to 0}\dfrac{\ln{N_\delta(F)}}{-\ln{\delta}}.\]
    \end{enumerate}
    V~případě,~že $\lowerdimB{F}=\upperdimB{F}$,~pak společnou hodnotu nazýváme \emph{box-counting dimenzí} množiny $F$,~značíme $\dimB{F}$,~přičemž platí
    \[\dimB{F}=\lim_{\delta\to 0}\dfrac{\ln{N_\delta(F)}}{-\ln{\delta}}.\]
\end{definition}
\begin{remark}
    Zde je důležité zmínit,~že pro v~dalším textu budeme uvažovat $\delta$ dostatečně malé,~takové,~že hodnota $-\ln{\delta}$ je vždy kladná. Dále též budeme pracovat (podle definice \ref{def:box-counting-dimenze}) pouze s~neprázdnými omezenými množinami,~abychom se vyhnuli problémům s~případy,~kdy $N_\delta(F)=\infty$ nebo $N_\delta(F)=0$.
\end{remark}
Abychom uvedli vše na pravou míru,~box-counting dimenzi lze taktéž definovat více způsoby. V~tuto uvažujeme obecně $\delta$-pokrytí dané množiny $F$,~tj. pokrytí \emph{obecnými} množinami o~průměru maximálně $\delta>0$. Lze se však zaměřit i~na konkrétní útvary,~jak ukazuje následující věta.
\begin{theorem}[Ekvivalentní definice box-counting dimenze]\label{thm:ekvivalentni-def-box-counting-dimenze}
    Nechť $F\subset \R^n$ je neprázdná omezená množina. Pak
    \begin{align*}
        \lowerdimB{F}&=\liminf_{\delta\to 0}\dfrac{\ln{M_\delta(F)}}{-\ln{\delta}},\\
        \upperdimB{F}&=\limsup_{\delta\to 0}\dfrac{\ln{M_\delta(F)}}{-\ln{\delta}},\\
        \dimB{F}&=\lim_{\delta\to 0}\dfrac{\ln{M_\delta(F)}}{-\ln{\delta}},
    \end{align*}
    kde pro $M_\delta(F)$ platí
    \begin{enumerate}[label=(\roman*)]
        \item\label{thm:pokryti-delta-uz-koulemi} $\displaystyle M_\delta(F)=\inf\set{m\;\middle|\;F\subseteq\bigcup\limits_{i=1}^m K_\delta(x_i)\;,\;x_j\in\R^n\;\text{pro}\;1\leqslant j\leqslant m}$,
        \item\label{thm:pokryti-delta-kvadry} $\displaystyle M_\delta(F)=\inf\set{m\;\middle|\;F\subseteq\bigcup\limits_{i=1}^m I_i\;,\;I_j=\prod_{k=1}^{n}\langle a_k,a_k+\delta\rangle\;\text{pro}\;1\leqslant j\leqslant m}$,
        \item\label{thm:pokryti-delta-sit} $\displaystyle M_\delta(F)=\left|\set{I\;\middle|\;I\cap F\neq\emptyset\;,\;I\in\mathcal{D}}\right|$,~kde $\mathcal{D}$ je $\delta$-síť.
        \item\label{thm:pokryti-delta-dis-ot-koulemi} $\displaystyle M_\delta(F)=\sup\set{m\;\middle|\;B_\delta(x_i)\cap B_\delta(x_j)=\emptyset\;;\;x_i,x_j\in\R^n\;\text{pro}\;1\leqslant i,j\leqslant m}$.
    \end{enumerate}
\end{theorem}

Pojďme si větu \ref{thm:ekvivalentni-def-box-counting-dimenze} nyní trochu rozebrat.
\begin{itemize}
    \item Body \ref{thm:pokryti-delta-uz-koulemi} a \ref{thm:pokryti-delta-dis-ot-koulemi} říkájí,~že $N_\delta(F)$ je rovno \emph{nejmenšímu počtu uzavřených koulí o~poloměru $\delta$,~které pokrývají $F$},~resp. \emph{nejvyšší počet disjunktních otevřených koulí,~které mají střed v~$F$}.
    \item Podobně body \ref{thm:pokryti-delta-kvadry} a \ref{thm:pokryti-delta-sit} tvrdí,~že $N_\delta(F)$ lze ekvivalentně definovat jako pokrytí kvádry o~stranách délky $\delta$,~resp. počet všech kvádrů z $\delta$-sítě,~které mají s~$F$ neprázdný průnik.
\end{itemize}
Pro představu viz obrázek \ref{fig:ilustrace-definic-bc-dimenze}. Důkaz věty je delší a opět jej vynecháme,~nicméně lze jej nalézt v~knize \citep[str. 30]{Falconer2014}.
\begin{figure}[h]
    \centering
    \begin{subfigure}{0.4\textwidth}
        \centering
        \includegraphics{ch02-bc-dimenze.pdf}
        \caption{Množina $B=\bigcup_{i=1}^4 B_i$}
        \label{subfig:bc-dimenze-pokryvana-mnozina}
    \end{subfigure}
    \qquad
    \begin{subfigure}{0.4\textwidth}
        \centering
        \includegraphics{ch02-bc-dimenze-delta-pokryti.pdf}
        \caption{$\delta$-pokrytí množiny $B$ (viz definice \ref{def:box-counting-dimenze})}
        \label{subfig:bc-dimenze-delta-pokryti}
    \end{subfigure}
    \qquad
    \begin{subfigure}{0.4\textwidth}
        \centering
        \includegraphics{ch02-bc-dimenze-pokryti-uz-koule.pdf}
        \caption{Pokrytí uzavřenými koulemi (viz bod \ref{thm:pokryti-delta-uz-koulemi})}
        \label{subfig:bc-dimenze-uz-koule}
    \end{subfigure}
    \qquad
    \begin{subfigure}{0.4\textwidth}
        \centering
        \includegraphics{ch02-bc-dimenze-pokryti-kvadry.pdf}
        \caption{Pokrytí pomocí kvádrů (viz bod \ref{thm:pokryti-delta-kvadry})}
        \label{subfig:bc-dimenze-kvadry}
    \end{subfigure}
    \qquad
    \begin{subfigure}{0.4\textwidth}
        \centering
        \includegraphics{ch02-bc-dimenze-pokryti-delta-sit.pdf}
        \caption{$\delta$-síť (viz bod \ref{thm:pokryti-delta-sit})}
        \label{subfig:bc-dimenze-delta-sit}
    \end{subfigure}
    \qquad
    \begin{subfigure}{0.4\textwidth}
        \centering
        \includegraphics{ch02-bc-dimenze-pokryti-ot-koule.pdf}
        \caption{Pokrytí otevřenými po dvou disjunktními koulemi (viz bod \ref{thm:pokryti-delta-dis-ot-koulemi})}
        \label{subfig:bc-dimenze-ot-koule}
    \end{subfigure}
    \caption{Ilustrace věty \ref{thm:ekvivalentni-def-box-counting-dimenze} (Inspirováno \citep[str. 29]{Falconer2014})}
    \label{fig:ilustrace-definic-bc-dimenze}
\end{figure}

Zároveň body \ref{thm:pokryti-delta-kvadry} a \ref{thm:pokryti-delta-sit} nám dávají dobré opodstatnění názvu tohoto typu dimenze,~neboť v~podstatě zkoumáme pokrývání daného obrazce "kostkami". Při aproximacích box-counting dimenze obrazce $F\subset\R^2$ tak lze pracovat s~mřížkou čtverců o~libovolné straně $\delta>0$,~kdy $N_\delta(F)$ stanovíme jako počet čtverců,~které se překrývají se zkoumaným obrazcem $F$. Když se tedy zpět vrátíme k~otázce rozebírané v~úvodu tohoto textu týkající se délky pobřeží (viz kapitola \ref{chapter:uvod_do_fraktalu}),~lze jeho "fráktálnost" do jisté míry vyjádřit právě popsaným způsobem (viz obrázek \ref{fig:aproximace-delky-pobrezi-vb}).
\begin{figure}[h]
    \centering
    \includegraphics[width=\textwidth]{Great_Britain_Box.pdf}
    \caption[Aproximace box-counting dimenze pobřeží Velké Británie]{Aproximace box-counting dimenze pobřeží Velké Británie. Převzato z Wikipedia Commons,~\url{https://en.wikipedia.org/wiki/Minkowski\%E2\%80\%93Bouligand\_dimension\#/media/File:Great\_Britain\_Box.svg}. Viz též \url{https://en.wikipedia.org/wiki/Minkowski\%E2\%80\%93Bouligand\_dimension}.}
    \label{fig:aproximace-delky-pobrezi-vb}
\end{figure}
Nyní se opět vrátíme k~fraktálům a výpočtům jejich dimenze,~čemuž jsme se věnovali již v~sekci \ref{sec:fraktalni_dimenze} kapitoly \ref{chapter:uvod_do_fraktalu},~konkrétně \ref{subsec:dimenze-fraktalu}. Tentokrát však budeme postupovat přímo podle definice box-counting dimenze \ref{def:box-counting-dimenze},~tedy budeme zvlášť zkoumat horní a dolní box-counting dimenzi.
\begin{example}[Cantorovo diskontinuum]\label{ex:cantorovo-diskontinuum}
    Již jsme měli možnost se přesvědčit,~že Cantorovo diskontinuum,~označme $F$,~má box-counting dimenzi $\ln{2}/\ln{3}$. Zkusme nyní výpočet zopakovat,~avšak vzlášť vypočítáme $\lowerdimB{F}$ a $\upperdimB{F}$ podíváme se,~zda se shodují.

    Jako první provedeme horní odhad. Je potřeba zvolit $\delta$ a na jeho základě dopočítat $N_\delta(F)$. V~$k$-té iteraci,~kde $k=0,1,2,\ldots$,~bude obecně $2^k$ intervalů,~každý o~délce $(1/3)^k$,~tedy pokud zvolíme $3^{-k}<\delta\leqslant 3^{-k+1}$,~pak intervaly o~délce nejvýše $\delta$ (viz věta \ref{thm:ekvivalentni-def-box-counting-dimenze},~bod \ref{thm:pokryti-delta-uz-koulemi}) tvoří $\delta$ pokrytí,~přičemž $N_\delta(F)\leqslant 2^k$. Tedy celkově pro $\delta$-pokrytí všech intervalů bude potřeba nejvýše $N_\delta(F)\leqslant 2^k$ intervalů $I_1,I_2,\ldots,I_{N_\delta(F)}$ o~průměru $3^{-k}<\diam{F_i}\leqslant 3^{-k+1}$ pro každé $i$. Z toho dostáváme
    \[\upperdimB{F}=\limsup_{\delta\to 0}\dfrac{\ln{N_\delta(F)}}{-\ln{\delta}}\leqslant\limsup_{k\to\infty}\dfrac{\ln{2^k}}{-\ln{3^{-k+1}}}=\limsup_{k\to\infty}\dfrac{k\ln{2}}{(k-1)\ln{3}}=\dfrac{\ln{2}}{\ln{3}}.\]
    Naopak pokud uvážíme intervaly délky $3^{-k-1}\leqslant\delta<3^{-k}$,~pak každý z nich má neprázdný průnik s~maximálně jedním intervalem $k$-té iterace $F$. Těch je,~jak již víme,~$2^k$,~tedy intervalů $I_1,I_2,\ldots,I_{N_\delta(F)}$ bude nejméně $2^k$ pro pokrytí $F$,~tzn. $N_\delta(F)\geqslant 2^k$. Tím dostáváme dolní odhad:
    \[\lowerdimB{F}=\liminf_{\delta\to 0}\dfrac{\ln{N_\delta(F)}}{-\ln{\delta}}\geqslant\liminf_{\delta\to 0}\dfrac{\ln{2^k}}{-\ln{3^{-k-1}}}=\liminf_{\delta\to 0}\dfrac{k\ln{2}}{(k+1)\ln{3}}=\dfrac{\ln{2}}{\ln{3}}.\]

    Protože $\lowerdimB{F}=\upperdimB{F}=\ln{2}/\ln{3}$,~tak box-counting dimenze Cantorova diskontinua je $\dimB{F}=\ln{2}/\ln{3}$. (Převzato z \citep[str. 32]{Falconer2014})
\end{example}
Podobně bychom postupovali pro rovinné obrazce.
\begin{example}[Kochova křivka]\label{ex:kochova-krivka}
    Opět ukážene horní a dolní odhad zvlášť. Kochovu křivku si označíme $K$.

    Obecně $k$-tá iterace Kochovy křivky bude obsahovat $4^n$ úseček,~každá o~délce $(1/3)^k$. Podobně jako v~předchozím příkladu \ref{ex:cantorovo-diskontinuum} zvolíme $3^{-k}<\delta\leqslant 3^{-k+1}$. Pokud si pro pokrytí zvolíme uzavřené koule
    \[K_\delta^1(x_1),K_\delta^2(x_2),\ldots,K_\delta^{N_\delta(K)}(x_{N_\delta(K)}),\;\text{kde}\;x_1,x_2,\ldots,x_{N_\delta(K)}\in\R^2\]
    pak $N_\delta(K)\leqslant 4^k$. Tedy
    \[\upperdimB{K}=\limsup_{\delta\to 0}\dfrac{\ln{N_\delta(K)}}{-\ln{K}}\leqslant\limsup_{k\to\infty}\dfrac{\ln{4^k}}{-\ln{3^{-k+1}}}=\limsup_{k\to\infty}\dfrac{k\ln{4}}{(k-1)\ln{3}}=\dfrac{\ln{4}}{\ln{3}}.\]

    Podobně pro dolní odhad uvážíme $3^{-k-1}\leqslant\delta<3^{-k}$. Vezmeme-li uzavřené koule $K_\delta^1(x_1),K_\delta^2(x_2),\ldots,K_\delta^{N_\delta(K)}(x_{N_\delta(K)})$,~pak žádný nemůže mít neprázdný průnik s~více než čtyřmi úsečkami,~tedy pro jejich pokrytí je zapotřebí alespoň $N_\delta(K)\geqslant 4^k/4=4^{k-1}$,~čímž dostáváme
    \[\lowerdimB{K}=\liminf_{\delta\to 0}\dfrac{\ln{N_\delta(K)}}{-\ln{K}}\geqslant\liminf_{k\to\infty}\dfrac{\ln{4^{k-1}}}{-\ln{3^{-k-1}}}=\liminf_{k\to\infty}\dfrac{(k-1)\ln{4}}{(k+1)\ln{3}}=\dfrac{\ln{4}}{\ln{3}}.\]
    Tzn.~$\dimB{K}=\ln{4}/\ln{3}$.
\end{example}
\begin{remark}
    Obecně množina $F$ skládající se z $m$ disjunktních kopií sebe samotné v~měřítku $r$ má dimenzi $\dimB{F}=-\ln{m}/\ln{r}$.
\end{remark}
Jako poslední se podíváme na ještě jedno možné pojetí box-counting dimenze. Připomeňme,~že $\delta$-okolím množiny $F$ v~metrickém prostoru $(X,\rho)$ rozumíme
\[F_\delta=\set{x\in X\mid\exists y\in X: \rho(x,y)<\delta}.\]
Budeme nyní sledovat,~jak "rychle" se mění objem $F_\delta$ pro $\delta\to 0$. A~se zmínkou objemu nám zde do hry opět vstupuje Lebesgueova míra $\lambda_n$, o níž jsme si povídali v sekci \ref{sec:lebesgueova-mira}. Podívejme se nejdříve na několik příkladů.
\begin{itemize}
    \item Pro úsečku $u\subset\R^3$ o~délce $\ell$ lze objem jejího $\delta$-okolí stanovit jako
    \[\lambda_3(u)=\dfrac{4}{3}\pi \delta^3+\pi\delta^2\ell.\]
    Pokud však uvážíme $\delta$ dostatečně malé,~lze první člen zanedbat a psát
    \[\lambda_3(u)\approx\pi\ell\delta^2.\]
    \item V~případě rovinného obrazce $F=\set{(x,y,0)\mid x,y\in\R^3}$ o~obsahu $A$ je objem $\lambda_3(F_\delta)\approx 2A\delta$.
    \item Pro kouli $B_r(x)\subset\R^3$ o~objemu $V$,~kde $x\in\R^3$ a $r>0$ je $\lambda_3((B_r(x))_\delta)\approx V$,~neboť změna objemu je v~tomto případě zanedbatelná.
\end{itemize}
Můžeme si všimnout,~že v~každém případě odhad objemu vychází $\lambda_3(F)\approx c\delta^{3-s}$,~kde $c>0$ je závislé na původní míře $F$ a $s$ udává dimenzi. Obecněji pro množinu $F\subseteq\R^n$ bychom došli k~$\lambda_n(F)\approx c\delta^{n-s}$. Nyní,~podobně jako v~úvodu této sekce,~zkusme opět vyjádřit $s$:
\begin{align*}
    \ln{\lambda_n(F)}&\approx\ln{c}+(n-s)\ln{\delta}\\
    s\ln{\delta}&\approx n\ln{\delta}-\ln{\lambda_n(F)}+\ln{c}\\
    s&\approx n-\dfrac{\ln{\lambda_n(F)}}{\ln{\delta}}+\dfrac{\ln{c}}{\ln{\delta}}.
\end{align*}
Poslední člen bude v~limitě opět nulový.

Lze ukázat,~že $s$ není v~tomto případě nic jiného,~než již námi zkoumaná\linebreak{}box-counting dimenze. To si shrneme a dokážeme ve větě \ref{thm:bc-dimenze-lebesgueova-mira}.
\begin{theorem}\label{thm:bc-dimenze-lebesgueova-mira}
    Nechť $F\in\mathcal{L}^n$. Pak platí: 
    \begin{enumerate}[label=(\roman*)]
        \item $\lowerdimB{F}=n-\liminf\limits_{\delta\to 0}\dfrac{\ln{\lambda_n(F_\delta)}}{\ln{\delta}}$,
        \item $\upperdimB{F}=n-\limsup\limits_{\delta\to 0}\dfrac{\ln{\lambda_n(F_\delta)}}{\ln{\delta}}$.
    \end{enumerate}
\end{theorem}
\todo{Přidat důkaz}