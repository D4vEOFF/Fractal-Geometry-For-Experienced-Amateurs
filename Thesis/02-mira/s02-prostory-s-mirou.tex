\section{Prostory s~mírou}\label{sec:prostory-s-mirou}

Jak již bylo zmíněno v~úvodu, klíčovým pojmem v~této kapitole (a pro studium fraktálů obecně) je takzvaná \emph{míra}. Ta pro nás představuje obecný způsob, jak můžeme množinám přiřadit v~jistém smyslu "velikost". Konkrétněji, byť vágně, lze říci, že sestává-li množina z konečného nebo spočetného množství "rozumných" částí, pak součet velikostí všech těchto dílčích množin je roven velikosti celé množiny, kterou nazveme její \emph{mírou}. Pro začátek celkem jednoduchá myšlenka.

Pro formální zavedení tohoto pojmu však budeme muset nejprve zavést ještě jiný pojem, a to tzv. \emph{$\sigma$-algebru}.

\subsection{$\sigma$-algebra}\label{subsec:sigma-algebra}

\begin{definition}[$\sigma$-algebra]\label{def:sigma-algebra}
    Nechť $X$ je libovolná množina a systém podmnožin $\mathcal{A}\subseteq\powset{X}$. Pak $\mathcal{A}$ je $\sigma$-algebra na množině $X$, pokud:
    \begin{enumerate}[label=(\roman*)]
        \item\label{def:sigma-algebra-podm1} $X\in\mathcal{A}$,
        \item\label{def:sigma-algebra-podm2} $\forall A\in X: A\in\mathcal{A}\implies X\setminus A\in\mathcal{A}$,
        \item\label{def:sigma-algebra-podm3} $\forall i\in\N: A_i\in\mathcal{A}\implies\bigcup_{i=1}^\infty A_i\in\mathcal{A}$.
    \end{enumerate}
\end{definition}

\begin{example}
    \begin{itemize}
        \item Triviálními příklady $\sigma$-algeber jsou množiny $\emptyset$, $\powset{X}$ a $\set{\emptyset,X}$ pro libovolnou množinu $X$.
        \item Pro konečnou množinu $X=\set{a,b,c,d}$ je jednou možnou $\sigma$-algebrou systém množin
        \[\Sigma=\set{\emptyset,\set{a,b},\set{c,d},\set{a,b,c,d}}.\]
    \end{itemize}
    Sami se zkuste přesvědčit, že všechny zmíněné příklady vyhovují definici \ref{def:sigma-algebra}.
\end{example}

Než vyslovíme něco dalšího o~$\sigma$-algebrách a jejich významu, podíváme se seznam některých vesměs jednoduchých pozorováních zformulovaných níže v~tvrzení \ref{thm:sigma-algebra-vlastnosti}.
\begin{theorem}[Vlastnosti $\sigma$-algebry]\label{thm:sigma-algebra-vlastnosti}
    Nechť $\mathcal{A}$ je $\sigma$-algebra na množině $X$. Pak platí:
    \begin{enumerate}[label=(\roman*)]
        \item $\emptyset\in\mathcal{A}$,
        \item $\forall i\in\N: A_i\in\mathcal{A}\implies\bigcap_{i=1}^\infty A_i\in\mathcal{A}$,
        \item $\forall i\leqslant n: A_i\in\mathcal{A}\implies\bigcup_{i=1}^n A_i\in\mathcal{A}\land\bigcap_{i=1}^n A_i\in\mathcal{A}$ pro $n\in\N$,
        \item $A,B\in\mathcal{A}\implies A\setminus B\in\mathcal{A}$.
    \end{enumerate}
\end{theorem}

Z tohoto tvrzení je již lépe vidět, proč jsou pro nás $\sigma$-algebry tak příjemným objektem. Jsou totiž \emph{uzavřené} na všechny základní množinové operace. To se nám bude později hodit při zavedení míry, ke které směřujeme. Důkaz těchto dílčích tvrzení přitom není nikterak složitý.
\begin{proof}
    Mějme $\sigma$-algebru $\mathcal{A}$ na množině $X$.
    \begin{enumerate}[label=\textit{(\roman*)}]
        \item Z podmínky \ref{def:sigma-algebra-podm1} definice \ref{def:sigma-algebra} víme, že $X\in\mathcal{A}$ a z podmínky \ref{def:sigma-algebra-podm2} tedy plyne $X\setminus X=\emptyset\in\mathcal{A}$.
        \item Mějme množiny $A_1,A_2,\ldots$, kde $A_i\in\mathcal{A}$ pro každé $i\in\N$. Společně s~využitím De Morganových zákonů plyne následující:
        \[\bigcap\limits_{i=1}^\infty A_i=\overbrace{X\setminus\underbrace{\bigcup\limits_{i=1}^\infty \overbrace{(X\setminus A_i)}^{\text{$\in\mathcal{A}$ podle \ref{def:sigma-algebra-podm2}}}}_{\text{$\in\mathcal{A}$ podle \ref{def:sigma-algebra-podm3}}}}^\text{$\in\mathcal{A}$ podle \ref{def:sigma-algebra-podm2}}\in\mathcal{A}.\]
        \item Nechť jsou dány množiny $A_1,A_2,\dots,A_n$, kde $A_i\in\mathcal{A}$ pro každé $i\leqslant n$. Když pro každé $j>n$ položíme $A_j=\emptyset$, pak platí
        \[\bigcup\limits_{i=1}^n A_i=\bigcup\limits_{i=1}^\infty A_i\in\mathcal{A}\]
        a podobně pro $\bigcap_{i=1}^n A_i\in\mathcal{A}$ podle předešlého bodu.
        \item Pro libovolné množiny $A,B\in\mathcal{A}$ platí
        \[A\setminus B=\overbrace{A\cup\underbrace{(X\setminus B)}_{\text{$\in\mathcal{A}$ podle \ref{def:sigma-algebra-podm2}}}}^{\text{$\in\mathcal{A}$ podle \ref{def:sigma-algebra-podm3}}}\in\mathcal{A}.\]
    \end{enumerate}
\end{proof}

\subsection{Míra}\label{subsec:mira}

V tuto chvíli máme již vše potřebné k~zavedení pojmu míra, resp. \emph{prostor s~mírou}.
\begin{definition}[Prostor s~mírou]\label{def:prostor-s-mirou}
    Nechť $\mathcal{A}$ je $\sigma$-algebra na množině $X$. Zobrazení $\mapping{\mu}{\mathcal{A}}{\langle0,\infty\rangle}$ se nazývá míra na $\mathcal{A}$, pokud platí:
    \begin{enumerate}[label=(\roman*)]
        \item $\mu(\emptyset)=0$,
        \item pro množiny $A_1,A_2,\dots$ po dvou disjunktní, kde $A_i\in\mathcal{A}$ pro všechna $i\in\N$, platí:
        \[\mu\left(\bigcup\limits_{i=1}^\infty A_i\right)=\sum_{i=1}^\infty\mu(A_i).\text{\rightnote{$\sigma$-aditivita}}\]
    \end{enumerate}
    Uspořádanou trojici $(X,\mathcal{A},\mu)$ nazýváme prostor s~mírou.
\end{definition}