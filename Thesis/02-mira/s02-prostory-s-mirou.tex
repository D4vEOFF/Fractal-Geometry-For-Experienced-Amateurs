\section{Prostory s~mírou}\label{sec:prostory-s-mirou}

Jak již bylo zmíněno v~úvodu, klíčovým pojmem v~této kapitole (a pro studium fraktálů obecně) je takzvaná \emph{míra}. Ta pro nás představuje obecný způsob, jak můžeme množinám přiřadit v~jistém smyslu "velikost". Konkrétněji, byť vágně, lze říci, že sestává-li množina z konečného nebo spočetného množství "rozumných" částí, pak součet velikostí všech těchto dílčích množin je roven velikosti celé množiny, kterou nazveme její \emph{mírou}. Pro začátek celkem jednoduchá myšlenka.

Pro formální zavedení tohoto pojmu však budeme muset nejprve zavést ještě jiný pojem, a to tzv. \emph{$\sigma$-algebru}.

\subsection{$\sigma$-algebra}\label{subsec:sigma-algebra}

\begin{definition}[$\sigma$-algebra]\label{def:sigma-algebra}
    Nechť $X$ je libovolná množina a systém podmnožin $\mathcal{A}\subseteq\powset{X}$. Pak $\mathcal{A}$ je \emph{$\sigma$-algebra} na množině $X$, pokud:
    \begin{enumerate}[label=(\roman*)]
        \item\label{def:sigma-algebra-podm1} $X\in\mathcal{A}$.
        \item\label{def:sigma-algebra-podm2} Pro každou množinu $A\in\mathcal{A}$ platí $X\setminus A\in\mathcal{A}$.
        \item\label{def:sigma-algebra-podm3} Pro libovolné množiny $A_1,A_2,\ldots\in\mathcal{A}$ platí $\bigcup_{i=1}^\infty A_i\in\mathcal{A}$.
    \end{enumerate}
\end{definition}
\todo{Psát na konci každého bodu v definici/větě \textbf{tečku}, \textbf{čárku} nebo \textbf{středník}?}

\begin{example}
    Jednoduché příklady $\sigma$-algeber:
    \begin{itemize}
        \item Triviálními příklady $\sigma$-algeber jsou množiny $\emptyset$, $\powset{X}$ a~$\set{\emptyset,X}$ pro libovolnou množinu $X$.
        \item Pro konečnou množinu $X=\set{a,b,c,d}$ je jednou možnou $\sigma$-algebrou systém množin
        \[\Sigma=\set{\emptyset,\set{a,b},\set{c,d},\set{a,b,c,d}}.\]
    \end{itemize}
    Sami se zkuste přesvědčit, že všechny zmíněné příklady vyhovují definici \ref{def:sigma-algebra}.
\end{example}

Než vyslovíme něco dalšího o~$\sigma$-algebrách a jejich významu, podíváme se seznam některých vesměs jednoduchých pozorováních zformulovaných níže v~tvrzení \ref{thm:sigma-algebra-vlastnosti}.
\begin{theorem}[Vlastnosti $\sigma$-algebry]\label{thm:sigma-algebra-vlastnosti}
    Nechť $\mathcal{A}$ je $\sigma$-algebra na množině $X$. Pak platí:
    \begin{enumerate}[label=(\roman*)]
        \item $\emptyset\in\mathcal{A}$.
        \item Pro libovolné množiny $A_1,A_2,\ldots\in\mathcal{A}$ platí $\bigcap_{i=1}^\infty A_i\in\mathcal{A}$.
        \item Pro všechny množiny $A_1,A_2,\ldots,A_n\in\mathcal{A}$ platí
        \[\bigcup_{i=1}^n A_i\in\mathcal{A}\land\bigcap_{i=1}^n A_i\in\mathcal{A}.\]
        \item Jsou-li $A,B\in\mathcal{A}$ pak $A\setminus B\in\mathcal{A}$.
    \end{enumerate}
\end{theorem}

Z tohoto tvrzení je již lépe vidět, proč jsou pro nás $\sigma$-algebry tak příjemným objektem. Jsou totiž \emph{uzavřené} na všechny základní množinové operace. To se nám bude později hodit při zavedení míry, ke které směřujeme. Důkaz těchto dílčích tvrzení přitom není nikterak složitý.
\begin{proof}
    Mějme $\sigma$-algebru $\mathcal{A}$ na množině $X$.
    \begin{enumerate}[label=\textit{(\roman*)}]
        \item Z podmínky \ref{def:sigma-algebra-podm1} definice \ref{def:sigma-algebra} víme, že $X\in\mathcal{A}$ a z podmínky \ref{def:sigma-algebra-podm2} tedy plyne $X\setminus X=\emptyset\in\mathcal{A}$.
        \item Mějme množiny $A_1,A_2,\ldots\in\mathcal{A}$. Společně s~využitím De Morganových zákonů plyne následující:
        \[\bigcap\limits_{i=1}^\infty A_i=\overbrace{X\setminus\underbrace{\bigcup\limits_{i=1}^\infty \overbrace{(X\setminus A_i)}^{\text{$\in\mathcal{A}$ podle \ref{def:sigma-algebra-podm2}}}}_{\text{$\in\mathcal{A}$ podle \ref{def:sigma-algebra-podm3}}}}^\text{$\in\mathcal{A}$ podle \ref{def:sigma-algebra-podm2}}\in\mathcal{A}.\]
        \item Nechť jsou dány množiny $A_1,A_2,\ldots,A_n\in\mathcal{A}$. Když pro každé $j>n$ položíme $A_j=\emptyset$, pak platí
        \[\bigcup\limits_{i=1}^n A_i=\bigcup\limits_{i=1}^\infty A_i\in\mathcal{A}\]
        a podobně pro $\bigcap_{i=1}^n A_i\in\mathcal{A}$ podle předešlého bodu.
        \item Pro libovolné množiny $A,B\in\mathcal{A}$ platí
        \[A\setminus B=\overbrace{A\cup\underbrace{(X\setminus B)}_{\text{$\in\mathcal{A}$ podle \ref{def:sigma-algebra-podm2}}}}^{\text{$\in\mathcal{A}$ podle \ref{def:sigma-algebra-podm3}}}\in\mathcal{A}.\]
    \end{enumerate}
\end{proof}

\subsection{Míra}\label{subsec:mira}

V tuto chvíli máme již vše potřebné k~zavedení pojmu míra, resp. prostor s~mírou.
\begin{definition}[Prostor s~mírou]\label{def:prostor-s-mirou}
    Nechť $\mathcal{A}$ je $\sigma$-algebra na množině $X$. Zobrazení $\mapping{\mu}{\mathcal{A}}{\langle0,\infty\rangle}$ se nazývá \emph{míra} na $\mathcal{A}$, pokud platí:
    \begin{enumerate}[label=(\roman*)]
        \item\label{def:mira-podm1} $\mu(\emptyset)=0$,
        \item\label{def:mira-podm2} pro množiny $A_1,A_2,\ldots\in\mathcal{A}$ po dvou disjunktní je
        \[\mu\left(\bigcup\limits_{i=1}^\infty A_i\right)=\sum_{i=1}^\infty\mu(A_i).\mathrightnote{$\sigma$-aditivita}\]
    \end{enumerate}
    Uspořádanou trojici $(X,\mathcal{A},\mu)$ nazýváme \emph{prostor s~mírou}.
\end{definition}

Vzhledem k tomu, co míra reprezentuje (tj. zobecnění délky, obsahu, objemu), jsou tyto požadavky intuitivně dosti smysluplné.

\begin{example}\label{ex:mira}
    Příklady prostorů s mírou:
    \begin{itemize}
        \item Asi pro nás nejtypičtější způsob, jak měřit "velikost" množiny, je podle \emph{počtu prvků}. Pro libovolnou množinu $X$ a potenční množinu $\powset{X}$ lze definovat prostor s mírou $(X,\powset{X},\mu)$, kde pro libovolnou množinu $A\in\powset{X}$ položíme $\mu(A)=|A|$. Takto definované míře $\mu$ říkáme \emph{aritmetická míra}.
        \item Máme libovolnou množinu $X$ a $\sigma$-algebru $\mathcal{A}$. Zvolme si pevně $a\in X$. Míru libovolné množiny $A\in\mathcal{A}$ lze definovat jako $\mu(A)=\chi_A(a)$, kde $\chi_A$ je charakteristická funkce množiny $A$.
        \item Zobrazení přiřazující náhodnému jevu pravděpodobnost je též případem míry. Označíme-li si $\Omega=\set{\omega_1,\omega_2,\ldots,\omega_n}$ množinu všech elementárních jevů a $\mathcal{F}\subseteq\powset{\Omega}$, pak $\mapping{\mathsf{P}}{\mathcal{F}}{\langle0,1\rangle}$ definovaná pro $A\in\mathcal{F}$ jako
        \[\mathsf{P}(A)=\dfrac{|A|}{|\Omega|}\]
        je mírou na $\mathcal{F}$. Speciálně $\mathsf{P}(\Omega)=1$.
    \end{itemize}
\end{example}

Ve všech případech zobrazení $\mu$ v příkladu \ref{ex:mira} se lze snadno přesvědčit, že se jedná o míru, tedy že splňuje podmínky \ref{def:mira-podm1} a \ref{def:mira-podm2} uvedené v definici \ref{def:prostor-s-mirou} výše.

Pojďme nyní prozkoumat vlastnosti míry trochu hlouběji.
\begin{theorem}[Vlastnosti míry]\label{thm:mira-vlastnosti}
    Nechť $\mu$ je míra na $\sigma$-algebře $\mathcal{A}$. Pak platí následující:
    \begin{enumerate}[label=(\roman*)]
        \item\label{thm:mira-konecne-sjednoceni} Jsou-li množiny $A_1,A_2,\ldots,A_n\in\mathcal{A}$ po dvou disjunktní, pak
        \[\mu\left(\bigcup_{i=1}^n A_i\right)=\sum_{i=1}^{n}\mu(A_i).\]
        \item\label{thm:mira-monotonie} Pokud $A,B\in\mathcal{A}$ a $A\subseteq B$, pak
        \[\mu(A)\leqslant\mu(B).\mathrightnote{monotonie míry}\]
        Navíc pokud $\mu(A)<\infty$, pak $\mu(B\setminus A)=\mu(B)-\mu(A)$.
        \item Pokud $A_1,A_2,\ldots\in\mathcal{A}$ je neroustoucí posloupnost množin, tj. taková, že $A_i\subseteq A_{i+1}$ pro každé $i\in\N$, pak
        \[\lim_{i\to\infty}\mu(A_i)=\mu\left(\bigcup_{i=1}^\infty A_i\right).\]
        \item Pokud $A_1,A_2,\ldots\in\mathcal{A}$ je neklesající posloupnost množin, tj. taková, že $A_i\supseteq  A_{i+1}$ pro každé $i\in\N$, a navíc $\mu(A_1)<\infty$, pak
        \[\lim_{j\to\infty}\mu(A_j)=\mu\left(\bigcap_{j=1}^\infty A_j\right).\]
        \item\label{thm:mira-sigma-subaditivita} Pokud $A_1,A_2,\ldots\in\mathcal{A}$, pak
        \[\mu\left(\bigcup_{i=1}^\infty A_i\right)\leqslant\sum_{i=1}^{\infty}\mu(A_i).\mathrightnote{$\sigma$-subaditivita}\]
    \end{enumerate}
\end{theorem}

Poslední vlastnost \ref{thm:mira-sigma-subaditivita} je tzv. \emph{$\sigma$-subaditivita}. Oproti $\sigma$-aditivitě se liší tím, že u množin $A_1,A_2,\dots$ se nepožaduje, aby byly po dvou disjunktní, tzn. mohou se "překrývat". Je však intuitivně nejspíše jasné, že součtem měr všech těchto množin určitě nemůžeme získat míru větší než je míra jejich sjednocení (dané "překryvy" započítáváme v sumě vícekrát). Podobně i monotonie dává intuitivně smysl, neboť část větší množiny jistě nemůže mít větší míru než celek. Na formální stránku věci se podíváme nyní.
\begin{proof}
    V důkazu využijeme některé vlastnosti $\sigma$-algebry z věty \ref{thm:sigma-algebra-vlastnosti}.
    \begin{enumerate}[label=\textit{(\roman*)}]
        \item Pokud pro každé $j>n$ položíme $A_j=\emptyset$, pak z definice míry plyne
        \[\mu\left(\bigcup_{i=1}^n A_i\right)=\mu\left(\bigcup_{i=1}^\infty A_i\right)\stackrel{\ref{def:mira-podm2}}{=}\sum_{i=1}^{\infty}\mu(A_i)\stackrel{\ref{def:mira-podm1}}{=}\sum_{i=1}^{n}\mu(A_i).\]
        \item Nechť jsou dány $A,B\in\mathcal{A}$, takové, že $A\subseteq B$. Pak $B=A\cup(B\setminus A)$, přičemž $A$ a $B\setminus A$ jsou disjunktní. Tedy podle bodu \ref{def:mira-podm2} lze psát $\mu(B)=\mu(A)+\mu(B\setminus A)\geqslant\mu(A)$, protože $\mu(B\setminus A)\geqslant 0$.
        \item Mějme nerostoucí posloupnost množin $A_1,A_2,\ldots\in\mathcal{A}$ (viz obrázek \ref{fig:nekl-posl-mnozin}).
        \begin{figure}[h]
            \centering
            \includegraphics{ch02-neklesajici-posl-mnozin.pdf}
            \caption{Neklesající posloupnost množin $A_1,A_2,\ldots$}
            \label{fig:nekl-posl-mnozin}
        \end{figure}
        Definujeme posloupnost množin $B_1,B_2,\ldots$ následovně:
        \[B_1=A_1,\;B_2=A_2\setminus A_1,\;\forall i\geqslant 2: B_i=A_i\setminus A_{i-1}.\]
    \end{enumerate}
\end{proof}