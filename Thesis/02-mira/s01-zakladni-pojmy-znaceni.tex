\section{Základní pojmy a značení}\label{sec:zakladni-pojmy-a-znaceni}

V tomto oddílu se v~krátkosti zaměříme na připomenutí některých pojmů a značení,~které budeme dále využívat. Související teorii týkající se mnoha záležitostí v~tomto případě vynecháme s~předpokladem,~že ji čtenář již zná. Pokud tomu však v~některých případech takto nebude,~lze tuto část textu považovat za výčet konceptů,~které pro zvládnutí nadcházející teorie budeme potřebovat.

\todo{Doplnit pojmy a značení podle dalšího textu
    \begin{itemize}
        \item Limita funkce
        \item Limes superior/inferior
        \item Metrický prostor
        \item Otevřená/uzavřená koule
        \item Otevřená/uzavřená množina
        \item Průměr množiny
        \item $\delta$-okolí množiny
        \item Pokrytí, zjemnění
        \item $\delta$-pokrytí
        \item $\delta$-síť
        \item Vnitřek, hranice množiny
        \item (bi-)lipschitzovské zobrazení
    \end{itemize}
}