\section{Základní pojmy a značení}\label{sec:zakladni-pojmy-a-znaceni}

V tomto oddílu se v~krátkosti zaměříme na připomenutí některých pojmů a značení,~které budeme dále využívat. Související teorii týkající se mnoha záležitostí v~tomto případě vynecháme s~předpokladem,~že ji čtenář již zná. Pokud tomu však v~některých případech takto nebude,~lze tuto část textu považovat za výčet konceptů,~které pro zvládnutí nadcházející teorie budeme potřebovat.

Výklad v této kapitole a dále v kapitole \ref{chapter:hausdorffuv-mp} se bude točit především okolo tzv. \emph{metrických prostorů} (viz definice \ref{def:metricky-prostor}).
\begin{definition}[Metrický prostor]\label{def:metricky-prostor}
    Metrickým prostorem nazýváme uspořádanou dvojici $(X,\varrho)$, kde $X\neq\emptyset$ a $\mapping{\varrho}{X\times X}{\R_0^+}$ je zobrazení splňující:
    \begin{enumerate}[label=(\alph*)]
        \item $\forall x,y\in X: \varrho(x,y)=0\iff x=y$,
        \item $\forall x,y\in X: x\neq y\implies \varrho(x,y)>0$,
        \item $\forall x,y\in X: \varrho(x,y)=\varrho(y,x)$,\rightnote{symetrie}
        \item $\forall x,y,z\in X: \varrho(x,z)\leqslant\varrho(x,y)+\varrho(y,z)$.\rightnote{trojúhelníková nerovnost}
    \end{enumerate}
\end{definition}

\subsection{Limity v metrických prostorech}\label{subsec:limity-v-mp}

S limitou posloupnosti a funkce jedné proměnné je čtenář nejspíše dobře seznámen. V kontextu metrických prostorů definujeme pojem limity následovně (viz definice \ref{def:limita-mp} a \ref{def:limsup-liminf-mp}).
\begin{definition}[Limita posloupnosti]\label{def:limita-mp}
    Mejme metrický prostor $(X,\varrho)$ a posloupnost $\set{x_n}_{n=1}^\infty$, kde $x_i\in X$ pro každé $i\in\N$. Pak posloupnost $\set{x_n}_{n=1}^\infty$ má limitu $x\in X$, píšeme $\lim_{n\to\infty}x_n=x$, nebo též $x_n\to x$, pokud
    \[\forall\varepsilon>0\;\exists n_0\in\N\;\forall n\geqslant n_0: \varrho(x_n,x)<\varepsilon.\]
\end{definition}
\begin{definition}[Limes superior, limes inferior]\label{def:limsup-liminf-mp}
    Nechť je dán metrický prostor $(X,\varrho)$ a posloupnost $\set{x_n}_{n=1}^\infty$. Pak definujeme:
    \begin{enumerate}[label=(\alph*)]
        \item Limes superior $\limsup_{n\to\infty}=\set{x\in X\mid \exists n_1,n_2,\ldots\in\N: \lim_{k\to\infty}x_{n_k}=x}$.
        \item Limes inferior $\limsup_{n\to\infty}=\lim_{n\to\infty}\inf\set{x_k\mid k\geqslant n}$.
    \end{enumerate}
\end{definition}

\todo{Doplnit pojmy a značení podle dalšího textu
    \begin{itemize}
        \item Limita funkce
        \item Limes superior/inferior
        \item Metrický prostor, úplný MP
        \item Cauchyovská posloupnost
        \item Bodová a stejnoměrná konvergence
        \item vzdálenost bodu od množiny
        \item Otevřená/uzavřená koule
        \item Otevřená/uzavřená množina
        \item Kompaktní množina, věta o kompaktnosti v $\R^n$, věta o uzavřenosti kompaktní množiny.
        \item Kvádr a objem kvádru
        \item Průměr množiny
        \item $\delta$-okolí množiny
        \item Pokrytí, zjemnění
        \item $\delta$-pokrytí
        \item $\delta$-mříž
        \item Vnitřek, hranice množiny
        \item (bi-)lipschitzovské zobrazení
    \end{itemize}
}