\section{Lebesgueova míra}\label{sec:lebesgueova-mira}

\todo{Doplnit zmínku o~Jordanově-Peanově obsahu}

V předešlé sekci \ref{sec:prostory-s-mirou} jsme se povídali o~pojmu \emph{míra} obecně a podívali jsme se na několik příkladů. Obecnou ideu měření "velikosti" lze založit např. na aproximaci obecné množiny pomocí \emph{spočetných sjednocení útvarů},~jejichž "velikost" umíme jednoduše určit. V~dalším textu se omezíme pouze na množinu $\R^n$.

Na zmíněné myšlence je postavena definice tzv. \emph{$n$-rozměrné Lebesgueovy míry},~kdy obecnou množinu budeme pokrývat pomocí \emph{kvádrů}. Připomeňme,~že obecně\linebreak\mbox{$n$-rozměrným} kvádrem\index{$n$-rozměrný kvádr} $I$ rozumíme kartézský součin \emph{intervalů}
\[\langle a_1,b_1\rangle,\ldots,\langle a_n,b_n\rangle\subseteq\R,\]
tj.
\[I=\prod_{i=1}^{n}\langle a_i,b_i\rangle=\langle a_1,b_1\rangle\times\langle a_2,b_2\rangle\times\dots\times\langle a_n,b_n\rangle\]
a jeho objem\index{kvádr!objem kvádru} definujeme jako
\[\vol^n(I)=\prod_{i=1}^{n}(b_i-a_i).\]
Nyní si definujeme tzv. \emph{vnější Lebesgueovu míru}.
\begin{definition}[Vnější Lebesgueova míra]\label{def:vnejsi-lebegueova-mira}
    Nechť $A\subseteq\R^n$. Pak vnější $n$-rozměr\-nou Lebesgueovou mírou\index{vnější $n$-rozměrná Lebesgueova míra} $A$ je
    \[\lambda_n^*(A)=\inf\set{\sum_{j=1}^{\infty}\vol^n(I_j)\;\middle|\;\text{$I_j$ je kvádr}\land A\subseteq\bigcup_{i=1}^\infty I_j}.\]
\end{definition}
Vnější Lebesgueova míra množiny intuitivně zachycuje informaci o~"velikosti" dané množiny.  Lze ihned vidět,~že pro libovolnou množinu $A\subseteq\R^n$ je $\lambda_n^*(A)\in\R_0^+$,~protože $\vol^n(I_j)\geqslant0$ pro každé $j\in\N$.
\begin{example}\label{ex:lebegueova-mira-trivialni-priklady}
    Ukažme si některé triviální příklady výpočtů vnější Lebesgueovy míry z definice (viz \ref{def:vnejsi-lebegueova-mira}),~tedy budeme hledat příslušné pokrytí dané množiny.
    \begin{itemize}
        \item Pro prázdnou množinu $\emptyset$ je $\lambda_n^*(\emptyset)=0$,~neboť $\emptyset\subseteq\emptyset$ (tedy prázdná množina je pokrytím sebe sama) a $\vol^n(\emptyset)=0$.
        \item Mějme libovolnou konečnou množinu $A=\set{x_1,x_2,\dots,x_n}\subseteq\R^n$. Pro každé $x_j$ stačí položit $I_j=\set{x_j}$ pro každé $1\leqslant j\leqslant n$,~což je degenerovaný interval,~jehož objem $\vol^n(I_j)=0$.
        \item Pro libovolnou spočetnou množinu $A=\set{x_i\mid i\in\N}\subseteq\R^n$ je $\lambda_n^*(A)=0$. Pokrytí volíme stejně jako v~předešlém bodě. Tedy např. pro $\Q\subset\R$ je $\lambda_1^*(\Q)=0$,~neboť $\Q$ je spočetná.
        \item Pro množinu reálných čísel $\R$ je $\lambda_1^*(\R)=\infty$,~avšak pro 
        \[A=\set{(x,0)\mid x\in\R}\subset\R^2\]
        (reálná osa v~$\R^2$) je $\lambda_2^*(A)=0$.
    \end{itemize}
\end{example}
\begin{example}[Výpočet vnější Lebesgueovy míry intervalu]\label{ex:lebegueova-mira-delka-intervalu}
    Jako poslední si ukážeme,~že vnější Lebesgueova míra v~případě intervalu (ať už otevřeného,~nebo uzavřeného) skutečně koresponduje s~jeho délkou,~tedy pro $I=(a,b)\subset\R$ je $\lambda_1^*(I)=b-a$. Zde je potřeba ukázat dvojici nerovností: $\lambda_1^*(I)\leqslant b-a$ a $\lambda_1^*(I)\geqslant b-a$.

    Zde je potřeba dávat pozor na to,~že pokrytí,~které hledáme,~musí být spočetné. Začneme první nerovností.

    \begin{itemize}
        \item \textbf{Důkaz $\lambda_1^*(I)\geqslant b-a$.} Nechť je dána posloupnost intervalů $J_1,J_2,\ldots$,~taková,~že $I\subseteq\bigcup_{i=1}^\infty J_i$. Protože $(a,b)$ je otevřený interval,~existuje interval $I^\prime=\langle a+\varepsilon,b-\varepsilon\rangle\subset(a,b)$ pro libovolné $\varepsilon>0$.
        
        Nechť je tedy dáno $\varepsilon>0$. Protože však interval $\langle a+\varepsilon,b-\varepsilon\rangle$ je kompaktní množina a navíc
        \[\langle a+\varepsilon,b-\varepsilon\rangle\subset(a,b)\subseteq\bigcup_{i=1}^\infty J_i\]
        lze podle Heineho-Borelovy věty (viz \todo{doplnit odkaz}) vybrat z pokrytí $J_1,J_2,\ldots$ konečné podporytí,~tzn. existuje konečná posloupnost množin $J_1^\prime,J_2^\prime,\ldots,J_n^\prime$,~taková,~že
        \[I^\prime\subseteq\bigcup_{i=1}^\infty J_i^\prime.\]
        Z tohoto pokrytí si nyní vybereme pouze takové intervaly $J_i^\prime$,~které mají neprázdný průnik s~$I$,~tedy
        \[J_i^\prime\cap I\neq\emptyset.\]
        Tyto intervaly si označíme $K_1,K_2,\ldots,K_m$,~kde $m\leqslant n$. Není těžké si rozmyslet,~že $K_1,K_2,\ldots,K_m$ tvoří opět pokrytí $I^\prime$ a navíc jejich sjednocení tvoří interval,~tj.
        \[\bigcup_{i=1}^m K_i=(L,R)\supset I^\prime.\]
        Zároveň víme,~že $L\leqslant a+\varepsilon$ a $R\geqslant b-\varepsilon$. Z toto tedy plyne,~že
        \[\vol^1((L,R))=\sum_{i=1}^{m}\vol^1(K_i)=L-R\geqslant (b-\varepsilon)-(a+\varepsilon)=b-a-2\varepsilon.\]
        Celkově máme
        \begin{align*}
            \sum_{i=1}^{\infty}\vol^1(J_i)&\geqslant\sum_{i=1}^{n}\vol^1(J_i^\prime)\geqslant\sum_{i=1}^{m}\vol^1(K_i)\geqslant\vol^1((L,R))\\
            &=R-L\geqslant (b-\varepsilon)-(a+\varepsilon)=b-a-2\varepsilon.
        \end{align*}
        Tím je dokázána nerovnost $\lambda_1^*(I)\geqslant b-a$.
        \item \textbf{Důkaz $\lambda_1^*(I)\leqslant b-a$.} Oproti předešlému výpočtu je důkaz této části velmi snadný. Samotný interval $I=(a,b)$ tvoří totiž pokrytí sebe samotného,~tzn.
        \[\lambda_1^*(I)\leqslant\vol^1((a,b))=b-a.\]
    \end{itemize}
    Z platnosti obou nerovností tedy máme,~že $\lambda_1^*(I)=b-a$.
\end{example}
\begin{remark}
    Vraťme se na chvíli k~větě \ref{thm:mira-vlastnosti} o~vlastnostech míry,~konkrétně bod \ref{thm:mira-nerost-posl}. Předpoklad $\mu(A_1)<\infty$ zde vynechat nelze. Stačí vzít množiny $A_j=\langle j,\infty)$,~tzn. $\lambda_n^*(A_j)=\lambda_n^*(\langle j,\infty))=\infty$ pro každé $j\in\N$. Snadno si rozmyslíme,~že
    \[\bigcap_{i=1}^\infty A_i=\emptyset,\]
    nicméně lze vidět,~že zatímco $\lim_{j\to\infty}\mu(A_j)=\infty$,~tak $\mu(\bigcap_{i=1}^\infty A_i)=0$.
\end{remark}

Z příkladů \ref{ex:lebegueova-mira-trivialni-priklady} a \ref{ex:lebegueova-mira-delka-intervalu} můžeme vidět,~že pro rozumně zvolené množiny zachycuje vnější Lebesgueova míra jejich intuitivní "velikost". V~případě intervalu odpovídá jeho délce,~v~případě diskrétní množiny je nulová a podobně např. pro obdélník lze ukázat,~že odpovídá jeho obsahu,~popř. pro kvádr jeho objemu.

Nyní se však nabízí jedna otázka. Čtenář by mohl již od chvíle,~kdy jsme zavedli pojem vnější Lebesgueovy míry (opět viz definice \ref{def:vnejsi-lebegueova-mira}) namítat,~co nás opravňuje nazývat zobrazení $\lambda_n^*$ mírou,~tj. ve smyslu definice \ref{def:prostor-s-mirou}. Jak víme,~že splňuje podmínku $\sigma$-aditivity? Na tuto otázku odpověď není zcela přímočará a vlastně není ani jednoduchá.

Bohužel pro libovolně zvolenou množinu $X$ a $\sigma$-algebru $\mathcal{A}$ v~případě vnější Lebesgueovy míry obecně neplatí vlastnost aditivity,~tedy existují množiny $A,B\in\mathcal{A}$,~takové,~že
\[\lambda_n^*(A\cup B)\neq\lambda_n^*(A)+\lambda_n^*(B).\]
Příklad takových množin využívá např. takzvaná \emph{Vitaliho konstrukce}\index{Vitaliho konstrukce},~se kterou přišel italský matematik \name{Giuseppe Vitali} (1875--1932) roku 1905,~využívající invariance vnější Lebesgueovy míry vůči posunutí,~tzn. $\lambda_n^*(x+A)=\lambda_n^*(A)$. \cite{OConnor2025} V~rámci tohoto textu se jí zde zabývat nebudeme,~avšak pro zájemce doporučuji zdroje \citep[str. 3]{Lukes2013} a \cite{Verner2025},~kde je tato konstrukce podrobněji rozepsána.

Je tedy potřeba se omezit na takové množiny,~kde je $\lambda_n^*$ aditivní. Existuje více způsobů,~jak lze charakterizovat takové množiny,~avšak my si zde uvedeme způsob,~se kterým přišel řecký matematik \name{Constantin Carathéodory} (1873--1950).
\begin{definition}[Lebesgueovská měřitelnost]\label{def:lebesgueovska-meritelnost}
    Množinu $A\subseteq\R^n$ nazveme (lebesgueovsky) měřitelnou\index{lebesgueovsky měřitelná množina},~pokud pro každou množinu $G$ platí
    \[\lambda_n^*(G)=\lambda_n^*(A\cap G)+\lambda_n^*(A\setminus G).\]
    Systém všech měřitelných množin v~$R^n$ značíme $\mathcal{L}^n$.  Pokud $A\in\mathcal{L}^n$,~pak číslo $\lambda_n(A)=\lambda_n^*$ nazýváme $n$-rozměrnou Lebesgueovou mírou\index{$n$-rozměrná Lebesgueova míra} množiny $A$.
\end{definition}

\todo{Musí být $G$ omezená (viz porovnání knih \emph{Real Analysis} a \emph{Measure and integral})?}