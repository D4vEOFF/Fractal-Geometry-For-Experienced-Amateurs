\section{Lebesgueova míra}\label{sec:lebesgueova-mira}

V předešlé sekci \ref{sec:prostory-s-mirou} jsme se povídali o pojmu \emph{míra} obecně a podívali jsme se na několik příkladů. Obecnou ideu měření "velikosti" lze založit např. na aproximaci obecné množiny pomocí spočetných sjednocení útvarů, jejichž "velikost" umíme jednoduše určit. V dalším textu se omezíme pouze na množinu $\R^n$.

Na zmíněné myšlence je postavena definice tzv. \emph{$d$-rozměrné Lebesgueovy míry}, kdy obecnou množinu budeme pokrývat pomocí \emph{kvádrů}. Připomeňme, že obecně $n$-rozměrným kvádrem $I$ rozumíme kartézský součin \emph{intervalů} $\langle a_1,b_1\rangle,\langle a_2,b_2\rangle,\ldots,\langle a_d,b_d\rangle\subseteq\R^n$, tj.
\[I=\prod_{i=1}^{d}\langle a_i,b_i\rangle=\langle a_1,b_1\rangle\times\langle a_2,b_2\rangle\times\dots\times\langle a_d,b_d\rangle\]
a jeho objem definujeme jako
\[\vol^n(I)=\prod_{i=1}^{d}(b_i-a_i).\]