\section{Systém iterovaných funkcí}\label{sec:ifs}

V předchozí části kapitoly viděli, jak lze fraktální objekty efektivně popisovat pomocí L-systémů, kde struktura vzniká paralelním přepisováním symbolů a jejich vizualizace se provádí prostřednictvím želví grafiky. Tento přístup nám umožňoval popsat určitou skupinu fraktálů, nicméně pro jiné fraktály by se nám hodil vhodnější popis jejich konstrukce. Např. Sierpińského trojúhelník lze popsat pomocí L-systému, nicméně v sekci \ref{sec:sobepodobnost} o soběpodobnosti jsme si jej záváděli spíše pomocí opakované aplikace určitých geometrických transformací (ač jsme neuvedli jejich explicitní vyjádření). Ty si později zavedeme jako tzv. \emph{systémy iterovaných funkcí}\index{systém iterovaných funkcí}.

Nejdříve se podíváme trochu více na matematickou podstatu. Hodně záležitostí jsme si již rozebrali v kapitole \ref{chapter:hausdorffuv-mp} o Hausdorffově metrickém prostoru, který zde bude hrát významnou roli, a dále na ně budeme navazovat.

