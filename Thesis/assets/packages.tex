%%% Hlavní soubor. Zde se definují základní parametry a~odkazuje se na ostatní části. %%%

%% Verze pro jednostranný tisk:
% Okraje: levý 40mm, pravý 25mm, horní a~dolní 25mm
% (ale pozor, LaTeX si sám přidává 1in)
\setlength\textwidth{145mm}
\setlength\textheight{247mm}
\setlength\oddsidemargin{15mm}
\setlength\evensidemargin{15mm}
\setlength\topmargin{0mm}
\setlength\headsep{0mm}
\setlength\headheight{0mm}
% \openright zařídí, aby následující text začínal na pravé straně knihy
\let\openright=\clearpage

%% Pokud tiskneme oboustranně:
% \documentclass[12pt,a4paper,twoside,openright]{report}
% \setlength\textwidth{145mm}
% \setlength\textheight{247mm}
% \setlength\oddsidemargin{14.2mm}
% \setlength\evensidemargin{0mm}
% \setlength\topmargin{0mm}
% \setlength\headsep{0mm}
% \setlength\headheight{0mm}
% \let\openright=\cleardoublepage

\usepackage[pdfa,colorlinks=false,urlcolor=black,bookmarksopen=true]{hyperref} 	% Záložky v~souboru

%% Vytváříme PDF/A-2u
\usepackage[a-2u]{pdfx}

%% Přepneme na českou sazbu a~fonty Latin Modern
\usepackage[czech]{babel}
\usepackage{lmodern}
\usepackage[T1]{fontenc}
\usepackage{textcomp}

%% Použité kódování znaků: obvykle latin2, cp1250 nebo utf8:
\usepackage[utf8]{inputenc}
\usepackage{csquotes}
\MakeOuterQuote{"}

%%% Další užitečné balíčky (jsou součástí běžných distribucí LaTeXu)
\usepackage{amsmath}        % rozšíření pro sazbu matematiky
\usepackage{amsfonts}       % matematické fonty
\usepackage{amsthm}         % sazba vět, definic apod.
\usepackage{amssymb}
\usepackage{bbding}         % balíček s nejrůznějšími symboly
			    % (čtverečky, hvězdičky, tužtičky, nůžtičky, ...)
\usepackage{bm}             % tučné symboly (příkaz \bm)
\usepackage{graphicx}       % vkládání obrázků
\usepackage{fancyvrb}       % vylepšené prostředí pro strojové písmo
\usepackage{indentfirst}    % zavede odsazení 1. odstavce kapitoly
\usepackage[numbers]{natbib}         % zajištuje možnost odkazovat na literaturu
			    % stylem AUTOR (ROK), resp. AUTOR [ČÍSLO]
\usepackage[nottoc]{tocbibind} % zajistí přidání seznamu literatury,
                            % obrázků a~tabulek do obsahu
\usepackage{icomma}         % inteligetní čárka v~matematickém módu
\usepackage{dcolumn}        % lepší zarovnání sloupců v~tabulkách
\usepackage{booktabs}       % lepší vodorovné linky v~tabulkách
\usepackage{paralist}       % lepší enumerate a~itemize
\usepackage{xcolor}         % barevná sazba
\usepackage{float}
\usepackage{ifthen}
\usepackage{cancel}
\usepackage{enumitem}
\usepackage{ragged2e}
\usepackage{needspace}		
\usepackage{wrapfig}		% plovoucí obrázky nalevo/napravo
\usepackage{pgfplots}
\usepackage{subcaption}
\pgfplotsset{compat=1.15}
\usepackage{mathrsfs}
\usetikzlibrary{arrows}

\justifying

\graphicspath{{./components/images/}}