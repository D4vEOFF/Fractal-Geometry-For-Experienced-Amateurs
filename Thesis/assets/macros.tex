%%% Tento soubor obsahuje definice různých užitečných maker a~prostředí %%%
%%% Další makra připisujte sem,~ať nepřekáží v~ostatních souborech.     %%%

%%% Drobné úpravy stylu

% Tato makra přesvědčují mírně ošklivým trikem LaTeX,~aby hlavičky kapitol
% sázel příčetněji a~nevynechával nad~nimi spoustu místa. Směle ignorujte.
% \makeatletter
% \def\@makechapterhead#1{
%  {\parindent \z@ \raggedright \normalfont
%   \Huge\bfseries \thechapter. #1
%   \par\nobreak
%   \vskip 20\p@
% }}
% \def\@makeschapterhead#1{
%  {\parindent \z@ \raggedright \normalfont
%   \Huge\bfseries #1
%   \par\nobreak
%   \vskip 20\p@
% }}
% \makeatother

\setlength{\parskip}{0.5em}
\setlength{\parindent}{0em}
\setlength{\cftbeforesecskip}{1.0ex}

% Toto makro definuje kapitolu,~která není očíslovaná,~ale je uvedena v~obsahu.
\def\chapwithtoc#1{
\chapter*{#1}
\addcontentsline{toc}{chapter}{#1}
}

% Trochu volnější nastavení dělení slov,~než je default.
\lefthyphenmin=2
\righthyphenmin=2

% Zapne černé "slimáky" na koncích řádků,~které přetekly,~abychom si
% jich lépe všimli.
\overfullrule=1mm

%%% Makra pro definice,~věty,~tvrzení,~příklady,~... (vyžaduje baliček amsthm)

\theoremstyle{plain}
\newtheorem{theorem}{Věta}[section]
\newtheorem{lemma}[theorem]{Lemma}
\newtheorem{proposition}[theorem]{Tvrzení}
\newtheorem{corollary}[theorem]{Důsledek}
\newtheorem*{proposition*}{Tvrzení}

\theoremstyle{definition}
\newtheorem{definition}[theorem]{Definice}
\newtheorem{example}[theorem]{Příklad}
\newtheorem{remark}[theorem]{Poznámka}
\newtheorem{convention}[theorem]{Úmluva}
\newtheorem{denoting}[theorem]{Značení}

%%% Prostředí pro důkazy

\renewenvironment{proof}[1][]{
  \par\medskip\noindent
  \textit{\ifthenelse{\equal{#1}{}}
  {Důkaz}
  {#1}}.
}{
\hspace*{\fill}$\qedsymbol$\par\medskip
}
\newenvironment{solution}[1][]{
  \par\medskip\noindent
  \textit{\ifthenelse{\equal{#1}{}}
  {Řešení}
  {#1}}.
}{
\hspace*{\fill}$\qedsymbol$\par\medskip
}

%%% Prostředí pro sazbu kódu,~případně vstupu/výstupu počítačových
%%% programů. (Vyžaduje balíček fancyvrb -- fancy verbatim.)

\DefineVerbatimEnvironment{code}{Verbatim}{fontsize=\small,~frame=single}

%%% Prostor reálných čísel,~přirozených čísel,~...
\newcommand{\R}{\mathbb{R}}
\newcommand{\C}{\mathbb{C}}
\newcommand{\N}{\mathbb{N}}
\newcommand{\Q}{\mathbb{Q}}
\newcommand{\Z}{\mathbb{Z}}

%%% Užitečné operátory pro statistiku a~pravděpodobnost
\DeclareMathOperator{\pr}{\textsf{P}}
\DeclareMathOperator{\E}{\textsf{E}\,}
\DeclareMathOperator{\var}{\textrm{var}}
\DeclareMathOperator{\sd}{\textrm{sd}}

%%% Goniometrické a cyklometrické funkce
\DeclareMathOperator{\cotg}{cotg}
\DeclareMathOperator{\arccotg}{arccotg}
\DeclareMathOperator{\arctg}{arctg}
\DeclareMathOperator{\tg}{tg}

%%% Dimenze
\DeclareMathOperator{\dimL}{dim_L}  % Lebesgueova pokrývací dimenze 
\DeclareMathOperator{\dimB}{dim_B}  % Box-counting dimenze
\DeclareMathOperator{\dimH}{dim_H}  % Hausdorffova dimenze

\DeclareMathOperator{\vol}{vol}
\DeclareMathOperator{\diam}{diam}
\newcommand{\upperdimB}{\ensuremath{\overline{\dim}_\textup{B}}}
\newcommand{\lowerdimB}{\ensuremath{\underline{\dim}_\textup{B}}}

%%% Míry, metriky a prostory
\newcommand{\hausdorffmetric}{\ensuremath{\varrho_\textup{H}}}
\newcommand{\hausdorffmeasure}[1]{\ensuremath{\mathcal{H}}^{#1}}
\newcommand{\hausdorffdeltameasure}[2]{\ensuremath{\mathcal{H}}_{#2}^{#1}}

\newcommand{\lebesguemeasure}[1]{\ensuremath{\lambda_#1}}
\newcommand{\lebesgueoutermeasure}[1]{\ensuremath{\lebesguemeasure{#1}^*}}

\newcommand{\hyperspace}{\ensuremath{\mathbb{H}}}
\newcommand{\borelsigmaalgebra}{\ensuremath{\mathscr{B}}}

\newcommand{\continuousfuncspace}{\ensuremath{\mathcal{C}}}

%%% Příkaz pro transpozici vektoru/matice
\newcommand{\T}[1]{#1^\top}
\newcommand{\mat}[1]{\ensuremath{\mathbf{#1}}}

%%% Vychytávky pro matematiku
\newcommand{\goto}{\rightarrow}
\newcommand{\gotop}{\stackrel{P}{\longrightarrow}}
\newcommand{\maon}[1]{o(n^{#1})}
\newcommand{\abs}[1]{\left|{#1}\right|}
\newcommand{\dint}{\int_0^\tau\!\!\int_0^\tau}
\newcommand{\isqr}[1]{\frac{1}{\sqrt{#1}}}

\newcommand{\napprox}{\not\approx}
\newcommand{\norm}[1]{|| #1 ||}
\newcommand{\dotprod}[2]{\left(#1|#2\right)}
\newcommand{\compl}[1]{\ensuremath{#1^\complement}}
\newcommand{\mapping}[3]{#1:#2 \rightarrow #3}
\newcommand{\mapto}[3]{#1:#2 \mapsto #3}
% \newcommand{\powset}[1]{\mathcal{P}(#1)}
\newcommand{\powset}[1]{\ensuremath{\mathscr{P}(#1)}}
\newcommand{\solutions}[1]{\left[K=\left\{#1\right\}\right]}
\newcommand{\set}[1]{\left\{#1\right\}}
\newcommand{\admid}{\;\middle\vert\;}
\newcommand{\sizeof}[1]{\left|#1\right|}
\newcommand{\dx}[1][]{
   \ifthenelse{\equal{#1}{}}%
      {\ensuremath{\,\mathrm{d}x}}%
      {\ensuremath{\,\mathrm{d}#1}}%
} % diferenciál
\renewcommand{\emptyset}{\varnothing}

\newcommand{\ZF}{\textsf{ZF}} % Zermelova-Fraenkelova teorie množin
\newcommand{\PA}{\textsf{PA}} % Peanova aritmetika

% Čítače
\newcommand{\createcnt}[2][0]{\newcounter{#2}\setcounter{#2}{#1}}
\newcommand{\printcnt}[1]{\arabic{#1}}
\newcommand{\printnstepcnt}[1]{\stepcounter{#1}\arabic{#1}}

%%% Vychytávky pro tabulky
\newcommand{\pulrad}[1]{\raisebox{1.5ex}[0pt]{#1}}
\newcommand{\mc}[1]{\multicolumn{1}{c}{#1}}

% Celá jména
\newcommand{\name}[1]{\mbox{\textsc{#1}}}

% Logické operátory
\renewcommand{\implies}{\Longrightarrow}
\renewcommand{\impliedby}{\Longleftarrow}
\renewcommand{\iff}{\Longleftrightarrow}

% Metrické prostory
\newcommand{\interior}[1]{\ensuremath{#1^\circ}}
\newcommand{\boundary}[1]{\ensuremath{\partial #1}}
\newcommand{\closure}[1]{\ensuremath{\overline{#1}}}

% Cesty
\newcommand{\chapterpath}[1]{components/ch#1}
\newcommand{\sectionpath}[1]{components/ch#1/sections}
\newcommand{\literaturepath}{components/literature}
\newcommand{\appendixpath}{components/appendix}

% TODO
\newcommand{\todo}[1]{\textcolor{red}{(\noindent TODO: #1)}}

% Rightsided note
\newcommand{\rightnote}[1]{\hspace*{\fill} $\triangleleft$ \textit{#1}}
\newcommand{\mathrightnote}[1]{\ensuremath{\tag*{$\triangleleft$ \textit{#1}}}}

% Images
\newcommand{\subfigwidth}{6cm}        % šířka okénka pro "podobrázky"
\newcommand{\wrappedfigwidth}{6cm}
\newcommand{\fullhd}{0.2}             % měřítko Full HD obrázku
\newcommand{\normalipe}{0.8}          % standardní měřítko IPE obrázku
\newcommand{\fractalscale}{0.3}       % měřítko obrázku softwarem vygenerovaného fraktálu