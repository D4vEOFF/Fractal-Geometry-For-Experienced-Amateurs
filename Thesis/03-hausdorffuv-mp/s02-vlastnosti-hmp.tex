\section{Kompaktní množiny a konvergence}\label{sec:konvergence-hmp}

\begin{theorem}\label{thm:konvergence-hmp}
    Nechť $A_1,A_2,\ldots$ je nerostoucí posloupnost množin, kde $A_i\in\hyperspace(X)$ pro každé $i\in\N$. Pak posloupnost $A_1,A_2,\ldots$ konverguje k množině
    \[A=\bigcap_{i=1}^\infty A_i\]
    v Hausdorffově metrice.
\end{theorem}

V souvislosti s konvergencí v prostoru $(\hyperspace(X),\hausdorffmetric)$ nebude na škodu si připomenout jedno známé související tvrzení z matematické analýzy.
\begin{theorem}[Cantorova]\label{thm:cantor}
    Následující tvrzení jsou ekvivalentní:
    \begin{enumerate}[label=(\roman*)]
        \item\label{thm:cantor-uplnost} $(X,\varrho)$ je úplný metrický prostor.
        \item\label{thm:cantor-nekl-posl-mnozin} Je-li $A_1,A_2,\ldots$ neklesající posloupnost uzavřených množin, kde $A_i\subseteq X$ pro každé $i\in\N$, takových, že $\diam{A_i}\to 0$, pak existuje $x\in X$ splňující
        \[\bigcap_{i=1}^\infty A_i=\set{x}.\]
    \end{enumerate}
\end{theorem}
Věta \ref{thm:cantor} je v podstatě rozšířením Cantorova principu vnořených intervalů v $\R$. Speciálně z toho vyplývá, že i v Hausdorffově metrickém prostoru platí podmínka \ref{thm:cantor-nekl-posl-mnozin}, neboť z věty \ref{thm:uplnost-hmp} víme, že $(\hyperspace(X),\hausdorffmetric)$ tvoří úplný metrický prostor.